\documentclass[]{article}
\usepackage{lmodern}
\usepackage{amssymb,amsmath}
\usepackage{ifxetex,ifluatex}
\usepackage{fixltx2e} % provides \textsubscript
\ifnum 0\ifxetex 1\fi\ifluatex 1\fi=0 % if pdftex
  \usepackage[T1]{fontenc}
  \usepackage[utf8]{inputenc}
\else % if luatex or xelatex
  \ifxetex
    \usepackage{mathspec}
  \else
    \usepackage{fontspec}
  \fi
  \defaultfontfeatures{Ligatures=TeX,Scale=MatchLowercase}
    \setmainfont[]{Times New Roman}
    \setsansfont[]{Times New Roman}
\fi
% use upquote if available, for straight quotes in verbatim environments
\IfFileExists{upquote.sty}{\usepackage{upquote}}{}
% use microtype if available
\IfFileExists{microtype.sty}{%
\usepackage{microtype}
\UseMicrotypeSet[protrusion]{basicmath} % disable protrusion for tt fonts
}{}
\usepackage[margin=1in]{geometry}
\usepackage{hyperref}
\hypersetup{unicode=true,
            pdftitle={Text Classification (Newspaper) using Naive Bayes},
            pdfauthor={Sheryl Mathew (11627236)},
            pdfborder={0 0 0},
            breaklinks=true}
\urlstyle{same}  % don't use monospace font for urls
\usepackage{color}
\usepackage{fancyvrb}
\newcommand{\VerbBar}{|}
\newcommand{\VERB}{\Verb[commandchars=\\\{\}]}
\DefineVerbatimEnvironment{Highlighting}{Verbatim}{commandchars=\\\{\}}
% Add ',fontsize=\small' for more characters per line
\usepackage{framed}
\definecolor{shadecolor}{RGB}{248,248,248}
\newenvironment{Shaded}{\begin{snugshade}}{\end{snugshade}}
\newcommand{\KeywordTok}[1]{\textcolor[rgb]{0.13,0.29,0.53}{\textbf{#1}}}
\newcommand{\DataTypeTok}[1]{\textcolor[rgb]{0.13,0.29,0.53}{#1}}
\newcommand{\DecValTok}[1]{\textcolor[rgb]{0.00,0.00,0.81}{#1}}
\newcommand{\BaseNTok}[1]{\textcolor[rgb]{0.00,0.00,0.81}{#1}}
\newcommand{\FloatTok}[1]{\textcolor[rgb]{0.00,0.00,0.81}{#1}}
\newcommand{\ConstantTok}[1]{\textcolor[rgb]{0.00,0.00,0.00}{#1}}
\newcommand{\CharTok}[1]{\textcolor[rgb]{0.31,0.60,0.02}{#1}}
\newcommand{\SpecialCharTok}[1]{\textcolor[rgb]{0.00,0.00,0.00}{#1}}
\newcommand{\StringTok}[1]{\textcolor[rgb]{0.31,0.60,0.02}{#1}}
\newcommand{\VerbatimStringTok}[1]{\textcolor[rgb]{0.31,0.60,0.02}{#1}}
\newcommand{\SpecialStringTok}[1]{\textcolor[rgb]{0.31,0.60,0.02}{#1}}
\newcommand{\ImportTok}[1]{#1}
\newcommand{\CommentTok}[1]{\textcolor[rgb]{0.56,0.35,0.01}{\textit{#1}}}
\newcommand{\DocumentationTok}[1]{\textcolor[rgb]{0.56,0.35,0.01}{\textbf{\textit{#1}}}}
\newcommand{\AnnotationTok}[1]{\textcolor[rgb]{0.56,0.35,0.01}{\textbf{\textit{#1}}}}
\newcommand{\CommentVarTok}[1]{\textcolor[rgb]{0.56,0.35,0.01}{\textbf{\textit{#1}}}}
\newcommand{\OtherTok}[1]{\textcolor[rgb]{0.56,0.35,0.01}{#1}}
\newcommand{\FunctionTok}[1]{\textcolor[rgb]{0.00,0.00,0.00}{#1}}
\newcommand{\VariableTok}[1]{\textcolor[rgb]{0.00,0.00,0.00}{#1}}
\newcommand{\ControlFlowTok}[1]{\textcolor[rgb]{0.13,0.29,0.53}{\textbf{#1}}}
\newcommand{\OperatorTok}[1]{\textcolor[rgb]{0.81,0.36,0.00}{\textbf{#1}}}
\newcommand{\BuiltInTok}[1]{#1}
\newcommand{\ExtensionTok}[1]{#1}
\newcommand{\PreprocessorTok}[1]{\textcolor[rgb]{0.56,0.35,0.01}{\textit{#1}}}
\newcommand{\AttributeTok}[1]{\textcolor[rgb]{0.77,0.63,0.00}{#1}}
\newcommand{\RegionMarkerTok}[1]{#1}
\newcommand{\InformationTok}[1]{\textcolor[rgb]{0.56,0.35,0.01}{\textbf{\textit{#1}}}}
\newcommand{\WarningTok}[1]{\textcolor[rgb]{0.56,0.35,0.01}{\textbf{\textit{#1}}}}
\newcommand{\AlertTok}[1]{\textcolor[rgb]{0.94,0.16,0.16}{#1}}
\newcommand{\ErrorTok}[1]{\textcolor[rgb]{0.64,0.00,0.00}{\textbf{#1}}}
\newcommand{\NormalTok}[1]{#1}
\usepackage{graphicx,grffile}
\makeatletter
\def\maxwidth{\ifdim\Gin@nat@width>\linewidth\linewidth\else\Gin@nat@width\fi}
\def\maxheight{\ifdim\Gin@nat@height>\textheight\textheight\else\Gin@nat@height\fi}
\makeatother
% Scale images if necessary, so that they will not overflow the page
% margins by default, and it is still possible to overwrite the defaults
% using explicit options in \includegraphics[width, height, ...]{}
\setkeys{Gin}{width=\maxwidth,height=\maxheight,keepaspectratio}
\IfFileExists{parskip.sty}{%
\usepackage{parskip}
}{% else
\setlength{\parindent}{0pt}
\setlength{\parskip}{6pt plus 2pt minus 1pt}
}
\setlength{\emergencystretch}{3em}  % prevent overfull lines
\providecommand{\tightlist}{%
  \setlength{\itemsep}{0pt}\setlength{\parskip}{0pt}}
\setcounter{secnumdepth}{0}
% Redefines (sub)paragraphs to behave more like sections
\ifx\paragraph\undefined\else
\let\oldparagraph\paragraph
\renewcommand{\paragraph}[1]{\oldparagraph{#1}\mbox{}}
\fi
\ifx\subparagraph\undefined\else
\let\oldsubparagraph\subparagraph
\renewcommand{\subparagraph}[1]{\oldsubparagraph{#1}\mbox{}}
\fi

%%% Use protect on footnotes to avoid problems with footnotes in titles
\let\rmarkdownfootnote\footnote%
\def\footnote{\protect\rmarkdownfootnote}

%%% Change title format to be more compact
\usepackage{titling}

% Create subtitle command for use in maketitle
\newcommand{\subtitle}[1]{
  \posttitle{
    \begin{center}\large#1\end{center}
    }
}

\setlength{\droptitle}{-2em}

  \title{Text Classification (Newspaper) using Naive Bayes}
    \pretitle{\vspace{\droptitle}\centering\huge}
  \posttitle{\par}
    \author{Sheryl Mathew (11627236)}
    \preauthor{\centering\large\emph}
  \postauthor{\par}
      \predate{\centering\large\emph}
  \postdate{\par}
    \date{20 October, 2018}

\usepackage{booktabs}
\usepackage{longtable}
\usepackage{array}
\usepackage{multirow}
\usepackage[table]{xcolor}
\usepackage{wrapfig}
\usepackage{float}
\usepackage{colortbl}
\usepackage{pdflscape}
\usepackage{tabu}
\usepackage{threeparttable}
\usepackage{threeparttablex}
\usepackage[normalem]{ulem}
\usepackage{makecell}

\begin{document}
\maketitle

\subsection{Data Collection}\label{data-collection}

\begin{Shaded}
\begin{Highlighting}[]
\KeywordTok{library}\NormalTok{(jsonlite)}
\KeywordTok{library}\NormalTok{(dplyr)}
\end{Highlighting}
\end{Shaded}

\begin{verbatim}
## 
## Attaching package: 'dplyr'
\end{verbatim}

\begin{verbatim}
## The following objects are masked from 'package:stats':
## 
##     filter, lag
\end{verbatim}

\begin{verbatim}
## The following objects are masked from 'package:base':
## 
##     intersect, setdiff, setequal, union
\end{verbatim}

\begin{Shaded}
\begin{Highlighting}[]
\KeywordTok{library}\NormalTok{(kableExtra) }

\NormalTok{api =}\StringTok{ 'https://content.guardianapis.com/search?'}
\NormalTok{api_key =}\StringTok{ '0e9a8c4e-1866-4503-a9eb-70e391c499c4'}
\NormalTok{page_size =}\StringTok{ '200'}
\NormalTok{pages =}\StringTok{ }\KeywordTok{c}\NormalTok{(}\DecValTok{1}\OperatorTok{:}\DecValTok{5}\NormalTok{)}
\NormalTok{queries =}\StringTok{ }\KeywordTok{c}\NormalTok{(}\StringTok{'business'}\NormalTok{,}\StringTok{'sports'}\NormalTok{,}\StringTok{'entertainment'}\NormalTok{,}\StringTok{'economy'}\NormalTok{,}\StringTok{'politics'}\NormalTok{,}\StringTok{'science'}\NormalTok{,}\StringTok{'health'}\NormalTok{,}\StringTok{'art'}\NormalTok{,}\StringTok{'technology'}\NormalTok{,}\StringTok{'crime'}\NormalTok{)}

\NormalTok{news_data =}\StringTok{ }\KeywordTok{data.frame}\NormalTok{()}
\NormalTok{data_count =}\StringTok{ }\KeywordTok{list}\NormalTok{()}
\NormalTok{sum_data_count =}\StringTok{ }\KeywordTok{list}\NormalTok{()}


\ControlFlowTok{for}\NormalTok{(query }\ControlFlowTok{in}\NormalTok{ queries)\{}
   \ControlFlowTok{for}\NormalTok{ (page }\ControlFlowTok{in}\NormalTok{ pages)\{}
\NormalTok{      url =}\StringTok{ }\KeywordTok{paste}\NormalTok{(api, }\StringTok{'q='}\NormalTok{, query, }\StringTok{'&page-size='}\NormalTok{, page_size, }\StringTok{'&page='}\NormalTok{, page, }\StringTok{'&api-key='}\NormalTok{, api_key,}
            \StringTok{'&show-fields=body'}\NormalTok{, }\DataTypeTok{sep =} \StringTok{""}\NormalTok{)}
\NormalTok{      json =}\StringTok{ }\KeywordTok{fromJSON}\NormalTok{(url)}
\NormalTok{      body =}\StringTok{ }\KeywordTok{as.data.frame}\NormalTok{(json}\OperatorTok{$}\NormalTok{response}\OperatorTok{$}\NormalTok{results}\OperatorTok{$}\NormalTok{fields)}
\NormalTok{      data =}\StringTok{ }\KeywordTok{as.data.frame}\NormalTok{(json}\OperatorTok{$}\NormalTok{response}\OperatorTok{$}\NormalTok{results)      }
\NormalTok{      data =}\StringTok{ }\KeywordTok{subset}\NormalTok{(data, }\DataTypeTok{select =} \OperatorTok{-}\KeywordTok{c}\NormalTok{(fields))}
\NormalTok{      data =}\StringTok{ }\KeywordTok{cbind}\NormalTok{(data,body)}
\NormalTok{      data_count =}\StringTok{ }\KeywordTok{append}\NormalTok{(data_count,}\KeywordTok{nrow}\NormalTok{(data))}
\NormalTok{      news_data =}\StringTok{ }\KeywordTok{rbind}\NormalTok{(news_data,data)}
\NormalTok{  \}}


\NormalTok{  select_data=}\KeywordTok{select}\NormalTok{(data,}\KeywordTok{c}\NormalTok{(}\StringTok{"sectionName"}\NormalTok{,}\StringTok{"body"}\NormalTok{))}
  \KeywordTok{print}\NormalTok{(}\KeywordTok{kable}\NormalTok{(}\KeywordTok{head}\NormalTok{(select_data), }\DataTypeTok{format =} \StringTok{"latex"}\NormalTok{, }\DataTypeTok{booktabs =}\NormalTok{ T,}
          \DataTypeTok{caption=}\KeywordTok{paste}\NormalTok{(}\StringTok{"Table containing"}\NormalTok{,}\KeywordTok{toupper}\NormalTok{(query),}\StringTok{" News"}\NormalTok{)) }\OperatorTok
\StringTok{      }\KeywordTok{kable_styling}\NormalTok{(}\DataTypeTok{latex_options =} \KeywordTok{c}\NormalTok{(}\StringTok{"striped"}\NormalTok{,}\StringTok{"hold_position"}\NormalTok{,}\StringTok{"scale_down"}\NormalTok{)))}
  \KeywordTok{cat}\NormalTok{(}\StringTok{"}\CharTok{\textbackslash{}n}\StringTok{"}\NormalTok{)}
  
\NormalTok{  sum_data_count =}\StringTok{ }\KeywordTok{append}\NormalTok{(sum_data_count,}\KeywordTok{sum}\NormalTok{(}\KeywordTok{as.integer}\NormalTok{(data_count)))}
\NormalTok{  data_count =}\StringTok{ }\DecValTok{0}
\NormalTok{\}}
\end{Highlighting}
\end{Shaded}

\rowcolors{2}{gray!6}{white}

\begin{table}[!h]

\caption{\label{tab:data_collection}Table containing BUSINESS  News}
\centering
\resizebox{\linewidth}{!}{
\begin{tabular}[t]{ll}
\hiderowcolors
\toprule
sectionName & body\\
\midrule
\showrowcolors
Environment & <p> </p> <p>It’s off-season at the coast, but you would only know by the bite in the wind. The sky is faultlessly blue, the sunlight golden, and parking is still a squeeze. Visitors flock year-round to Formby’s <a href="https://www.nationaltrust.org.uk/formby/trails/formby-red-squirrel-walk">patch of pine forest</a>, separated from the Irish Sea by soft white sand dunes. While a determined few have brought their buckets and spades, most of us are here for the wildlife.</p> <p>Just steps from the car park you can see red squirrels, actively foraging throughout the seasons. It’s a great spot to bring nature-loving friends for a close encounter, but my last visit in 2009 was unexpectedly bleak. The previous year, Formby had been hit by the deadly <a href="https://www.theguardian.com/environment/2013/nov/22/red-squirrels-poxvirus-resistance">squirrelpox</a> and lost 85\% of the population. But they have made a remarkable recovery, and are almost back to their pre-pox numbers. </p> <aside class="element element-rich-link element--thumbnail"> <p> <span>Related: </span><a href="https://www.theguardian.com/environment/2013/nov/22/red-squirrels-poxvirus-resistance">Red squirrels showing resistance to poxvirus</a> </p> </aside>  <p>Within minutes of joining the dedicated squirrel trail, I am delighted to see several small shapes scampering between the regimented lines of the conifer plantation. Autumn is a particularly busy time for them as they cache the seasonal bounty of nuts and seeds for winter sustenance.</p> <p>Living on the edge of town and visited daily by adoring fans, these are arguably urban squirrels. These reds are not bothered by excitable dogs, brightly clothed toddlers or a small amateur paparazzi scrum on the path. The teenage girls with a selfie stick elicit a few warning squeaks, though, before their Instagram crush zips up a tree away from the attention.</p>  <figure class="element element-image" data-media-id="62f0387e0023ff8aa5a876e9377516eade82aceb"> <img src="https://media.guim.co.uk/62f0387e0023ff8aa5a876e9377516eade82aceb/0\_0\_5050\_3356/1000.jpg" alt="A red squirrel meets a photographer at Formby." width="1000" height="665" class="gu-image" /> <figcaption> <span class="element-image\_\_caption">A red squirrel meets a photographer at Formby.</span> <span class="element-image\_\_credit">Photograph: Wildlife in Pixels/Alamy</span> </figcaption> </figure>  <p>Red squirrels are undeniably worthy pin-ups, from their twitchy blond noses to their fluffy auburn tails. And their ear tufts, which start to grow longer at this time of year, ready to provide extra protection from the cold. </p> <p>In quiet moments, I hear the clatter of claws on crisp pine bark before I can spot the squirrels whipping around tree trunks. Where they briefly pause in the pine needles, I admire their varied colours, from bright ginger through to sooty brick red. The National Trust is celebrating <a draggable="true" href="https://www.nationaltrust.org.uk/formby">50 years of caring for the area</a> and its charismatic residents. Formby’s reds have truly enduring appeal.</p> <ul> <li>This article was amended on 14 November 2017 to correct the location of Formby.</li> </ul> <p><em><a href="https://twitter.com/GdnCountryDiary">Follow Country diary on Twitter: @gdncountrydiary</a></em></p> <p><br><br></p>\\
Australia news & <p>Malcolm Turnbull has blamed the media and his political opponents for portraying the China-Australia relationship as troubled, claiming issues between the countries are being settled with mutual respect.</p> <p>Turnbull made the comments at the Australia China Business Council on Tuesday, amid tension about <a href="https://www.smh.com.au/politics/federal/australia-will-compete-with-china-to-save-pacific-sovereignty-says-bishop-20180617-p4zm1h.html">Australian competition with China for influence in the South Pacific</a> and concern from China that it is being <a href="https://www.theguardian.com/australia-news/2017/dec/09/china-says-turnbulls-remarks-have-poisoned-the-atmosphere-of-relations">targeted by the Coalition’s foreign interference package</a>.</p> <p><a href="https://www.theguardian.com/world/guardian-australia-morning-mail/2014/jun/24/-sp-guardian-australias-morning-mail-subscribe-by-email">• Sign up to receive the top stories every morning</a></p>  <figure class="element element-embed" data-alt="Sign up to receive the top stories every morning">  <iframe src="https://www.theguardian.com/email/form/plaintone/4148" height="52px" data-form-title="Sign up for Guardian Australia's Morning Mail" data-form-description="Get our editors' pick of the biggest headlines every weekday" scrolling="no" seamless frameborder="0" class="iframed--overflow-hidden email-sub\_\_iframe js-email-sub\_\_iframe js-email-sub\_\_iframe--article" data-form-success-desc="Thanks for signing up"></iframe> </figure>  <p>Debate in the Coalition over <a href="https://www.theguardian.com/technology/2018/jun/14/huawei-denies-being-locked-out-of-bidding-to-help-build-5g-network">Huawei’s bid to participate in the 5G network</a> is another flashpoint in relations, with a warning on Tuesday from the chairman of the top parliamentary foreign affairs committee that Huawei should be blocked and foreign minister Julie Bishop reasserting Australia’s right to act on security advice to do so.</p> <p>Turnbull said “in the media and sometimes you’ll see from politicians … a lot more negativity presented than is actually the case”.</p> <p>The Australian prime minister suggested it was “important to reinforce the reality” that Australia’s relationship with China is strong, citing individual relationships forged by businesses trading with China and the 1.2m Australians of Chinese descent.</p> <p>Several questioners provided anecdotal evidence that Chinese businesspeople were concerned about the direction of the Australian government. One suggested the perception was that “the relationship does not match the trade between the two countries”.<br></p> <p>Turnbull responded that “the relationship is very strong”.</p> <p>“It’s important not to be distracted by media and political commentary which is often designed to highlight difference, friction … or possibly to even accentuate friction,” he said.</p> <aside class="element element-rich-link element--thumbnail"> <p> <span>Related: </span><a href="https://www.theguardian.com/australia-news/2018/jun/19/penny-wong-warns-against-racial-fault-lines-from-past-betraying-china-ties">Penny Wong warns against 'racial fault lines from past' betraying China ties</a> </p> </aside>  <p>Turnbull’s upbeat assessment also <a href="https://www.theguardian.com/world/2018/apr/12/turnbull-says-tension-with-china-has-risen-since-foreign-influence-row">contrasted with his frank admission in April</a> that there is “a degree of tension” in the China-Australia relationship due to the Coalition’s foreign interference laws, which did not rate a mention in his speech.</p> <p>Earlier on Tuesday <a href="https://www.theguardian.com/world/2018/jun/19/less-bias-and-bigotry-needed-in-australian-sino-relations-says-china-ambassador">the Chinese ambassador called for Australia</a> and<strong> </strong>China to harbour less “bias and bigotry” towards each other, warning that no one was benefiting from the current “cold war mentality”. </p> <p>The Turnbull government has also <a href="https://www.theguardian.com/australia-news/2018/feb/12/kevin-rudd-accuses-turnbull-government-of-anti-chinese-jihad">been accused of an “anti-Chinese jihad”</a> by the former Labor prime minister Kevin Rudd over its foreign interference laws.</p> <p>Despite the prime minister’s effort to sheet home blame to the media, the attorney general has <a href="https://www.theguardian.com/media/2018/jun/08/christian-porter-defends-media-against-chinese-diplomats-claim-of-fake-news">recently defended the accuracy of Australian reporting</a> of foreign influence after accusations from a Chinese diplomat that it “fabricates stories”.</p> <p>On Tuesday the chair of the parliamentary joint committee on foreign affairs and trade, Liberal senator David Fawcett, urged colleagues in the Coalition party room to ban Chinese communications giant Huawei from building Australia’s next 5G networks.</p> <p>Fawcett told Guardian Australia that Australia “needs to proceed carefully” and engage with its partners “with our eyes open”.</p> <p>“I welcome trade with China, but sometimes they don’t play by the same rules ... and where there are areas that potentially create a risk, we need to take appropriate measures,” he said.</p> <p>Fawcett cited legal requirements in China that could compel individuals and companies to cooperate with government directives, including for intelligence work.</p> <p>On Thursday <a href="https://www.theguardian.com/technology/2018/jun/14/huawei-denies-being-locked-out-of-bidding-to-help-build-5g-network">Huawei’s Australian chairman, John Lord</a>, told the ABC the company was still in talks with the Turnbull government about participating in the 5G wireless network and he had not been advised of any government decision to block its participation on security grounds.</p> <p>Asked at the business council how Australian companies could safely engage with Chinese companies such as Huawei, Bishop said the government worked “very hard to get the balance right” between innovation and security.</p> <p>“In a globalised world the flow of people, capital and ideas is unstoppable,” she said. “But there are times when the Australian government has a responsibility to look at security and look at national interest issues.</p> <aside class="element element-rich-link element--thumbnail"> <p> <span>Related: </span><a href="https://www.theguardian.com/technology/2018/jun/14/huawei-denies-being-locked-out-of-bidding-to-help-build-5g-network">Huawei denies being locked out of bidding to help build 5G network</a> </p> </aside>  <p>“This is not directed at China, this is what we do in relation to all matters where the Australian government has a say … we ensure we take into account national security concerns, the advice of our security experts and act accordingly.”</p> <p>Bishop praised Australia’s comprehensive strategic partnership with China, noting the two did not always agree but what mattered was how they resolved those differences.</p> <p>Bishop added that Australia also disagreed on policy matters with the US, citing Australia’s opposition to protectionism, the US withdrawal from the Paris climate agreement and the Iran nuclear deal.</p> <p>Bishop credited China for its role putting pressure on North Korea to denuclearise, adding that “time will tell” whether the historic meeting between Kim Jong-Un and the US president Donald Trump would produce a lasting peace and denuclearisation.</p> <p>The trade minister, Steve Ciobo, downplayed anxiety about the Australia-China relationship, suggesting there was “a lot of navel-gazing going on” and pointing to \$180bn of trade between the two nations.</p>\\
Business & <div id="block-5af34032e4b08031c378ff57" class="block" data-block-contributor=""> <p class="block-time published-time"> <time datetime="2018-05-09T18:53:50.653Z">7.53pm <span class="timezone">BST</span></time> </p>    <div class="block-elements">  <p><strong>And finally, here’s Debbie Carson on today’s oil price spike:</strong></p> <blockquote class="quoted"> <p>Crude oil prices rose to three-and-a-half-year highs following the news that the Trump administration revoked the nuclear deal with Iran. </p> <p>Brent crude oil prices, the global benchmark, and US West Texas Intermediate rallied above \$77 and \$71 per barrel, respectively, in the aftermath of the announcement.<br></p> </blockquote>  <figure class="element element-tweet" data-canonical-url="https://twitter.com/PenkethRiviera/status/994280706426769422">  <blockquote class="twitter-tweet"><p lang="en" dir="ltr">Brent Crude <a href="https://t.co/wCjyw2gfOL">https://t.co/wCjyw2gfOL</a> <a href="https://twitter.com/hashtag/oilprices?src=hash\&amp;ref\_src=twsrc\%5Etfw">\#oilprices</a> <a href="https://t.co/mJQooUpPwB">pic.twitter.com/mJQooUpPwB</a></p>\&mdash; Penketh Riviera (@PenkethRiviera) <a href="https://twitter.com/PenkethRiviera/status/994280706426769422?ref\_src=twsrc\%5Etfw">May 9, 2018</a></blockquote>  </figure>  <blockquote class="quoted"> <p>The rise came after the treasury secretary, Steven Mnuchin, told reporters he did not expect a major oil price hikes because other countries would increase output to offset such losses.</p> <p>Although Donald Trump’s decision to withdraw from the Joint Comprehensive Plan of Action was not a surprise, reinstating all US nuclear-related sanctions was more than expected, said Barclays analysts in a research note.</p> <p>Bart Melek, global head of commodity strategy at TD Securities, said this announcement comes at a time where there are increased geopolitical tensions in the Middle East and global crude oil inventories are normalizing after being in a supply glut for the past few years. In the US inventory levels are now below the five-year average.</p> <p>“There’s a broad consensus that [supplies are] going to rebalance and tighten up, but you throw the possibility of disruptions of flows from Iran, and the markets started to worry,” he said.</p> </blockquote> <aside class="element element-rich-link"> <p> <span>Related: </span><a href="https://www.theguardian.com/business/2018/may/09/oil-prices-hit-three-and-a-half-year-high-after-us-exits-from-iran-deal">Oil prices hit three-and-a-half-year high after US exits from Iran deal</a> </p> </aside>  <p><strong><em>Goodnight! GW</em></strong></p> </div>   </div> <div id="block-5af33ef2e4b08031c378ff4e" class="block" data-block-contributor=""> <p class="block-time published-time"> <time datetime="2018-05-09T18:34:46.459Z">7.34pm <span class="timezone">BST</span></time> </p>    <div class="block-elements">  <p>My colleague Phillip Inman has written a Q\&amp;A about the rising oil price.</p> <p>Here’s a flavour:</p> <p><strong>What does the nuclear deal have to do with the oil price? <br></strong></p> <blockquote class="quoted"> <p>When Iran pledged to limit its nuclear ambitions to civil energy production under the deal with the P5+1 group of world powers – the US, UK, France, China, Russia and Germany – sanctions were lifted on its oil exports, giving a significant boost to global oil supplies.</p> </blockquote> <p><strong>How big is Iran’s contribution to global supplies? </strong></p> <blockquote class="quoted"> <p>Iran is the third largest producer in the Organisation of the Petroleum Exporting Countries (Opec), which makes it a heavy hitter in the production of oil. It produces 2.5m barrels a day, equal to about 3\% of global demand. Trump’s threat to reimpose sanctions and effectively keep up to half of Iranian oil in the ground has hit the oil price since he was elected in November 2016. The price of Brent crude is now up 50\% year on year.</p> </blockquote> <p>Here’s the full piece:</p> <aside class="element element-rich-link"> <p> <span>Related: </span><a href="https://www.theguardian.com/business/2018/may/09/why-are-oil-prices-soaring-as-us-exits-iran-nuclear-deal">Why are oil prices soaring as US exits Iran nuclear deal?</a> </p> </aside> </div>   </div> <div id="block-5af32f91e4b0591cf9f646f5" class="block" data-block-contributor=""> <p class="block-time published-time"> <time datetime="2018-05-09T17:33:13.003Z">6.33pm <span class="timezone">BST</span></time> </p>    <div class="block-elements">  <p>The rally in oil stocks has driven Britain’s blue-chip stock index up almost 100 points today.</p> <p>The FTSE 100 closed 96 points higher at 7,662 tonight, a gain of 1.3\%, and its highest level in over three months.</p> <p>David Madden of CMC Markets says:</p> <blockquote class="quoted"> <p>The FTSE 100 is the best performing major European index today thanks to the rally in oil stocks. The British equity benchmark has a disproportionally large exposure to the energy sector, and the pickup from <strong>BP</strong> and <strong>Royal Dutch Shell</strong> has made it the standout market in Europe. The US withdrawal from the Iranian nuclear deal has sent the oil market to new multi-year highs, and the major oil titans are reaping the rewards.</p> </blockquote> <p>Over in New York shares are also rising, with the Dow up almost 200 points. Not much sign of panic over Donald Trump’s announcement.....</p> </div>   </div> <div id="block-5af30703e4b0329aaa4ac4ef" class="block is-key-event" data-block-contributor=""> <p class="block-time published-time"> <time datetime="2018-05-09T14:47:47.426Z">3.47pm <span class="timezone">BST</span></time> </p>   <h2 class="block-title">US inventory figures drive oil price higher</h2>  <div class="block-elements">  <p><strong>Newsflash: America’s reserves of crude and refined oil have fallen sharply, sending the oil price even higher.</strong></p> <p>US crude stocks fell by 2.2 million barrels last week, according to new figures from the Energy Information Administration.</p> <p>Economists had only expected a decline of 719,000, so this is a sign that demand for energy is stronger than expected.</p> <p>Oil traders have responded quickly, driving the price of a barrel of Brent crude up t0 \$77.43 - a new 3.5 year high.</p> <p>The EIA also reported that gasoline stocks dropped by more than expected, by 2.17m barrels per day.</p>  <figure class="element element-tweet" data-canonical-url="https://twitter.com/BrynneKKelly/status/994224306023534592">  <blockquote class="twitter-tweet"><p lang="en" dir="ltr"><a href="https://twitter.com/hashtag/Crudeoil?src=hash\&amp;ref\_src=twsrc\%5Etfw">\#Crudeoil</a> US EIA Inventory LEVELS vs Same Week in Prior Years: <a href="https://t.co/dtsUrnNYFC">pic.twitter.com/dtsUrnNYFC</a></p>\&mdash; Brynne Kelly (@BrynneKKelly) <a href="https://twitter.com/BrynneKKelly/status/994224306023534592?ref\_src=twsrc\%5Etfw">May 9, 2018</a></blockquote>  </figure>  <figure class="element element-tweet" data-canonical-url="https://twitter.com/ForexLive/status/994223965571899392">  <blockquote class="twitter-tweet"><p lang="en" dir="ltr">Fresh highs for WTI crude oil weekly inventory data <a href="https://t.co/ajxWPtZ7Fz">https://t.co/ajxWPtZ7Fz</a> <a href="https://t.co/q54w5mJ2lF">pic.twitter.com/q54w5mJ2lF</a></p>\&mdash; ForexLive (@ForexLive) <a href="https://twitter.com/ForexLive/status/994223965571899392?ref\_src=twsrc\%5Etfw">May 9, 2018</a></blockquote>  </figure> </div>   </div> <div id="block-5af306aee4b0329aaa4ac4ec" class="block" data-block-contributor=""> <p class="block-time published-time"> <time datetime="2018-05-09T14:34:12.628Z">3.34pm <span class="timezone">BST</span></time> </p>    <div class="block-elements">  <p>City Index point out that French oil giant Total could lose out on a valuable contract in Iran.</p>  <figure class="element element-tweet" data-canonical-url="https://twitter.com/CityIndex/status/994222984314449920">  <blockquote class="twitter-tweet"><p lang="en" dir="ltr">France\&\#39;s <a href="https://twitter.com/hashtag/TOTAL?src=hash\&amp;ref\_src=twsrc\%5Etfw">\#TOTAL</a> arguably faces biggest and possibly earliest impact from new <a href="https://twitter.com/hashtag/Iran?src=hash\&amp;ref\_src=twsrc\%5Etfw">\#Iran</a> sanctions. Planned deal to develop South Pars gas field in Persian Gulf. Has invested \textasciitilde{}\$1bn so far. <a href="https://twitter.com/search?q=\%24TOT\&amp;src=ctag\&amp;ref\_src=twsrc\%5Etfw">\$TOT</a> NY shrs actually up 1.2\% right now. Vvery likely on oil price support \textasciicircum{}KO</p>\&mdash; City Index (@CityIndex) <a href="https://twitter.com/CityIndex/status/994222984314449920?ref\_src=twsrc\%5Etfw">May 9, 2018</a></blockquote>  </figure> </div>   </div> <div id="block-5af303e1e4b0329aaa4ac4bf" class="block" data-block-contributor=""> <p class="block-time published-time"> <time datetime="2018-05-09T14:21:36.488Z">3.21pm <span class="timezone">BST</span></time> </p>    <div class="block-elements">  <figure class="element element-image" data-media-id="ac3aff62c9f128bd362e20a5b27fbe1d8544044e"> <img src="https://media.guim.co.uk/ac3aff62c9f128bd362e20a5b27fbe1d8544044e/0\_0\_5760\_3840/1000.jpg" alt="Tehran, the capital of Iran, today." width="1000" height="667" class="gu-image" /> <figcaption> <span class="element-image\_\_caption">Tehran, the capital of Iran, today.</span> <span class="element-image\_\_credit">Photograph: Anadolu Agency/Getty Images</span> </figcaption> </figure> </div>   </div> <div id="block-5af30180e4b05ff36022d318" class="block" data-block-contributor=""> <p class="block-time published-time"> <time datetime="2018-05-09T14:14:49.030Z">3.14pm <span class="timezone">BST</span></time> </p>    <div class="block-elements">  <p><strong>The slump in the Iranian rial today is a reminder of the damage that new sanctions will cause to Iran’s economy.</strong></p> <p>Here’s Associated Press’s take:</p> <blockquote class="quoted"> <p>Many Iranians are worried about what Trump’s decision could mean for their country.</p> <p> The Iranian rial is already trading on the black market at 66,000 to the dollar, despite a government-set rate of 42,000 rials. Many say they have not seen any benefits from the nuclear deal.</p> <p> Iran’s poor economy and unemployment sparked nationwide protests in December and January that saw at least 25 people killed and, reportedly, nearly 5,000 arrested.</p> </blockquote> </div>   </div> <div id="block-5af2fa56e4b05a463ca1e3ce" class="block is-key-event" data-block-contributor=""> <p class="block-time published-time"> <time datetime="2018-05-09T13:54:49.514Z">2.54pm <span class="timezone">BST</span></time> </p>   <h2 class="block-title">US energy stocks jump</h2>  <div class="block-elements">  <p><strong>Ding Ding! US oil company stocks are rallying in early trading, as Wall Street opens for business.</strong></p> <p>The Dow Jones industrial average has gained 107 points at the open, as New York traders shrug off Iranian worries.</p> <p>Oil giants such as Exxon and Chevron are both benefitting from the pick-up in crude prices, and the prospect that Iran’s supplies will be squeezed out of the market.</p>  <figure class="element element-image" data-media-id="23e9fda7fefb86bb492d09455617cd13966568d5"> <img src="https://media.guim.co.uk/23e9fda7fefb86bb492d09455617cd13966568d5/0\_0\_1028\_601/1000.png" alt="The open of Wall Street" width="1000" height="585" class="gu-image" /> <figcaption> <span class="element-image\_\_caption">The open of Wall Street</span> <span class="element-image\_\_credit">Photograph: Bloomberg TV</span> </figcaption> </figure>  <p><strong>Craig Erlam</strong> of trading firm <strong>OANDA</strong> says:</p> <blockquote class="quoted"> <p>Oil has been rallying for days in response to rumours that Trump would announce the withdrawal, which clearly suggests that traders believe the sanctions will further tighten global supply at a time when some of the world’s largest producers have already significantly reduced inventories.</p> <p>There is clearly the potential for these countries to fill the void left by the sanctions but if it aids their cause then they’ll likely opt against it.</p> </blockquote> </div>   </div> <div id="block-5af2fae8e4b05a463ca1e3d0" class="block" data-block-contributor=""> <p class="block-time published-time"> <time datetime="2018-05-09T13:43:13.271Z">2.43pm <span class="timezone">BST</span></time> </p>    <div class="block-elements">  <figure class="element element-interactive interactive" data-interactive="https://interactive.guim.co.uk/embed/iframe-wrapper/0.1/boot.js" data-canonical-url="https://interactive.guim.co.uk/charts/embed/may/2018-05-09T13:17:58/embed.html" data-alt="Brent crude oil price"> <a href="https://interactive.guim.co.uk/charts/embed/may/2018-05-09T13:17:58/embed.html">Brent crude oil price</a> </figure> </div>   </div> <div id="block-5af2f787e4b0123ab002213e" class="block" data-block-contributor=""> <p class="block-time published-time"> <time datetime="2018-05-09T13:33:40.645Z">2.33pm <span class="timezone">BST</span></time> </p>    <div class="block-elements">  <p>Here’s our round-up of the companies who could be hurt by Donald Trump’s decision to rip up the Iranian nuclear deal, from oil firms to car producers and airline manufacturers:</p> <aside class="element element-rich-link"> <p> <span>Related: </span><a href="https://www.theguardian.com/world/2018/may/09/how-trump-iran-deal-exit-could-hit-aviation-oil-and-car-industries">How Trump's Iran deal exit could hit aviation, oil and car industries</a> </p> </aside> </div>   </div> <div id="block-5af2f0f8e4b0091033b8b82a" class="block" data-block-contributor=""> <p class="block-time published-time"> <time datetime="2018-05-09T13:29:49.162Z">2.29pm <span class="timezone">BST</span></time> </p>    <div class="block-elements">  <p><strong>The Iranian clampdown could also affect oil giant BP.</strong></p> <p>BP recently agreed to sell a North Sea gas field which it co-owns with Iran’s state owned oil company. New sanctions could potentially cause that deal to unravel.</p> <p>My colleague Rob Davies explains:</p> <blockquote class="quoted"> <p>Late last year BP agreed to sell three North Sea gas fields to Serica Energy for \$400m. One of the fields, Rhum, is co-owned by a subsidiary of Iran’s national oil company. That means a licence is required from the US Office of Foreign Asset Control (OFAC) to allow US nationals or companies to work on it.</p> <p>If Serica cannot obtain that licence because of new sanctions it faces the risk of being unable to call on US-owned companies, in the event, for instance, of a fire or oil spill, severely restricting its emergency options.</p> <p>The sale is not yet complete and it is unclear whether it will go ahead now. “We take care to ensure we always comply with applicable sanctions,” BP said.</p> </blockquote> <p>The BBC’s <strong>Simon Jack</strong> agrees that the deal could be in doubt.</p>  <figure class="element element-tweet" data-canonical-url="https://twitter.com/BBCSimonJack/status/994116938824642567">  <blockquote class="twitter-tweet"><p lang="en" dir="ltr">BP potentially affected by any new sanctions. Massive business in US and It co-owns the Rum gasfield in the North Sea with Iran\&\#39;s state owned oil company. Not huge - but supplied 4\% of UK\&\#39;s gas demand in 2016. Depends how aggressively sanctions are imposed.</p>\&mdash; Simon Jack (@BBCSimonJack) <a href="https://twitter.com/BBCSimonJack/status/994116938824642567?ref\_src=twsrc\%5Etfw">May 9, 2018</a></blockquote>  </figure>  <figure class="element element-tweet" data-canonical-url="https://twitter.com/BBCSimonJack/status/994118595952115712">  <blockquote class="twitter-tweet" data-conversation="none"><p lang="en" dir="ltr">Serica entered agreement to buy BP\&\#39;s stake but the deal hasn\&\#39;t closed yet.</p>\&mdash; Simon Jack (@BBCSimonJack) <a href="https://twitter.com/BBCSimonJack/status/994118595952115712?ref\_src=twsrc\%5Etfw">May 9, 2018</a></blockquote>  </figure> </div>   </div> <div id="block-5af2ebe6e4b05a463ca1e33e" class="block" data-block-contributor=""> <p class="block-time published-time"> <time datetime="2018-05-09T12:56:02.823Z">1.56pm <span class="timezone">BST</span></time> </p>    <div class="block-elements">  <figure class="element element-image" data-media-id="8e1090568af430e30f838eda0bc4d2c088d29946"> <img src="https://media.guim.co.uk/8e1090568af430e30f838eda0bc4d2c088d29946/0\_490\_7283\_4371/1000.jpg" alt="The O2 Arena, the River Thames, and the skyscrapers of Canary Wharf." width="1000" height="600" class="gu-image" /> <figcaption> <span class="element-image\_\_caption">The skyscrapers of Canary Wharf in London, behind the O2 Arena on the River Thames</span> <span class="element-image\_\_credit">Photograph: Vladimir Zakharov/Getty Images</span> </figcaption> </figure>  <p><strong>Britain’s stock market has responded to the threat of geopolitical upheaval, by hitting its highest level in over three months.</strong></p> <p>The FTSE 100 has gained 35 points to reach 7601, for the first time since the end of January.</p> <p>Oil companies are leading the rally, with Royal Dutch Shell up 2\% and BP gaining 2.2\%, as Brent crude bobs just below the \$77 mark.</p> <p><strong>Connor Campbell</strong> of City firm <strong>SpreadEx</strong> explains:</p> <blockquote class="quoted"> <p>Trump’s Iranian nuclear deal pull-out and subsequent sanctions, via its impact on Brent Crude, remained the core driver of trading on Wednesday.</p> <p>Though Brent couldn’t quite hold above \$77 per barrel, the black stuff is still trading at its best price since 2014. This propelled BP and Shell more than 2\% higher apiece, gains that provided the bedrock of the FTSE’s own growth. </p> </blockquote> <p>Other European indices are calm today, with the pan-Europe Stoxx 600 up 0.2\%.</p> <p>Wall Street is also expected to open higher, as investors take the Iranian nuclear issue in their stride. </p>  <figure class="element element-tweet" data-canonical-url="https://twitter.com/IGSquawk/status/994183716271546368">  <blockquote class="twitter-tweet"><p lang="en" dir="ltr">US Opening Calls:<a href="https://twitter.com/hashtag/DOW?src=hash\&amp;ref\_src=twsrc\%5Etfw">\#DOW</a>  24463  +0.47\%<a href="https://twitter.com/hashtag/SPX?src=hash\&amp;ref\_src=twsrc\%5Etfw">\#SPX</a>  2683  +0.41\%<a href="https://twitter.com/hashtag/NASDAQ?src=hash\&amp;ref\_src=twsrc\%5Etfw">\#NASDAQ</a>  6838  +0.34\%<a href="https://twitter.com/hashtag/IGOpeningCall?src=hash\&amp;ref\_src=twsrc\%5Etfw">\#IGOpeningCall</a></p>\&mdash; IGSquawk (@IGSquawk) <a href="https://twitter.com/IGSquawk/status/994183716271546368?ref\_src=twsrc\%5Etfw">May 9, 2018</a></blockquote>  </figure> </div>   </div> <div id="block-5af2e052e4b0123ab002204e" class="block" data-block-contributor=""> <p class="block-time published-time"> <time datetime="2018-05-09T11:53:07.357Z">12.53pm <span class="timezone">BST</span></time> </p>    <div class="block-elements">  <p><strong>Over in parliament, foreign secretary Boris Johnson is updating MPs about Iran.</strong></p> <p>He says that Britain has no intention of walking away from the nuclear deal (the Joint Comprehensive Plan of Action).</p> <p>America must now “spell out” their view of the way ahead, Johnson continues.</p> <p>He also urges Iran to show restraint, adding that a nuclear-armed Iran would never be acceptable to Britain.</p> <p>Our Politics Live blog has full details:</p> <aside class="element element-rich-link"> <p> <span>Related: </span><a href="https://www.theguardian.com/politics/blog/live/2018/may/09/pmqs-theresa-may-jeremy-corbyn-leveson-debate-trump-ignored-our-warnings-over-iran-nuclear-deal-foreign-office-minister-admits-politics-live">Iran nuclear deal: UK won't walk away, Boris Johnson tells MPs - Politics live</a> </p> </aside> </div>   </div> <div id="block-5af2dda0e4b0123ab002201f" class="block" data-block-contributor=""> <p class="block-time published-time"> <time datetime="2018-05-09T11:49:19.001Z">12.49pm <span class="timezone">BST</span></time> </p>    <div class="block-elements">  <p><strong>Iranian citizens and businesses are facing fresh pain today as their currency, the rial, falls to fresh record lows.</strong></p> <p>Reuters has the details:<br></p> <blockquote class="quoted"> <p>The dollar was being offered for as much as 75,000 rials, compared to around 65,000 just before Trump announced his decision on Tuesday night, according to foreign exchange website <a href="https://www.bonbast.com/">Bonbast.com</a>, which tracks the free market.</p> <p>Dealers in Tehran quoted similar levels on Wednesday, according to an Iranian economist outside the country who is in touch with them. One dealer said the rial had hit 78,000, while another said he had made two sales of dollars at 80,000.</p> </blockquote> <p><a href="https://twitter.com/AmirPaivar">Amir Paivar</a> of the BBC Persian TV points out that there is confusion in the market, after several days of heavy losses.</p>  <figure class="element element-tweet" data-canonical-url="https://twitter.com/AmirPaivar/status/994117816835084288">  <blockquote class="twitter-tweet"><p lang="en" dir="ltr">BREAKING - one reliable website tracking <a href="https://twitter.com/hashtag/Iran?src=hash\&amp;ref\_src=twsrc\%5Etfw">\#Iran</a>\&\#39;s <a href="https://twitter.com/hashtag/currency?src=hash\&amp;ref\_src=twsrc\%5Etfw">\#currency</a> exchange puts the rate at US\$:75,000 rials - a historic low for Iranian currency and a whopping 14.5\% down for the rial compared to three days ago (last two days due to uncertainty trade was almost at a halt). <a href="https://twitter.com/hashtag/IranDeal?src=hash\&amp;ref\_src=twsrc\%5Etfw">\#IranDeal</a></p>\&mdash; Amir Paivar (@AmirPaivar) <a href="https://twitter.com/AmirPaivar/status/994117816835084288?ref\_src=twsrc\%5Etfw">May 9, 2018</a></blockquote>  </figure>  <figure class="element element-tweet" data-canonical-url="https://twitter.com/AmirPaivar/status/994130457687220225">  <blockquote class="twitter-tweet" data-conversation="none"><p lang="en" dir="ltr">NOTE: Traders offering significantly different exchange rate for US\$ to the rial. Confusion in the market. Couple of other sources put the rate at US\$:66,000. <a href="https://twitter.com/hashtag/IranDeal?src=hash\&amp;ref\_src=twsrc\%5Etfw">\#IranDeal</a></p>\&mdash; Amir Paivar (@AmirPaivar) <a href="https://twitter.com/AmirPaivar/status/994130457687220225?ref\_src=twsrc\%5Etfw">May 9, 2018</a></blockquote>  </figure> </div>   </div> <div id="block-5af2d6f7e4b0091033b8b6a2" class="block" data-block-contributor=""> <p class="block-time published-time"> <time datetime="2018-05-09T11:17:02.728Z">12.17pm <span class="timezone">BST</span></time> </p>    <div class="block-elements">  <p>Brent crude is heading towards \$80 per barrel, says Ken Odeluga, market analyst at City Index.</p> <p>Uncertainty over the Iranian situation, combined with speculation in the markets, are capable of “taking prices even higher in the near-term”, he explains. </p> </div>   </div> <div id="block-5af2ce49e4b0091033b8b611" class="block" data-block-contributor=""> <p class="block-time published-time"> <time datetime="2018-05-09T10:50:30.710Z">11.50am <span class="timezone">BST</span></time> </p>    <div class="block-elements">  <figure class="element element-image" data-media-id="e6322e9d1e42632913d53b1229f094661896f027"> <img src="https://media.guim.co.uk/e6322e9d1e42632913d53b1229f094661896f027/0\_0\_2379\_1313/1000.jpg" alt="The Soroush oil fields in the Persian Gulf, south of Tehran." width="1000" height="552" class="gu-image" /> <figcaption> <span class="element-image\_\_caption">The Soroush oil fields in the Persian Gulf, south of Tehran.</span> <span class="element-image\_\_credit">Photograph: Raheb Homavandi/Reuters</span> </figcaption> </figure>  <p>Trump’s decision to abandon the Iranian nuclear deal will have a major impact on the oil market, says City investor <strong>Richard Robinson.</strong></p> <p>Robinson, who manages the Ashburton Global Energy Fund, says it will probably remove hundreds of thousands of barrels of oil from the market each day.</p> <blockquote class="quoted"> <p>The worst-case scenario, involving strict adherence and policing of sanctions, could see as much as 700kbbld removed from the market. A less disciplined approach, with ambiguous US guidance, could remove less than 200k barrels per day. Trump’s rhetoric sounded fairly unambiguous.</p> <p>Although Trump would have ideally liked Europe to join him in re-imposing sanctions, the eventual effect is likely to be similar. Europe currently imports between 500-600kbbld of Iranian crude and it is expected this number will drop by approximately 60\%. Iranian barrels can be easily substituted for Iraqi crudes.</p> <p>European companies are present in both Iran and the US. Companies such as Total and Eni may choose to stop lifting Iranian crudes altogether, for fear of being precluded from the US market. It is expected Asian refiners, with exposure to both regions, may also choose to follow the ban.</p> <p>China, on the other hand, may choose to take advantage of the widening discounts and purchase more Iranian crude – through various shell companies.</p> </blockquote> </div>   </div> <div id="block-5af2c4dbe4b0123ab0021ece" class="block" data-block-contributor=""> <p class="block-time published-time"> <time datetime="2018-05-09T10:09:22.838Z">11.09am <span class="timezone">BST</span></time> </p>    <div class="block-elements">  <p>Companies who have been dealing with Iran now risk violating US sanctions, and need to plan accordingly, says <strong>Michael Harris</strong>, director of financial crime compliance at <a href="http://www.risk.lexisnexis.co.uk/">LexisNexis Risk Solutions</a>.</p> <blockquote class="quoted"> <p>Any organisation conducting business with Iranian individuals, companies or related entities could actually find themselves subject to sanctions due to the United States’ withdrawal from the Joint Comprehensive Plan of Action, and subsequent re-implementation of secondary sanctions. </p> <p>With such a significant change in position from the US, it is vital that firms re-evaluate their dealings with censured entities and regularly screen their customers against the latest and most relevant sanctions lists. If not, they risk breaching sanctions, facing unprecedented fines and reputational damage, as well as repercussions for future business with the United States.”</p> </blockquote> </div>   </div> <div id="block-5af2baf4e4b0123ab0021e53" class="block" data-block-contributor=""> <p class="block-time published-time"> <time datetime="2018-05-09T09:50:23.912Z">10.50am <span class="timezone">BST</span></time> </p>    <div class="block-elements">  <p><strong>US planemaker Boeing may also be unhappy with Trump’s decision.</strong></p> <p>Like Airbus, Boeing has signed billions of dollars worth of deals with Iranian airlines in the last couple of years. Last night, treasury secretary Steven Mnuchin warned that Boeing and Airbus’s licences to trade with Iran “will be revoked.”</p>  <figure class="element element-tweet" data-canonical-url="https://twitter.com/CNBC/status/994072620562829313">  <blockquote class="twitter-tweet"><p lang="en" dir="ltr">Boeing has about \$20 billion in agreements with Iranian airlines for planes — but Trump\&\#39;s plan to pull out of the Iran nuclear deal puts them in jeopardy. <a href="https://t.co/LwHSw811oc">https://t.co/LwHSw811oc</a></p>\&mdash; CNBC (@CNBC) <a href="https://twitter.com/CNBC/status/994072620562829313?ref\_src=twsrc\%5Etfw">May 9, 2018</a></blockquote>  </figure> </div>   </div> <div id="block-5af2ae4fe4b0091033b8b4e8" class="block is-key-event" data-block-contributor=""> <p class="block-time published-time"> <time datetime="2018-05-09T09:14:16.251Z">10.14am <span class="timezone">BST</span></time> </p>   <h2 class="block-title">Companies with Iranian ties take a hit</h2>  <div class="block-elements">  <p><strong>Donald Trump’s decision to reimpose sanctions on Iran is already hurting companies with ties to Tehran.</strong></p> <p>Shares in French carmakers Renault and Peugeot have fallen by over 1.2\% this morning. They’ve both built links with Iran to build and sell cars, so now face being locked out again.</p> <p>Airbus is another possible casualty, having recently <a href="http://www.airbus.com/newsroom/press-releases/en/2017/01/iran-air-takes-delivery-of-its-first-of-100-airbus-aircraft.html">sold 100 planes to IranAir</a>. Airbus shares are down almost 1\% this morning; it will be caught by the sanctions, as its aircraft use many US-build parts.</p> <p><strong>Carl Bildt</strong>, former prime minister of Sweden, says European companies will suffer more from Trump’s actions than US ones.</p>  <figure class="element element-tweet" data-canonical-url="https://twitter.com/carlbildt/status/994117770865532929">  <blockquote class="twitter-tweet"><p lang="en" dir="ltr">US Iran sanctions are hardly hitting any US companies, but aim primarily at European ones. This is a behavior with secondary sanctions that EU in the past has considered to be utterly unacceptable.</p>\&mdash; Carl Bildt (@carlbildt) <a href="https://twitter.com/carlbildt/status/994117770865532929?ref\_src=twsrc\%5Etfw">May 9, 2018</a></blockquote>  </figure>  <p>Politico’s <strong>Nahal Toosi</strong> agrees:</p>  <figure class="element element-tweet" data-canonical-url="https://twitter.com/nahaltoosi/status/993930637496156161">  <blockquote class="twitter-tweet"><p lang="en" dir="ltr">This can\&\#39;t be stressed enough: These sanctions Trump is reimposing are largely secondary sanctions, meaning they target European and other companies that do business with Iran. So it\&\#39;s not just about a diplomatic dust-up between the US and Europe. It\&\#39;s a financial hit on Europe.</p>\&mdash; Nahal Toosi (@nahaltoosi) <a href="https://twitter.com/nahaltoosi/status/993930637496156161?ref\_src=twsrc\%5Etfw">May 8, 2018</a></blockquote>  </figure> </div>   <p class="block-time updated-time">Updated <time datetime="2018-05-09T09:43:19.356Z">at 10.43am BST</time></p>  </div> <div id="block-5af2b240e4b0091033b8b505" class="block" data-block-contributor=""> <p class="block-time published-time"> <time datetime="2018-05-09T08:49:41.658Z">9.49am <span class="timezone">BST</span></time> </p>    <div class="block-elements">  <p><strong>China, India and Korea could all be hit by Trump’s decision.</strong></p> <p>As this chart shows, they are the biggest customers for the oil which Iran has been pumping vigorously since sanctions were lifted in 2016.<br></p>  <figure class="element element-image" data-media-id="56a23f60159cbb897d4d93df9e31cbfb8100101e"> <img src="https://media.guim.co.uk/56a23f60159cbb897d4d93df9e31cbfb8100101e/32\_36\_1147\_692/1000.png" alt="Iran oil sales" width="1000" height="603" class="gu-image" /> </figure> </div>   </div> <div id="block-5af2ac11e4b05a463ca1e080" class="block" data-block-contributor=""> <p class="block-time published-time"> <time datetime="2018-05-09T08:13:26.074Z">9.13am <span class="timezone">BST</span></time> </p>    <div class="block-elements">  <p><strong>Donald Trump’s decision has gone down predictably badly in Iran.</strong></p> <p>MPs symbolically burned a US flag in the country’s parliament this morning and chanted “death to America”</p> <p>The Iranian parliament’s speaker, Ali Larijani, accused the US president of conducting a “diplomatic show” and claimed that president Trump “does not have the mental capacity” for the job.</p>  <figure class="element element-tweet" data-canonical-url="https://twitter.com/AP/status/994125109882425344">  <blockquote class="twitter-tweet"><p lang="en" dir="ltr">Iranian lawmakers set a paper U.S. flag ablaze today at parliament in Tehran after Trump\&\#39;s nuclear deal pullout, shouting, \&quot;Death to America!\&quot;. Trump withdrew the U.S. from the deal on Tuesday and restored harsh sanctions against Iran. Full <a href="https://twitter.com/AP?ref\_src=twsrc\%5Etfw">@AP</a> coverage: <a href="https://t.co/MqybGfTnzr">https://t.co/MqybGfTnzr</a> <a href="https://t.co/Do4uSOxOy5">pic.twitter.com/Do4uSOxOy5</a></p>\&mdash; The Associated Press (@AP) <a href="https://twitter.com/AP/status/994125109882425344?ref\_src=twsrc\%5Etfw">May 9, 2018</a></blockquote>  </figure> </div>   </div> <div id="block-5af2a720e4b0123ab0021df0" class="block" data-block-contributor=""> <p class="block-time published-time"> <time datetime="2018-05-09T08:03:53.620Z">9.03am <span class="timezone">BST</span></time> </p>    <div class="block-elements">  <p><strong>Geopolitical anxiety is driving investors into the safety of the US dollar today, sending it to its highest level since December 2017.</strong></p>  <figure class="element element-tweet" data-canonical-url="https://twitter.com/SanlamAU/status/994097736537423872">  <blockquote class="twitter-tweet"><p lang="en" dir="ltr">US Dollar Index <a href="https://twitter.com/hashtag/DXY?src=hash\&amp;ref\_src=twsrc\%5Etfw">\#DXY</a> (orange) hits a year-to-date high while  JP Morgan Emerging Market <a href="https://twitter.com/hashtag/EM?src=hash\&amp;ref\_src=twsrc\%5Etfw">\#EM</a> Currency Index (white) hits a 1 year low... <a href="https://t.co/u9ge5gzmN7">pic.twitter.com/u9ge5gzmN7</a></p>\&mdash; Sanlam Private Wealth [Australia] (@SanlamAU) <a href="https://twitter.com/SanlamAU/status/994097736537423872?ref\_src=twsrc\%5Etfw">May 9, 2018</a></blockquote>  </figure>  <p>That’s bad for emerging market currencies, such as the Turkish lira, which has lurched to a fresh all-time low of 4.3736 liras to the dollar.<br></p> <p>That might force Istanbul’s central bank to intervene to prop up the lira</p>  <figure class="element element-tweet" data-canonical-url="https://twitter.com/MarkABentley/status/994103153065938945">  <blockquote class="twitter-tweet"><p lang="en" dir="ltr">Lira hits new low of 4.37/\$.<br><br>Central bank intervention today? <a href="https://t.co/ZJfBCal59I">pic.twitter.com/ZJfBCal59I</a></p>\&mdash; Mark Bentley (@MarkABentley) <a href="https://twitter.com/MarkABentley/status/994103153065938945?ref\_src=twsrc\%5Etfw">May 9, 2018</a></blockquote>  </figure> </div>   </div> <div id="block-5af2a6a2e4b05a463ca1e060" class="block is-key-event" data-block-contributor=""> <p class="block-time published-time"> <time datetime="2018-05-09T07:43:59.506Z">8.43am <span class="timezone">BST</span></time> </p>   <h2 class="block-title">Brent rises over \$77</h2>  <div class="block-elements">  <p>Brent crude is continuing to rise, and just hit \$77.20 per barrel for the first time since November 2014.</p>  <figure class="element element-tweet" data-canonical-url="https://twitter.com/Ken\_CityIndex/status/994117684534153216">  <blockquote class="twitter-tweet"><p lang="en" dir="ltr"><a href="https://twitter.com/hashtag/OIL?src=hash\&amp;ref\_src=twsrc\%5Etfw">\#OIL</a>: After a fairly short dip immediately following President Donald Trump\&\#39;s decision to pull U.S. out of <a href="https://twitter.com/hashtag/Iran?src=hash\&amp;ref\_src=twsrc\%5Etfw">\#Iran</a> nuclear deal, oil has set series of new 3.5 yr highs. Latest <a href="https://twitter.com/hashtag/Brent?src=hash\&amp;ref\_src=twsrc\%5Etfw">\#Brent</a> crude peak \$77.20/bbl. \textasciicircum{}KO</p>\&mdash; Ken Odeluga (@Ken\_CityIndex) <a href="https://twitter.com/Ken\_CityIndex/status/994117684534153216?ref\_src=twsrc\%5Etfw">May 9, 2018</a></blockquote>  </figure> </div>   </div> <div id="block-5af2a64ae4b05a463ca1e05e" class="block" data-block-contributor=""> <p class="block-time published-time"> <time datetime="2018-05-09T07:43:05.395Z">8.43am <span class="timezone">BST</span></time> </p>    <div class="block-elements">  <p>Trump’s “very aggressive rhetoric” against Iran last night has jolted the oil market, says <strong>Lukman Otunuga</strong>, research analyst at <strong>FXTM</strong> </p> <blockquote class="quoted"> <p>While it was widely anticipated that Trump would pull out of the Iran agreement, what is likely to leave a lasting impact on the markets is the threat that he would also penalize those who help Iran.</p> <p>These overall risks are encouraging traders to price in some new geopolitical risk premium, and his threat can potentially be seen as a blow for U.S allies. There is a threat of Trump’s stark tone questioning U.S relations with its European allies, especially given that the likes of France and the United Kingdom had appealed for Trump not to withdraw. </p> </blockquote> </div>   </div> <div id="block-5af2a2dbe4b0091033b8b4a6" class="block" data-block-contributor=""> <p class="block-time published-time"> <time datetime="2018-05-09T07:34:24.450Z">8.34am <span class="timezone">BST</span></time> </p>    <div class="block-elements">  <p>Some major European companies will be very disappointed by Trump’s decision, having signed some large deals with Tehran since sanctions were lifted under Barack Omaba.</p> <p><a href="https://qz.com/1272832/trump-quits-iran-deal-us-sanctions-are-a-heavy-blow-for-europe/">Quartz explains:</a></p> <blockquote class="quoted"> <p>Energy giants like <a href="https://www.ft.com/content/f3c2d084-0e83-11e8-8cb6-b9ccc4c4dbbb">Total</a> and <a href="https://www.bloomberg.com/news/articles/2016-12-07/shell-and-total-said-to-sign-initial-oil-deals-with-iran">Royal Dutch Shell</a> have lucrative agreements to work with Iran, while <a href="https://www.reuters.com/article/us-renault-iran/renault-forms-new-joint-venture-company-in-iran-idUSKBN1AN12L">Renault</a> has a joint venture to make 150,000 cars a year, and Franco-German plane maker Airbus has <a href="https://www.reuters.com/article/us-airbus-orders-iran/iran-jetliner-deal-could-take-longer-to-complete-airbus-says-idUSKBN1F42B1">reportedly delivered</a> just three out of 100 jets promised to Iran, in a deal worth billions....</p> <p>Any European companies with a US arm that agreed a deal with Iran would now be violating US law, says Adam Smith, a lawyer at Gibson Dunn who is a former Treasury sanctions official.</p> <p>Those not active in the US could be hit with a “with-us or-against-us sanction,” in which Washington would tell the company that if it wants to keep trading with Iran they can’t trade with America, Smith said.</p> </blockquote> </div>   </div> <div id="block-5af29129e4b0091033b8b461" class="block is-key-event" data-block-contributor=""> <p class="block-time published-time"> <time datetime="2018-05-09T07:05:38.116Z">8.05am <span class="timezone">BST</span></time> </p>   <h2 class="block-title">The agenda: Oil jumps after Trump's decision on Iran</h2>  <div class="block-elements">  <figure class="element element-image" data-media-id="25f3eac8a8d17716ecf8f4451fd0dc0382e1b605"> <img src="https://media.guim.co.uk/25f3eac8a8d17716ecf8f4451fd0dc0382e1b605/0\_0\_4305\_2928/1000.jpg" alt="US President Donald Trump signs a document reinstating sanctions against Iran after announcing the US withdrawal from the Iran Nuclear deal, in the Diplomatic Reception Room at the White House in Washington, DC, yesterday" width="1000" height="680" class="gu-image" /> <figcaption> <span class="element-image\_\_caption">US President Donald Trump signs a document reinstating sanctions against Iran yesterday</span> <span class="element-image\_\_credit">Photograph: Saul Loeb/AFP/Getty Images</span> </figcaption> </figure>  <p><strong>Good morning, </strong><strong>and welcome to our rolling coverage of the world economy, the financial markets, the eurozone and business.</strong></p> <p>The oil price has been driven to its highest level in three and a half years after Donald Trump ignored the pleas of European allies and pulled America out of the Iran nuclear deal.</p> <p>Brent crude - sourced from the North Sea - was swept up to \$76.85 this morning, as traders digest the prospect of a new crisis in the Gulf.</p> <p>The oil price jumped following Trump’s decision to impose “the highest level of economic sanctions” on Iran, and to reimpose sanctions on any foreign company that continues to do business with it.</p>  <figure class="element element-image" data-media-id="d324cbfe81c51e9eb8f4c5cd6a16660ef3223645"> <img src="https://media.guim.co.uk/d324cbfe81c51e9eb8f4c5cd6a16660ef3223645/0\_0\_1133\_548/1000.jpg" alt="The brent crude oil price over the last five years" width="1000" height="484" class="gu-image" /> <figcaption> <span class="element-image\_\_caption">The brent crude oil price over the last five years</span> <span class="element-image\_\_credit">Photograph: Thomson Reuters</span> </figcaption> </figure>  <p>In a hardline move, Trump damned the Iran agreement as “a horrible one-sided deal that should never, ever have been made”.</p> <p>This has disappointed those who had urged him not to unilaterally pull out, and risk encouraging Iran to develop nuclear weapons. Trump, though, insisted the deal wasn’t working.</p> <aside class="element element-rich-link"> <p> <span>Related: </span><a href="https://www.theguardian.com/world/2018/may/08/iran-deal-trump-withdraw-us-latest-news-nuclear-agreement">Iran deal: Trump breaks with European allies over 'horrible, one-sided' nuclear agreement</a> </p> </aside>  <p>The decision means that Iran’s oil production, which picked up after sanctions were lifted in 2016, will surely now decline. <strong>That could lead to a tighter market and new upward pressure on prices.</strong></p> <p>There’s also a danger that geopolitical tensions in the Middle East will rise, leading to supply disruptions.</p> <p>Analysts at <strong>Royal Bank of Canada</strong> told clients:<br></p> <blockquote class="quoted"> <p>The U.S. will reimpose all sanctions waived under the 2015 landmark agreement, he said, calling it “defective at its core.”</p> <p>As a counter, Iranian President Rouhani warned his country is ready to resume enriching uranium within weeks. In a strongly worded joint statement, the UK, France and Germany rejected Trump’s decision.</p> <p>Oil rose more than 3\% following the announcement while the dollar index strengthened further.</p> </blockquote> <p><strong>Jasper Lawler </strong>of <strong>CMC Markets</strong> reckons we should brace for higher oil prices, which will drive up fuel and energy costs for consumers. </p> <p>He explains:</p> <blockquote class="quoted"> <p>With Venezuela still firmly in crisis and now Iran potentially facing sanctions, we could easily find that plus \$70 per barrel becomes the new norm in a market which has already been tightened by OPEC.</p> </blockquote> <h2>Also coming up today</h2> <p>Britain’s retail sector has suffered its worst monthly sales fall in more than two decades, in the latest sign that consumers are struggling.</p> <aside class="element element-rich-link"> <p> <span>Related: </span><a href="https://www.theguardian.com/business/2018/may/09/uk-retailers-suffer-sharpest-sales-drop-for-22-years-in-april">UK retailers suffer sharpest sales drop for 22 years in April</a> </p> </aside>  <p>We’ll also be watching Argentina, which has begun talking the International Monetary Fund about a new credit line, in an attempt to prop up its currency and stem market panic.</p> <p>Also, pub chain <strong>J Wetherspoon </strong>and bakers <strong>Greggs</strong> are reporting results to the City.</p> <p><strong>Greggs</strong> has warned that recent bad weather hit sales (as people didn’t brave the icy streets for a sausage roll), while <a href="https://www.investegate.co.uk/wetherspoon--jd--plc--jdw-/rns/trading-statement/201805090700084277N/"><strong>Wetherspoon’s </strong>used most of its statement</a> to argue against Britain remaining in an EU customs union (it also said sales growth slowed a little)....</p>  <figure class="element element-tweet" data-canonical-url="https://twitter.com/GarryWhite/status/994105774325555200">  <blockquote class="twitter-tweet"><p lang="en" dir="ltr">\&quot;The EU masquerades as a free trade organisation, but it is really a protection racket which imposes import taxes on the 93\% of the world\&\#39;s population that is not in the EU,\&quot; says Tim Martin in JD Wetherspoon trading update.</p>\&mdash; Garry White (@GarryWhite) <a href="https://twitter.com/GarryWhite/status/994105774325555200?ref\_src=twsrc\%5Etfw">May 9, 2018</a></blockquote>  </figure> </div>   <p class="block-time updated-time">Updated <time datetime="2018-05-09T08:09:50.030Z">at 9.09am BST</time></p>  </div>\\
Music & <p>The American rockers Paramore have <a href="https://www.nme.com/news/music/paramore-axe-controversial-track-misery-business-live-shows-2377010" title="">retired their breakthrough smash Misery Business from their live shows</a>, owing to frontwoman Hayley Williams’ discomfort with the slut-shaming lyrics (“Once a whore, you’re nothing more)” penned when she was 17. However, pop has a long, dishonourable lineage of songs that now sound unfortunate or offensive. Here are some examples:</p> <h2><strong>The Beatles – I Saw Her Standing There</strong></h2> <p>“She was just 17/You know what I mean.” We do. Not Fab.</p> <h2><strong>Beastie Boys – Girls</strong></h2> <p>Whisper it, but even the hallowed Beasties once portrayed women as domestic/sexual servants.</p>  <figure class="element element-image element--thumbnail" data-media-id="06e985d85d5bab1ebc2cbf47906790e38ce17de5"> <img src="https://media.guim.co.uk/06e985d85d5bab1ebc2cbf47906790e38ce17de5/859\_189\_2666\_2666/1000.jpg" alt="Mick Jagger of the Rolling Stones." width="1000" height="1000" class="gu-image" /> <figcaption> <span class="element-image\_\_caption">Mick Jagger of the Rolling Stones.</span> <span class="element-image\_\_credit">Photograph: Matthew Baker/Getty Images</span> </figcaption> </figure>  <h2><strong>The Rolling Stones – Under My Thumb</strong></h2> <p>Surprisingly, perhaps defiantly, this politically incorrect 1966 song about subordinating a “squirmin’ dog of a girl” remains in their setlist.</p> <h2><strong>Sinitta – So Macho</strong></h2> <p>1986 smash filled with political incorrectness (“I don’t want no seven stone weakling/ Or a boy who thinks he’s a girl”) and still widely played in gyms.</p> <h2><strong>Lily Allen – Not Big</strong></h2> <p>In 2018, is it big or clever for Allen to ridicule her ex’s penis?</p>  <figure class="element element-image element--thumbnail" data-media-id="9e501119386596642d908413ec8c65f04395744a"> <img src="https://media.guim.co.uk/9e501119386596642d908413ec8c65f04395744a/0\_1\_1908\_1145/1000.jpg" alt="Lily Allen." width="1000" height="600" class="gu-image" /> <figcaption> <span class="element-image\_\_caption">Lily Allen.</span> <span class="element-image\_\_credit">Photograph: Publicity image</span> </figcaption> </figure>  <h2><strong>Gary Puckett \&amp; the Union Gap – Young Girl</strong></h2> <p>Still a local radio staple, Gazza confesses to underage carnal desires.</p> <h2><strong>Mungo Jerry – In the Summertime</strong></h2> <p>“Have a drink, have a drive …”.</p> <h2><strong>Akinyele feat Kool G Rap – Break A Bitch Neck</strong></h2> <p>Hip-hop’s litany of offence plumbs the depths.</p> <h2><strong>Guns N’ Roses – One In A Million</strong></h2> <p>Few songs are as offensive as GNR’s 1988 rant about “immigrants and faggots”.</p>  <figure class="element element-image element--thumbnail" data-media-id="398bb8a22e3c70f5f5e5523b3e7c5a0eeebf9f50"> <img src="https://media.guim.co.uk/398bb8a22e3c70f5f5e5523b3e7c5a0eeebf9f50/0\_175\_2400\_2399/1000.jpg" alt="The Crystals." width="1000" height="1000" class="gu-image" /> <figcaption> <span class="element-image\_\_caption">The Crystals.</span> <span class="element-image\_\_credit">Photograph: Michael Ochs Archives/Getty Images</span> </figcaption> </figure>  <h2><strong>The Crystals – He Hit Me (And It Felt Like A Kiss)</strong></h2> <p>Co-songwriter Carole King now disowns husband Gerry Goffin’s 1962 lyrics, inspired by babysitter Little Eva’s justification for a boyfriend’s abuse.</p> <h2><strong>Elton John – Island Girl</strong></h2> <p>The now reliably woke, Sir Elt once asked “Island girl, what you wanting with the white man’s world?” and adopted cod patois.</p> <h2><strong>The Vapors – Turning Japanese</strong></h2> <p>A fine 70s power-pop hit about masturbation. Great! Apart from Asian stereotyping.</p> <h2><strong>Gary Glitter – What Your Mama Don’t See (Your Mama Don’t Know)</strong></h2> <p>The outcast Leader’s paedophilia convictions make this 1980 single particularly uncomfortable.</p>  <figure class="element element-image element--thumbnail" data-media-id="8ded244e3da29750b5c2ce630bb801b58fe94d4a"> <img src="https://media.guim.co.uk/8ded244e3da29750b5c2ce630bb801b58fe94d4a/692\_94\_1243\_1243/1000.jpg" alt="Katy Perry." width="1000" height="1000" class="gu-image" /> <figcaption> <span class="element-image\_\_caption">Katy Perry.</span> <span class="element-image\_\_credit">Photograph: Vianney Le Caer/REX/Shutterstock</span> </figcaption> </figure>  <h2><strong>Katy Perry – Ur So Gay</strong></h2> <p>The “I Kissed A Girl” singer uses “gay” as an insult and adds: “You don’t even like boys.”</p> <h2><strong>Frank Loesser – Baby It’s Cold Outside</strong></h2> <p>Covered by June Carter and Michael Buble, a 1944 song with allusions towards date rape.</p>\\
Guardian Small Business Network & <p>For all the support they promise startups, business accelerators are arguably not delivering. Too many startup founders are not getting to the finishing line of a big pay day on exit or stock market launch.</p> <p>The UK is ranked third in the world by the Organisation for Economic Cooperation and Development (OECD) for the amount of startups created but only 13th for the number that go on to become established medium-sized companies. Lack of access to financing as a business matures is clearly an issue.</p> <p>A <a href="https://www.gov.uk/government/uploads/system/uploads/attachment\_data/file/634338/financing\_growth\_in\_innovative\_firms\_consultation\_web.pdf">government report </a>has shown that fewer than one in 10 firms that obtain seed funding in the UK go on to receive later stage fourth round investment, compared with nearly a quarter in the US. Last week the Treasury announced it will set up a <a href="https://www.gov.uk/government/news/new-national-investment-fund-to-back-innovative-uk-firms">national investment fund</a> to address an estimated £4bn funding gap between US and British firms.</p> <p>But an often overlooked problem is that many startups take a disjointed short-term approach to growth which<em> </em>is killing the golden goose before it gets to market.<br></p> <aside class="element element-rich-link element--thumbnail"> <p> <span>Related: </span><a href="https://www.theguardian.com/small-business-network/2017/jul/25/nows-the-time-for-change-in-venture-capital-meet-the-unicorn-hunters">'Now's time for change in venture capital': meet the unicorn hunters</a> </p> </aside>  <p>Entrepreneurship has become a trendy career choice and the credit crunch has prompted people to start their own business as the jobs market has shrunk. Startups were formed at the record pace of <a href="http://www.telegraph.co.uk/business/2016/07/12/record-80-new-companies-being-born-an-hour-in-2016/">80 an hour</a> last year, according to research by StartUp Britain. But along with the boom in startups there has been a race to the bottom to get investment. As these businesses mature their performance drops off, in part due to a lack of long-term planning.</p> <p>In effect, they become zombie startups, the term for companies that keep going after funding runs out but don’t actually grow, and investors no longer see them as attractive nor worth a punt.</p> <p>Once startups gain funding they need to perform well and demonstrate that they are worthy of more follow-on investment up the chain. Quick-fix goals don’t work. We need startup founders equipped with the right management skills and who can create solid teams that attract investors.</p> <p>Having run a major equity-free business accelerator, Entrepreneurial Spark, for five years, I have seen beneath the veneer of the startup world. Entrepreneurial Spark helps to make entrepreneurs credible and investable, and produces an annual impact report on investments raised. But it is difficult to monitor what goes on after early seed rounds of investment as startups leave these programmes and go out into the world.</p>  <aside class="element element-pullquote element--supporting"> <blockquote> <p>Let’s stop lauding that one unicorn that makes it and start putting structures in place for more ventures to progress</p> </blockquote> </aside>  <p>For years tech startups have been pitching models that don’t generate cash for venture capital firms and angel investors, just customer data – in the hope that one day this data will be valuable. But many investors now demand that the businesses they invest in generate proper sales to real customers. As Guy Kawasaki, the Silicon Valley guru, declared in his book The Art of Rainmaking, “sales fix everything”.<br></p> <p>If we do not encourage our startups and founders to focus on long-term goals and performance, then the statistics will not improve, which will continue to have a damaging effect on the whole system of business building in the UK.</p> <p>Let’s stop lauding that one unicorn that makes it and start putting structures and support in place for more early stage ventures to progress up the snake and ladders board of investment.</p> <p><em>Jim Duffy is the co-founder of <a href="http://moonshot-academy.com/">Moonshot Academy</a></em></p> <p><strong>Sign up to become a member of the </strong><a href="https://register.theguardian.com/small-business/"><strong>Guardian Small Business Network here</strong></a><strong> for more advice, insight and best practice direct to your inbox.</strong></p>\\
Society & <p>A delay in agreeing a Brexit transition deal could harm NHS patients, a senior MP has warned.<br></p> <p>In a letter to the health secretary, Jeremy Hunt, the chair of the Commons health committee, Sarah Wollaston, said that any holdups could put patient care at risk.</p> <p>Wollaston said it would mean more health businesses diverting money towards contingency planning for a “disorderly” withdrawal from the European Union.</p> <p>“Patient care, both in the UK and Europe, is at risk of being compromised in the event of a disorderly Brexit. Businesses and services – like government – need to plan for all outcomes to avoid any disruption to the supply of medical products,” she said.</p> <p>The MP added that the health service and businesses, including those manufacturing and distributing medicines, “remain in the dark” about Britain’s exit from the EU. Many are planning for a worst-case scenario because “time is running out” for a transition deal to follow the UK’s formal exit in March 2019. </p> <p>There are hopes that a deal between Britain and the EU about the two-year transition period will be decided at March’s European council summit.<br></p> <p>Wollaston warned in her letter: “If the announcement, and details, of a transition period is delayed beyond March 2018, more businesses will be forced to invest money in contingency plans at the expense of this funding going towards advancing patient care. </p> <p>“This is an unnecessary cost and distraction, which should be avoided,” she said. <br><br>The Conservative MP also called on the government to agree with the EU a joint public statement on protecting the interests of patients in the event of a “no deal” Brexit that would see the <a href="https://www.theguardian.com/politics/2018/jan/29/brexit-could-leave-patients-unable-to-access-new-drugs">UK severing its ties with the European Medicines Agency</a>, which regulates the supply of medical products. <br></p> <p>“Failing this, and in the event that agreement to a transition is not reached by the end of March, the committee seeks a commitment from the government to make its own statement about the UK’s unilateral preparations for a no-deal situation,” she said.</p> <p>The government should also publish its contingency planning for healthcare to provide “clarity”, Wollaston said. </p>  <figure class="element element-image" data-media-id="9629f6b4308aff405ba217d01f445aca2625f597"> <img src="https://media.guim.co.uk/9629f6b4308aff405ba217d01f445aca2625f597/0\_212\_4256\_2555/1000.jpg" alt="Jonathan Ashworth" width="1000" height="600" class="gu-image" /> <figcaption> <span class="element-image\_\_caption">Jonathan Ashworth, shadow health secretary: ‘Sarah Wollaston is correct to warn time is running out, with many manufacturers now planning for a worst-case scenario.</span> <span class="element-image\_\_credit">Photograph: Rex/Shutterstock</span> </figcaption> </figure>  <p>She argued this would strengthen the UK’s negotiating position by demonstrating a “credible fallback position” and allow public scrutiny.</p> <p>Jonathan Ashworth MP, the shadow health and social care secretary, said that Wollaston’s “powerful warning” reaffirmed Theresa May’s astonishing complacency in securing the best possible Brexit deal for patients and staff. </p> <p>“It is critically important that patients’ access to crucial drugs is not in any way restricted once we leave the European Union, and Sarah Wollaston is correct to warn time is running out, with many manufacturers now planning for a worst-case scenario,” he said.</p> <p>A Department of Health and Social Care spokesperson said: “Robust and thorough planning is under way to ensure patients in the UK continue to access the best and most innovative medicines. We have made clear EU rules will continue to apply during any implementation period to ensure a smooth transition.”<br><br tabindex="-1"></p>\\
\bottomrule
\end{tabular}}
\end{table}

\rowcolors{2}{white}{white}

\rowcolors{2}{gray!6}{white}

\begin{table}[!h]

\caption{\label{tab:data_collection}Table containing SPORTS  News}
\centering
\resizebox{\linewidth}{!}{
\begin{tabular}[t]{ll}
\hiderowcolors
\toprule
sectionName & body\\
\midrule
\showrowcolors
Business & <p>Sports Direct’s board has said it did not know about or authorise an apparent attempt to secretly record a group of MPs who paid a surprise visit to its warehouse.</p> <p><a href="https://www.theguardian.com/business/2016/nov/07/six-mps-surprise-visit-sports-direct-shirebrook-warehouse">Six MPs from the business select committee arrived at the site in Shirebrook, Derbyshire, on Monday after giving the company an hour’s notice</a>. They were accepting an invitation made by the founder and chief executive, Mike Ashley, when he appeared before the committee in June. </p> <p>After they were shown round, the MPs went to a meeting room where they claimed a small camera used to record the day’s proceedings was placed under a tray of sandwiches.</p> <p><a href="http://otp.investis.com/clients/uk/sports\_direct1/rns/regulatory-story.aspx?cid=723\&amp;newsid=815732">In a statement to the stock exchange</a>, Sports Direct’s board said it had not been proved that the camera was to be used to record the MPs and complained that the episode had overshadowed good things that came from the visit.</p> <p>The company added: “The board would like to make it clear that it did not authorise or have any knowledge of the possible recording device.”</p> <aside class="element element-rich-link element--thumbnail"> <p> <span>Related: </span><a href="https://www.theguardian.com/business/2016/nov/07/six-mps-surprise-visit-sports-direct-shirebrook-warehouse">Sports Direct accused of secretly recording MPs during warehouse visit</a> </p> </aside>  <p>Sports Direct said it was delighted MPs had “finally accepted the company’s invitation to visit the Sports Direct’s warehouse on an unannounced basis” but that the board was disappointed they chose a day when they knew Ashley would be away from Shirebrook. </p> <p><a href="https://www.theguardian.com/business/2016/mar/26/karen-byers-sports-direct-hard-hitting-head-retail">Karen Byers, global operations head and one of Ashley’s closest associates</a>, was understood to have shown the MPs round in Ashley’s absence.</p> <p>Ashley invited the MPs to Shirebrook <a href="https://www.theguardian.com/business/2015/dec/09/how-sports-direct-effectively-pays-below-minimum-wage-pay">after the Guardian’s investigation exposed how Sports Direct workers were being paid less than the minimum wage</a>. The business committee subsequently <a href="https://www.theguardian.com/business/2016/jul/22/mike-ashley-running-sports-direct-like-victorian-workhouse">accused Ashley of running the business like a Victorian workhouse in a report published in July</a>.</p> <p>The board, led by embattled chairman Keith Hellawell, said it understood the MPs met many staff members who were positive about the company and that employees expressed unhappiness about how the company had been portrayed. </p> <p>Ashley said: “I stand firmly behind the people of Sports Direct, who through no fault of their own have been made a political football by MPs and unions.”<br></p> <p>Anna Turley, one of the MPs who visited Shirebrook, said they were made to feel unwelcome and were taken on a “wild goose chase” around the vast site before being shown the warehouse at the centre of the row about working practices.</p> <p>Turley told the BBC’s Today programme: “We didn’t expect them to be overjoyed when we were there but would say: ‘This is the same as any other day. We are proud of our warehouse, we are proud of our staff, come and meet them and say hello.’”</p> <p>Turley said conditions at Shirebrook were dark, dingy and chaotic but that the MPs had not expected things to be perfect because Ashley had admitted things needed to improve. She said the camera incident had left her and her colleagues uneasy about the treatment of employees.</p> <p>“If they do this to us as parliamentarians, what do they do to staff who aren’t able to answer back, who aren’t unionised and who fear for their jobs?”</p>\\
Sport & <p>On Sunday, the Houston Texans made it clear what they thought of the team’s owner apparently describing NFL players as “inmates” when <a href="https://www.theguardian.com/sport/2017/oct/29/houston-texans-players-protest-bob-mcnair-inmates-comments">the majority of the roster knelt for the national anthem</a>. The Texans defensive end Jadeveon Clowney followed that up by dressing in a prison jumpsuit at a Halloween party on Monday night.</p> <p>The Texans, however, have denied it was a jab at owner Bob McNair. “[There] was no hidden meaning behind his Halloween costume,” the Texans’ senior director of communications, Amy Palcic, <a href="http://www.chron.com/sports/texans/article/Jadeveon-Clowney-texans-inmate-Halloween-costume-12320303.php?ipid=happening">told the Houston Chronicle</a>. “He was not taking a shot at anyone. It was just that – a costume at a Halloween party.”</p>  <figure class="element element-tweet" data-canonical-url="https://twitter.com/JordanHeckFF/status/925410648921657344">  <blockquote class="twitter-tweet"><p lang="en" dir="ltr">Jadeveon Clowney dressed up as an inmate for Halloween, likely a shot at recent comments made by the Texans owner. <a href="https://t.co/BkGrerWxTt">pic.twitter.com/BkGrerWxTt</a></p>\&mdash; Jordan Heck (@JordanHeckFF) <a href="https://twitter.com/JordanHeckFF/status/925410648921657344?ref\_src=twsrc\%5Etfw">October 31, 2017</a></blockquote>  </figure>  <p>Clowney, the No1 overall pick in the 2014 draft, has not issued a statement.</p> <p>McNair angered Texans players last week when he was quoted in an ESPN story as saying “we can’t have the inmates running the prison”, during an NFL owners meeting, where issues such as player protests were discussed.</p> <p>McNair <a href="https://www.theguardian.com/sport/2017/oct/27/nfl-owner-bob-mcnair-inmates-running-the-prision">subsequently apologised for his comments</a> and said he was not referring to players, although another owner at the meeting said it was hard to understand who else the billionaire was talking about.</p> <p>“I regret that I used that expression,” said McNair. “I never meant to offend anyone and I was not referring to our players. I used a figure of speech that was never intended to be taken literally. I would never characterize our players or our league that way and I apologize to anyone who was offended by it.”</p>\\
Business & <p>Sports Direct’s efforts to rehabilitate its reputation after a year-long scandal over working practices took a farcical turn on Monday when the retailer was accused of secretly recording a group of MPs visiting its controversial warehouse.</p> <p>Six MPs from the business select committee, which in the summer published a scathing report into company, also claimed the company treated the parliamentarians with “hostility” and implemented “diversionary tactics” as they carried out an impromptu inspection of the sportswear group’s warehouse.</p> <p>The allegations of secret recordings were made after the parliamentarians had finished their tour, when they adjourned to a meeting room in Sports Direct’s offices in order to discuss the day’s events and take refreshments.</p> <aside class="element element-rich-link element--thumbnail"> <p> <span>Related: </span><a href="https://www.theguardian.com/business/2016/jul/22/mike-ashley-running-sports-direct-like-victorian-workhouse">Mike Ashley running Sports Direct like 'Victorian workhouse'</a> </p> </aside>  <p>Anna Turley, the Labour MP for Redcar and a member of the committee, told the Guardian: “The sandwich lady came in and she put them on [a stool] in the corner. I watched her do it because I thought it was a bit weird her putting them in the corner of the room rather than where we were sitting. I saw her kneel down and put a device under the stool.</p> <p>“I watched her and waited until she got out of the room and I went over [to the stool] and it was a camera that [Sports Direct representatives] had with them on the visit, [when] they were recording every question we asked and everything we said.”</p>  <figure class="element element-tweet" data-canonical-url="https://twitter.com/annaturley/status/795689611951542272">  <blockquote class="twitter-tweet"><p lang="en" dir="ltr">Here is the camera I found which was placed under the stool on which the sandwiches were placed for our private meeting at <a href="https://twitter.com/hashtag/sportsdirect?src=hash">\#sportsdirect</a> <a href="https://t.co/aD6StnX5T9">pic.twitter.com/aD6StnX5T9</a></p>\&mdash; Anna Turley MP (@annaturley) <a href="https://twitter.com/annaturley/status/795689611951542272">November 7, 2016</a></blockquote>  </figure>  <figure class="element element-tweet" data-canonical-url="https://twitter.com/annaturley/status/795694275493294080">  <blockquote class="twitter-tweet"><p lang="en" dir="ltr">Here is another pic from a colleague. This is where it was hidden \&amp; where I found it before picking it up \&amp; placing on top as previous pic. <a href="https://t.co/HPIoUJanQ4">pic.twitter.com/HPIoUJanQ4</a></p>\&mdash; Anna Turley MP (@annaturley) <a href="https://twitter.com/annaturley/status/795694275493294080">November 7, 2016</a></blockquote>  </figure>  <p>Iain Wright, chairman of the business select committee, said the whole day had reminded him of visiting factories in China.</p> <p>He added: “I knew it wasn’t going to be the finished product. I knew we were going to see work in progress, as it were. But the hostility, the controlling manner, with which they dealt with us, the diversionary tactics ... I wanted to to go to [the old part of the warehouse which had been at the centre of the scandals] and it took us about three hours to get there”.</p> <p>Wright, along with five colleagues, had arrived at the group’s Shirebrook, Derbyshire headquarters at about midday on Monday, after notifying the company earlier that morning that the MPs were arriving for a surprise tour.</p> <p>The visit had come after Wright’s committee had accused Mike Ashley, the billionaire Sports Direct founder, of running the business like a Victorian workhouse <a href="https://www.theguardian.com/business/2016/jul/22/mike-ashley-running-sports-direct-like-victorian-workhouse">in a report published in July</a>.</p> <p>The MPs concluded that Ashley had built his success on a business model that treats workers “without dignity or respect”, after they launched an inquiry following <a href="https://www.theguardian.com/business/2015/dec/09/how-sports-direct-effectively-pays-below-minimum-wage-pay">an undercover Guardian investigation</a> last year that exposed how Sports Direct workers were being paid less than the minimum wage.</p> <p>It is understood Ashley was not present at Shirebrook for the inspection and the MPs were instead shown around by the group’s global operations head, <a href="https://www.theguardian.com/business/2016/mar/26/karen-byers-sports-direct-hard-hitting-head-retail">Karen Byers</a>.</p> <p>In his excoriating appearance before the committee in June, Ashley pledged to look at areas where the company might improve conditions of its workers, including looking at the “<a href="https://www.theguardian.com/business/2015/dec/09/how-sports-direct-effectively-pays-below-minimum-wage-pay">six strikes and you’re out</a>” policy which threatened workers with the sack after a series of perceived crimes such as long toilet breaks. He also pledged to review if the company needs to engage so many of its workers’ temporary contracts.</p> <p>The committee has no real powers to punish the tycoon if he fails to deliver on his pledges, although Wright had promised to “continue to hold Mr Ashley’s feet to the fire, in as constructive a manner as possible, checking on the progress he makes on improving working conditions for workers at his premises”.</p> <p>In September, the company announced a <a href="https://www.theguardian.com/business/2016/sep/06/sports-direct-inquiry-key-points-working-practices">suspension of the six strikes policy</a> and a trial to move some temporary staff on to permanent contracts. It had previously addressed the minimum wage breach by increasing the pay of warehouse workers, while it had also committed to reimbursing with backpay those affected.</p>  <figure class="element element-image" data-media-id="233b6934f90d0878c014c6a99d6578b2c31e9dc0"> <img src="https://media.guim.co.uk/233b6934f90d0878c014c6a99d6578b2c31e9dc0/1376\_582\_2868\_1721/1000.jpg" alt="Iain Wright, chair of the business select committee" width="1000" height="600" class="gu-image" /> <figcaption> <span class="element-image\_\_caption">Iain Wright, business committee chair.</span> <span class="element-image\_\_credit">Photograph: Martin Godwin for the Guardian</span> </figcaption> </figure>  <p>The retailer has suffered a stream of criticism over its working practices, with officials from the Unite union campaigning against a strict culture in the warehouse that has made workers afraid to speak out over low pay and conditions in case they lose their jobs.</p> <p>Last year, <a href="https://www.theguardian.com/business/2015/dec/09/sports-direct-staff-shirebrook-strikes-policy-sick-leave">primary schoolteachers told the Guardian</a> that parents working at Sports Direct were too frightened to take time off work, resulting in pupils attending school while ill or returning home to empty houses.</p> <p>Wright and Turley were joined on his visit by fellow committee members Amanda Solloway, Peter Kyle, Michelle Thomson and Craig Tracey.</p> <p>Ashley appeared in front of the select committee in June having initially challenged the authority of parliament to summon him.</p> <p>He called the parliamentarians “a joke”, saying MPs needed to come to visit Shirebrook for themselves, but eventually backed down. He then extended an open invitation for MPs to visit the site.</p> <p>Sports Direct did not return phone calls seeking comment.</p>\\
Business & <p>Sports Direct has become the media’s poster-child for corporate governance failings. The retailer’s announcement that it is to appoint a worker to the board (<a href="https://www.theguardian.com/business/2017/mar/09/sports-direct-workers-representative-mike-ashley" title="">Sports Direct to recruit workers’ representative after scandals,</a> theguardian.com, 9 March) might please\&\#xa0;some Whitehall officials and certain elements of the shareholder community, but let’s be clear: it will do little, if anything, to resolve the corporate governance problems at the\&\#xa0;company.</p> <aside class="element element-rich-link"> <p> <span>Related: </span><a href="https://www.theguardian.com/business/2017/mar/09/sports-direct-workers-representative-mike-ashley">Sports Direct to recruit workers' representative after scandals</a> </p> </aside>  <p>All boards, in particular those of FTSE\&\#xa0;350 companies, should try to become more inclusive and representative of the society in which they operate. For that reason alone, Sports Direct’s announcement is welcome. However, the worker representative will be incapable of checking the authority of Mike Ashley,\&\#xa0;a\&\#xa0;board director at the company and the majority shareholder.</p> <p>For real, long-lasting change for the good of all investors, customers and employees, a radical overhaul of Sports Direct’s governance is necessary. It is not just a case of launching an independent review into its corporate governance, as Ashley did in fact do last year. There needs to be more independent oversight of Ashley, who, because of his shareholding, effectively controls the company.</p> <p>This is not a simple question of appointing a worker to the board. At present the board has five independent directors, four of whom have been in situ for more than five years.</p> <p>Time and time again they have failed to rein in Ashley and I fail to see how a worker could do anything more.</p> <p>Without fundamental and structural reform, a wholesale change in the behaviour of the largest sports retailer in\&\#xa0;the UK is very hard to imagine.<br><strong>Oliver Parry </strong><br><em>Head of corporate governance, Institute of Directors</em></p> <p><strong>•<em> Join the debate – email </em></strong><a href="mailto:guardian.letters@theguardian.com" title=""><strong><em>guardian.letters@theguardian.com</em></strong></a></p> <p><strong>•<em> Read more Guardian letters – </em></strong><a href="https://www.theguardian.com/tone/letters" title=""><strong><em>click here to visit gu.com/letters</em></strong></a></p>\\
Television \& radio & <h2><strong>54 Hours: The Gladbeck Hostage Crisis </strong></h2> <p><strong>9pm, BBC Four</strong></p> <aside class="element element-rich-link element--thumbnail"> <p> <span>Related: </span><a href="https://www.theguardian.com/world/2018/mar/09/german-bank-raid-and-hostage-grab-of-80s-plays-out-in-tv-drama">German bank raid and hostage-grab of 80s plays out in TV drama</a> </p> </aside>  <p>This taut and brutal two-parter attracted headlines and huge ratings when it aired in Germany last year. It tells a story strikingly familiar to German viewers but largely unknown here: the events of 16 to 18 August 1988, when armed bank robbers instigated a hostage situation that spilled across German and Dutch state lines. Tonight’s opener hurls viewers without warning into the centre of the crisis as a botched police raid leads to a standoff, an escape and, ultimately, violence. Continues next week.<em> Gwilym Mumford</em></p> <h2><strong>Invictus Games 2018\&\#xa0;</strong></h2> <p><strong>5.25pm, BBC One</strong></p> <p>The multi-sport event for injured or sick armed forces personnel and veterans takes place this year in Australia. We begin with the opening ceremony at the Sydney Opera House, with action during the week to include swimming, cycling, archery, powerlifting, rugby, athletics, sailing\&\#xa0; and sitting volleyball. <em>David Stubbs\&\#xa0;</em></p> <h2><strong>Black Hollywood: “They’ve Gotta Have Us” </strong></h2> <p><strong>9pm, BBC Two</strong></p> <p>Part two of Simon Frederick’s superb overview of black cinema boils down to a history of tokenism: the impact of movies such as Boyz N the Hood dissipated, with the 90s and 00s short on proper representation. Denzel\&\#xa0;Washington and Viola Davis are among those lauded for punching through. <em>Jack Seale</em></p> <h2><strong>Mr Dynamite: The Rise of James Brown\&\#xa0;</strong></h2> <p><strong>9pm, Sky Arts</strong></p> <p>A look at the life and times of the former “hardest-working man in showbusiness” from director Alex Gibney. Tracing the Godfather of Soul’s extraordinary career from the Chitlin’ Circuit onwards, this workmanlike effort is mainly distinguished by the amount of rare and previously unseen footage it offers. <em>Ali Catterall</em></p> <h2><strong>Killing Eve </strong></h2> <p><strong>9.25pm, BBC One</strong></p> <aside class="element element-rich-link element--thumbnail"> <p> <span>Related: </span><a href="https://www.theguardian.com/tv-and-radio/2018/oct/12/from-fridging-to-nagging-husbands-how-killing-eve-upturns-sexist-cliches">From fridging to nagging husbands: How Killing Eve upturns sexist cliches</a> </p> </aside>  <p>Villanelle may have snuffed out MI5’s best chance of tracking down the Twelve by killing Frank, but her botched offing of Nadia has handed Eve and co a key lead. Both parties descend on the Russian prison where Nadia is being held in an episode that lacks the zip of recent instalments, but is still superior to most TV drama. <em>GM</em></p> <h2><strong>My Favourite Sketch</strong></h2> <p><strong>10pm, Gold</strong></p> <p>Sue Perkins is Sally Phillips’s final guest of a series that has been more enjoyable than any clip show has a right to be. Among Perkins’s choices are Not the Nine O’Clock News’s Gerald the Gorilla, Will Ferrell’s “more cowbell” SNL skit and a vintage bit from Julie Walters and the much-missed Victoria Wood. <em>GM</em></p> <h2>Film choice</h2>  <figure class="element element-image" data-media-id="e446f76f2d3659637fe342ddedcb9cab2f764005"> <img src="https://media.guim.co.uk/e446f76f2d3659637fe342ddedcb9cab2f764005/0\_0\_2928\_1758/1000.jpg" alt="Thandie Newton and Chiwetel Ejiofor in Half of a Yellow Sun." width="1000" height="600" class="gu-image" /> <figcaption> <span class="element-image\_\_caption">Thandie Newton and Chiwetel Ejiofor in Half of a Yellow Sun.</span> <span class="element-image\_\_credit">Photograph: Allstar/Monterey Media</span> </figcaption> </figure>  <p><strong>Half of a Yellow Sun, 12.35am, BBC Two</strong></p> <p>A flawed if heartfelt adaptation of Chimamanda Ngozi Adichie’s novel, set during the 60s Biafran war. There’s a fine pair of leads in Chiwetel Ejiofor’s self-important professor Odenigbo and Thandie Newton as his partner Olanna. But trying to marry post-imperial conflict with the couple’s domestic trials proves tricky. <em>Paul Howlett</em></p> <h2>Today’s best live sport</h2> <p><strong>ODI Cricket: Sri Lanka v England, 6am, Sky Sports Main Event<br></strong>The fourth in a five-match series.\&\#xa0;</p> <p><strong>Premier League Football: Chelsea v Manchester United, 12noon, Sky Sports Main Event<br></strong>Huddersfield v Liverpool follows at 5pm on BT Sport 1.\&\#xa0;</p> <p><strong>Champions Cup Rugby Union: Munster v Gloucester, 12.30pm, Channel 4<br></strong>Wasps v Bath (3pm) and Saracens v Lyon (5.15pm) air on BT Sport 2.</p>\\
Sport & <p>Sport can unite people from all walks of life. It can bring together all races, colors, creeds and orientations; it can bridge the gap between the most polarized political opponents.</p> <p>It’s the kind of pablum we hear every time the Olympic Games or World Cup rolls around. But there’s some truth to it, too. During the Christmas ceasefire along the Western Front, German and allied troops engaged in football matches. And then there was that time you high-fived a complete stranger at a sporting event without knowing his complete voting history, let alone the last time he washed his hand.</p> <aside class="element element-rich-link element--thumbnail"> <p> <span>Related: </span><a href="https://www.theguardian.com/sport/blog/2017/mar/02/daytona-500-infield-nascar">Finding America at the Daytona 500: why Nascar's decline is fake news | Bryan Armen Graham</a> </p> </aside>  <p>It is with this idealistic sense of community that I always endeavor to watch sports, if for no other reason than to gain a brief escape from the rest of our hyper-partisan culture. But then I happened to discover that there is something called Breitbart Sports. Yes, Breitbart.com, the flagship website of the nationalistic, alt-right. The angry white man to the New York Times’ old gray lady, the publication whose headlines (and longtime publisher) have somehow come to inform the most powerful and misinformed man in the world, is slinging sports news and takes with the rest of us.</p> <p>The idea of sport as a uniter made it through the first world war, but could it withstand an association with Breitbart? The site, remember has produced such headline hits as “The Solution To Online ‘Harassment’ is Simple: Women Should Log Off”; “Birth Control Makes Women Unattractive and Crazy”; and “There’s No Hiring Bias Against Women in Tech, They Just Suck At Interviews”.</p> <p>In fairness, Breitbart has also produced headlines that are anti-other-people-than-just-women. It has also never claimed to have a goal of uniting people, as proven by the outlet’s longtime championing of professional “provocateur” Milo Yiannopoulos. How would the media organization that welcomed Curt Schilling from ESPN cover the world of ball and stick, hits and kicks? I decided to find out by reading Breitbart’s sports section for an entire week.</p> <h2>Day 1</h2> <p>The first story in Breitbart’s sports section for the day is <a href="http://www.breitbart.com/sports/2017/03/01/nod-gender-inclusivity-university-minnesota-drops-king-queen-homecoming-titles/">the news</a> that the University of Minnesota is removing the King and Queen titles from their Homecoming Court in order to be “gender neutral”. Homecoming is only tangentially related to football, and the article doesn’t address the Gophers’ chances of taking the Big Ten West under new head coach PJ Fleck, but it’s on the sports page nonetheless. One of the top comments on the story reads: “Boys have a penis. Girls have a vagina.” We’re off and running.</p> <p>The next story of the day is about former NBA center Amar’e Stoudemire saying that he would refuse to shower with a gay team-mate, a story <a href="http://\%20https//www.theguardian.com/sport/2017/mar/01/amare-stoudemire-homophobic-slur-nba-israeli-basketball">covered later</a> by the Guardian. Breitbart also followed up on a story that ran first in the Guardian about New Orleans Saints players being turned away from a London nightclub for being “too urban”, and highlighted the club’s statement that the denial of entrance had anything to do with race. Then it was time for an article on Breitbart’s sports muse: Colin Kaepernick. <a draggable="true" href="http://www.breitbart.com/sports/2017/03/01/will-anthem-kneeling-hurt-colin-kaepernick-free-agency/">In a piece</a> about how anthem protests could hurt free agents, we get the line: “Will GMs be crying out for [his] services after he disrespected the flag in 2016?”</p> <p>As anyone knows who has heard Kaepernick or other NFL players speak out about their protests, they said kneeling was in no way to disrespect the flag or the military, but to bring attention to issues that concerned them in America. But maybe Breitbart mentioned that in one of the <a draggable="true" href="http://www.breitbart.com/search/?s=kaepernick">other 14,400 mentions</a> of the middling QB on their site.</p> <h2>Day 2</h2> <p>Last Thursday featured <a href="http://www.breitbart.com/sports/2017/03/02/report-colin-kaepernick-stand-national-anthem-2017/">10 posts</a> on Breitbart’s sports section, four of them about Kaepernick. This was the day the free agent announced he would not kneel for the anthem in 2017. There was the bare-bones news on Kaepernick ending his protest, a clip of Stephen A Smith calling him “incredibly opportunistic”, a clip of Colin Cowherd dubbing him a “sellout” and, to complete the yelling head trifecta, video from Skip Bayless’s debate show.</p>  <figure class="element element-image" data-media-id="540c776d55f4b224021f485519dd94c5ab69d2b9"> <img src="https://media.guim.co.uk/540c776d55f4b224021f485519dd94c5ab69d2b9/0\_304\_4776\_2866/1000.jpg" alt="Colin Kaepernick: why is Breitbart so concerned with a middling NFL quarterback?" width="1000" height="600" class="gu-image" /> <figcaption> <span class="element-image\_\_caption">Colin Kaepernick: why is Breitbart so concerned with a middling NFL quarterback?</span> <span class="element-image\_\_credit">Photograph: Troy Wayrynen/USA Today Sports</span> </figcaption> </figure>  <p>When not covering Kaepernick or the plight of the straight, Christian white man in America, Breitbart can often be counted on <a href="http://www.breitbart.com/search/?s=\%22stick+to+sports\%22">to lead</a> the “stick to sports” movement. This crowd has now gone all-in on the narrative that ESPN is losing subscribers because the network has shifted to a liberal voice. This is a stunningly ludicrous idea. Not that ESPN has shifted left. It has. But that a sports network’s politics would cause anyone to decide to stop watching sports entirely – and also every other TV channel they have, including Fox News. Only a madman would do that, and it’s a fact that the impact of cord-cutting is being felt throughout television and across sports. While the right blamed Kaepernick for lower NFL television ratings at the start of the 2016 season, there was no explanation of how the 49ers quarterback caused a <a href="https://www.theguardian.com/sport/blog/2017/mar/02/daytona-500-infield-nascar">ratings plummet</a> in Nascar, a racing circuit that has a Trump-supporting CEO and millions of Trump-supporting fans.</p> <h2>Day 3</h2> <p>Ten more posts on the website’s sports page, six of them related to politics. For an outlet that so wants everyone to \#StickToSports, Breitbart Sports struggles to meet that demand.</p> <h2>Days 4-5</h2> <p>On the weekend that opened with President Trump going on a Twitter tirade before hitting the links at his Florida resort, the president’s golf game didn’t get any coverage on Breitbart Sports. (However, you can still read <a href="http://www.breitbart.com/sports/2016/10/28/good-king-obamas-golf-trip-tiger-woods-cost-taxpayers-3-6-million-dollars/">this story</a> from October on how President Obama playing golf with Tiger Woods cost taxpayers \$3.6m. Sad!) The weekend sports coverage was light – perhaps due to the <a href="http://www.breitbart.com/big-government/2015/02/03/7-reasons-millennials-are-the-worst-generation/">poor work ethic</a> of millennials? – with only nine total posts over two days, a third of which were from news wires. Of the six original posts, two were about politics and a third was about Tim Tebow and American opinions on him are as divided as any issue, so let’s just call it an even three.</p> <h2>Day 6</h2> <p>Nine posts to open the week, with ESPN getting the heavy focus. Breitbart is very much not a fan of ESPN. But it’s not for the reasons millions of others might find the network grating, such as contrived, time-filling debates and storylines or Chris Berman’s sweaty buffoonery. No, Breitbart’s demo has come to see ESPN as the symbol of the liberal media trying to shove its viewpoints down the throats of Real America. The Breitbart audience wants sports to be a safe space for their worldview and ESPN has not complied, making many an alt-right snowflake irate. The most commented-on story of the day <a href="http://www.breitbart.com/sports/2017/03/06/big-layoffs-coming-financially-struggling-espn">is a piece</a> celebrating a round of layoffs set to come to the network – and that’s followed up two posts later by more coverage of how ESPN’s politics have supposedly led to cord-cutting, a viewpoint pulled primarily from a blogpost by proud sports media troll Clay Travis.</p> <p>There is no coverage of how cord-cutting has hit right-leaning sports network FS1, how that network just failed to renew the contracts of Jay Onrait and Dan O’Toole, or how FS1’s ratings remain abysmal and a fraction of what ESPN pulls on its most liberal day.</p> <h2>Day 7</h2> <p>The final day of Breitbart Sports immersion and the rhythm and viewpoint was now expected. A Tebow article, a couple of pieces touching on politics, a Kaepernick article and an anti-ESPN piece that brought up anthem protests again. There was also <a href="http://www.breitbart.com/sports/2017/03/07/nfl-free-agency-good-year-go-left-tackle-shopping/">one piece</a> that \#StuckToSports and did a good job informing me on the NFL left tackle free agency market. It got one Facebook share and one comment, a good bit below an article earlier in the day on Nike releasing a line of hijabs that received 8,583 shares and more than 1,300 comments. The Breitbart audience clearly isn’t into anything celebrating the left, even if it’s left tackles.</p> <p>My journey into Steve Bannon’s sports man-cave complete, I can say that regular readers of Breitbart Sports will probably never be overcome by any unifying feeling from sports – not when a former 49ers backup, ESPN and the left are so clearly dead-set on destroying the American way of life. There will be no friendly game of soccer during the war for America’s soul, especially considering soccer is so loved by liberals. That could be seen as caving to the left.</p> <p>Like countless other sites on the cluttered web, Breitbart’s sports page just yells to its choir. While liberals might be delicate snowflakes, Breitbart Sports is as predictable and boring as can be. At least sports themselves are still fun and exciting.</p>\\
\bottomrule
\end{tabular}}
\end{table}

\rowcolors{2}{white}{white}

\rowcolors{2}{gray!6}{white}

\begin{table}[!h]

\caption{\label{tab:data_collection}Table containing ENTERTAINMENT  News}
\centering
\resizebox{\linewidth}{!}{
\begin{tabular}[t]{ll}
\hiderowcolors
\toprule
sectionName & body\\
\midrule
\showrowcolors
Media & <p>The US journalist who revealed that Harvey Weinstein faced allegations of sexual abuse stretching over decades is to present a new late-night show on Channel 4.</p> <p>Ronan Farrow <a href="https://www.theguardian.com/film/2017/oct/10/harvey-weinstein-allegations-mount-as-three-women-accuse-him-of">published an 8,000-word article in the New Yorker</a> earlier this week in which 13 women made detailed allegations against the film executive. </p> <p>Farrow is the son of the actor Mia Farrow and the film director Woody Allen, and used to work in the Obama administration.</p> <p>He is working on the new show with Madeleine Smithberg, who is one of the creators of The Daily Show in the US, which is presented by Trevor Noah.</p> <p>Channel 4 said the programme intended to offer a “satirical take on the UK as seen through the eyes of America”, and it wanted to build on the <a href="https://www.theguardian.com/tv-and-radio/2016/sep/17/how-the-last-leg-had-last-laugh-paralympic-games">success of The Last Leg</a>, which is one of the closest programmes on British television to the late-night shows in the US.</p> <p>The broadcaster’s entertainment division is working on the show with Soshefeigh, a Canadian production company.</p> <p>Ed Havard, the head of entertainment at Channel 4, said the combination of Soshefeigh, Farrow and Smithberg “taps into some of the strongest satirical and journalistic pedigree in the US, and we’re excited about the possibility of bringing that perspective to a UK audience. </p> <p>“This is part of our wider strategy of identifying new and emerging talent – on and off screen – who we can build satirical and topical formats around.” </p>\\
Society & <p>How are you spending your summer? Sweltering in the heat on the train to work? Taking the kids to the cinema? Or catching a flight somewhere to unwind?</p> <p>If you’re disabled, you may encounter problems with any of these, as transport and recreation continue to be plagued by poor treatment. The BBC’s security correspondent, Frank Gardner, who uses a wheelchair, was <a href="https://twitter.com/frankrgardner/status/1026574742319517696?s=21">stranded on a plane</a> at Heathrow airport this month (a repeated occurrence for him). Last month, the comedian Tanyalee Davis spoke of being <a href="https://www.theguardian.com/society/2018/jul/17/comedian-humiliated-as-guard-disputes-use-of-trains-disabled-space">“harassed and humiliated”</a> for using a disabled space on a train for her mobility scooter, and ended up being trapped as she was taken 50 miles out of her way. Even watching a film can end in discrimination. In May, the British Film Institute had to apologise after staff <a href="https://www.theguardian.com/society/2018/apr/30/woman-with-aspergers-ejected-from-cinema-for-laughing-at-western">forcibly removed a woman</a> with Asperger syndrome from a screening in what onlookers described as a disgusting sign of “naked intolerance”. </p> <p>These aren’t rare incidents. On Thursday, the disability charity Scope publishes figures showing the extent of social exclusion faced by disabled people. This follows the Equality and Human Rights Commission warning in 2016 that<a href="https://www.theguardian.com/society/2016/jul/19/people-with-disabilities-treated-like-second-class-citizens-says-watchdog"> Britain’s failure to implement disability rights</a> and address poor access amounts to treating disabled people as “second-class citizens”. </p> <p>We are not simply disabled by our bodies but by the way society is organised. It isn’t my use of a wheelchair that makes my life disabled, it’s the fact not all buildings have a ramp.</p> <p>Back in the 1980s and early 90s, disabled campaigners used <a href="https://www.theguardian.com/society/2015/nov/04/disabled-people-fight-equal-rights-exhibition-manchester">direct action and lobbying</a> to protest for civil rights, from accessible transport to entertainment venues. Anyone who remembers images of wheelchair users chaining themselves to buses in London as police moved in to kettle them, knows equal rights under the law have been hard won.</p> <p>Decades later, I can’t help but think we’re edging towards a tipping point again. The Equality Act, the product of the campaigning in the 90s, demands that buildings make “reasonable adjustments” to provide disabled access, but it’s widely unenforced. Britain’s infrastructure, from the train network to public toilets, can be equally inaccessible.</p> <p>But there are some signs of progress. Last week, the UK’s first <a href="https://purpletuesday.org.uk/">accessible shopping day</a> was announced. On 13 November, the government initiative backed by brands such as Argos, Barclays, Marks \&amp; Spencer and Sainsbury’s will feature new disability-friendly ways to make it easier for disabled people to go to the shops. </p> <p><a href="https://www.gov.uk/government/news/high-street-could-be-boosted-by-212-billion-purple-pound-by-attracting-disabled-people-and-their-families">Research by the Department for Work and Pensions</a> puts shopping and eating and drinking out as the most difficult experiences for disabled people. The government’s <a href="https://www.gov.uk/government/publications/inclusive-transport-strategy">Inclusive Transport Strategy</a>, launched last month, included £2m <a href="http://www.btaloos.co.uk/?p=1946">to install changing places toilets</a> in England’s motorway services that will have adult-sized changing beds, hoists to help lift people out of wheelchairs and extra room. This is long overdue and now needs to be rolled out to all large train stations, airports, and shopping centres. </p> <p>Too many women are being <a href="https://www.theguardian.com/society/2018/aug/06/disabled-women-surgery-catheter-accessible-toilets">forced to have unnecessary surgery</a> due to the lack of toilet provision; both male and female readers have told me they routinely use “adult nappies” on long journeys, despite not being incontinent, because stations don’t have facilities. The alternative is to never travel.</p> <p>The government’s strategy also includes making “up to £300m” available to extend its programme to make <a href="https://www.networkrail.co.uk/communities/passengers/station-improvements/access-for-all/">railway stations more accessible</a>. But Freedom of Information requests by the news site Disability News Service show the Conservatives <a href="https://www.disabilitynewsservice.com/figures-finally-show-how-government-slashed-spending-on-rail-access-scheme/">reduced spending</a> on the scheme over the last five years (from £81.1m in 2013-14 to just £14.6m last year). The shift to automation and staff cuts also mean smaller train stations are <a href="https://www.disabilitynewsservice.com/government-repeatedly-ignores-its-own-advisers-on-toxic-train-access/">increasingly unstaffed</a> and essentially prohibit disabled people who need assistance.</p> <aside class="element element-rich-link element--thumbnail"> <p> <span>Related: </span><a href="https://www.theguardian.com/society/2018/jun/20/society-weekly-email-newsletter-sign-up">Sign up for the Society Weekly email newsletter</a> </p> </aside>  <p>And the positive measures do not make up for <a href="https://www.theguardian.com/society/2013/mar/27/welfare-cuts-disabled-people">drastic cuts</a> in services and support in recent years. Making shops accessible is all well and good, but social care cuts mean many disabled people can’t leave the house.</p> <p>Let’s celebrate the gains but the fight has to go on. Equal rights for disabled people are still a long way off.</p>\\
Television \& radio & <p>The Bake Off contestants might have been revealed today but, MasterChef Australia, currently celebrating its 10th year running, is the most beautiful culinary show on TV. It not only stands out among food shows – although even just as a masterclass in diverse cuisines it ranks pretty highly – but, for sheer entertainment and emotional bangs per buck, it delivers as well as any box set series. Which is fortunate, as it requires quite an investment from you. Each season is something like 65 hours, and that’s before you factor in the time you’ve spent googling current and former contestants and their progress in life because you’re so invested in it.</p> <aside class="element element-rich-link element--thumbnail"> <p> <span>Related: </span><a href="https://www.theguardian.com/tv-and-radio/2018/feb/23/master-chef-returns">MasterChef: Cooking doesn’t get more repetitive than this</a> </p> </aside>  <p>Australia was the first country to turbocharge the MasterChef format. Over seven months of filming, 24 finalists are hothoused, living together, cooking together every day, and facing challenges up to and including replicating Michelin-starred dishes, that are purpose made to push them over the edge. There are times when it feels like you’re watching some sort of shamanic initiation, as keen amateurs are broken down to the point of dissolution of ego, only to be rebuilt as superhuman cooks. And it reaps rewards: it has consistently been one of the most-watched shows on Australian TV, and not just its winners but more and more of the finalists each season instantly become part of the Australian culinary establishment.</p> <p>This supersized format has been franchised all over the world, but the Australian version remains the best by a country mile thanks to two things: its geniality, and the food culture. Now, obviously, one should be under no illusions about television production, and MasterChef Australia is laden with casting and editing decisions that are as ruthlessly designed to press all your buttons without mercy as anything in any other reality TV show. Sometimes it’s absolutely transparent when the manipulations are taking place, and if contestants are missing their children, or have a noble job that they’ve left to be on the show, or have had a life-threatening illness, it’ll all be milked. </p>  <figure class="element element-image" data-media-id="fe5f8f02d5e9f742ebcae001c1de58382a0267d7"> <img src="https://media.guim.co.uk/fe5f8f02d5e9f742ebcae001c1de58382a0267d7/0\_163\_2700\_1620/1000.jpg" alt="Dynamite ... MasterChef Australia’s judges, Gary Mehigan, George Calombaris and Matt Preston" width="1000" height="600" class="gu-image" /> <figcaption> <span class="element-image\_\_caption">Dynamite ... MasterChef Australia’s judges, Gary Mehigan, George Calombaris and Matt Preston.</span> <span class="element-image\_\_credit">Photograph: Endemol Shine International</span> </figcaption> </figure>  <p>But somehow, underlying all this, there’s an unmistakeable Aussie camaraderie, a give-a-bloke-a-fair go attitude among the contestants and judges that pulls the rug out from the cynical viewer again and again. And actually, you don’t get <em>that</em> much back story. Most of the emotional tug is from the relationships and the growing personalities that you see developing in front of you on screen, episode by episode.</p> <p>The judges hold it all together perfectly. In fact, two of them are English-born, but both Gary Mehigan and Matt Preston are thoroughly naturalised. Mehigan and the third judge George Calombaris are both respected chefs, while Preston is one of Australia’s top restaurant critics. </p> <p>Purely in televisual terms, they are dynamite: the tiny, shaven headed Calombaris, the beefy and genial Mehigan and the archly avuncular man-mountain Preston play off one another constantly, and they are hands-on: Calombaris and Mehigan both running the kitchens during service challenges. None are soppy – for all the heartstring-tugging, there is never any illusion about the brutal nature of the catering business – but they are engaged, and Calombaris in particular creates great moments when his pugnacious exterior gives way to Yoda-like wisdom at moments of great crisis for contestants.</p> <p>But all of this would just be theatre if it weren’t for the substance of the show – the cooking. And that is where its Australianness comes into its own, because what we see is not the inward-looking, protectionist country of Malcolm Turnbull and Tony Abbott, but an immigrant nation, a part of the Asia-Pacific region, where people’s stories are told through food. </p> <p>This season’s stars include a burly, bear-like Indian-Singaporean prison officer, a fearsomely high-camp Vietnamese EDM DJ born in a refugee camp, an almost impossibly lovable young Mauritian boxer, an effervescent Italian nonna, and on it goes. And through them you can learn more about the multiple strands and fusions of cuisine that run through that whole corner of the globe. There are guest spots too, from Nigella Lawson, Heston Blumenthal, Gordon Ramsay, Prince Charles – and some of them are fun – but they pale in comparison to the real reasons for watching MasterChef Australia, the real reasons you will laugh, cry and salivate: the people, their lives, and their food.</p> <p><em>MasterChef Australia starts on W on 29 August at 7pm</em></p>\\
UK news & <p>Nazi memorabilia was offered for sale and attendees reportedly wore Nazi uniforms at a “living history” event held on National Trust property.</p> <p>For the past seven years Lacock Abbey in Wiltshire has hosted the Lacock at War event, organised by the West Wiltshire Military Vehicle Trust (WWMVT).</p> <p>The event paid tribute to British military history with displays of real and replica vehicles used by the army in both world wars. There was also the promise of musical entertainment and “a selection of traders selling military and vintage civilian clothing and collectibles”.</p> <p>However, people were shocked to see attendees dressed in Nazi uniforms and stalls where Nazi memorabilia was on sale.</p> <p>Attendees said that they saw Third Reich emblem pins and Nazi uniform patches featuring the eagle and swastika insignia on sale.</p> <p>They added that they saw a yellow Star of David patch, which Jews were made to wear under the Nazi regime. Organisers insisted that the armbands had been part of a historical display rather than for sale, and described the item as a “piece of history”.</p> <p>The National Trust said that the displays were “unacceptable”.</p> <p>Amanda, a descendant of eastern European Jews, who prefered to withhold her surname, told the Guardian: “There were people dressed as German soldiers, [with] swastikas and Third Reich emblems.”</p> <p>She added that those dressed in Nazi uniform “were parading themselves unapologetically. There seemed to be no efforts made to consider Jewish people in this.</p> <p>“I understand the interest in British world war two collectibles, but seeing somebody sell an SS beret like it belonged in a dressing-up box made me sick to my stomach.”</p> <p>Another attendee said: “I saw Nazi badges and some pins, lots of things bearing the swastika and the eagle, were mixed in with other countries’ badges at a stall, and there were price tags on all of them. A hat with the skull [<em>Totenkopf</em>] from the SS uniforms was also for sale.”</p> <p>She added: “Those items belong in museums, it’s wrong to profit from that stuff.”</p> <p>John Edward Wardle, the Lacock at War organiser at WWMVT, denied allegations that any of the event’s attendees wearing Nazi uniform.</p> <p>He said: “We are sorry that your complainant was offended and we have investigated and addressed your concerns. I have spoken to various people who attended the show, re-enactors, members of the public, and the show committee, and no one saw anybody dressed in Nazi uniform, or wearing swastika armbands.”</p> <p>He added: “if someone was dressed in Nazi or SS uniform then they would be asked to remove it and escorted from the show.”</p> <p>Two photographs on Facebook, taken this weekend at Lacock, show three men dressed in Nazi uniforms with eagle and swastika insignia. Over 10 photographs since found on social media show several attendees at Lacock dressed in Nazi uniform. In one image, posted on Instagram in 2017, the poster appears, in civilian clothes, next to two men dressed in uniforms bearing swastikas. In another image, posted in 2017 and captioned “\#luftwaffe flak crew”, a man is seen in a white uniform wearing the iron cross and an eagle and swastika insignia.</p> <p>Wardle insisted these images showed re-enactors in German uniforms, not Nazi uniforms, because “not all German soldiers were members of the Nazi party, those that were wore swastika armbands … not one of the re-enactors is wearing a swastika armband, therefore they are not Nazis”.</p> <p>In a statement the National Trust said: “We are aware of concerns raised over a ‘living history’ event at Lacock over the weekend and have contacted the organisers for an urgent explanation.</p> <p>“The event was organised by the Military Vehicle Trust and included uniforms and materials, which understandably caused distress and led to a complaint.</p> <p>“We will make it very clear to the MVT that these displays were insensitive, unacceptable and should not be repeated.”</p>\\
Opinion & <p>Even when life is a cruise, we are just a guard rail away from a rude awakening. For British passenger Kay Longstaff, the <a href="https://www.theguardian.com/world/2018/aug/19/british-woman-rescued-10-hours-adriatic-sea-cruise-ship-fall">fall from the back of the Norwegian Star cruise ship</a> has had an unexpectedly happy ending: rescued after 10 hours’ floating in the comparatively warm and balmy Adriatic sea. The circumstances of her initial plunge have yet to be established, and while theories have ascribed her survival to yoga and levels of subcutaneous body fat, most who go over the edge are not so lucky as to be plucked from the sea by the Croatian coastguard.</p> 
<p>The annals of those who have disappeared overboard – most thoroughly documented by the excellent <a href="http://www.cruisejunkie.com/">Cruise Junkie database</a> – show a rate of at least one recorded fall per month from cruise ships over recent decades. The incidence of mortality from the natural passing of an often older clientele is far higher – every stately ship, many carrying thousands of passengers, has a morgue. And the intermittent outbreaks of <a href="https://www.theguardian.com/uk-news/2016/may/09/hundreds-of-uk-cruise-passengers-fall-ill-in-possible-norovirus-outbreak">norovirus</a>, liable to sweep through a cruise ship like wildfire in a heatwave if unchecked, will pose a greater danger for many.</p> 
<aside class="element element-rich-link element--thumbnail"> 
 <p> <span>Related: </span><a href="https://www.theguardian.com/world/2015/nov/07/fisherman-lost-at-sea-436-days-book-extract">Lost at sea: the man who vanished for 14 months</a> </p> 
</aside> 
<p>But going over the edge captures the imagination: if vertigo, as some psychologists have posited, conceals a desire to jump, how many more feel it when the blue sea is below? And particularly, when that might be the more alluring waters visited by cruise ships, rather than a cross-Channel ferry. Of the cases where witnesses have been present, at least one reported (Australian) fatality was the result of someone deciding the waters looked inviting, after a drink too many.</p> 
<figure class="element element-atom"> 
 <gu-atom data-atom-id="d562a515-09dc-4b7a-a061-3f3fc8c6943e" data-atom-type="media"> 
  <div>
   <iframe frameborder="0" allowfullscreen="true" src="https://www.youtube.com/embed/JsnatrSiHDQ?showinfo=0\&amp;rel=0"></iframe>
  </div>
 </gu-atom> 
</figure> 
<p>Should the impact of striking the water from the equivalent of a 10-storey building or more <a href="https://www.theguardian.com/australia-news/2014/nov/25/search-called-off-for-man-who-fell-from-cruise-ship-off-sydney">not prove fatal</a>, the survivor may quickly find themselves alone: in the time it takes to relay a message and turn a moving ship around a considerable distance can grow. Technology has helped – CCTV footage has allowed ships to accurately calculate when and where a fall occurred, as happened in Longstaff’s case. But stories of miracle rescues, where shock, hypothermia, choppy waters or sharks are held at bay, will not necessarily be welcomed by cruise lines, for whom a search and disrupted schedules could spell significant financial costs.</p> 
<p>Mostly, though, motive, circumstance and truth are lost in the waves. Some have undoubtedly chosen to end their lives in this way. Some were simple accidents, with drink often playing a part, as the clientele and entertainment on offer aboard modern cruise ships have charted a somewhat different course from QE2 tradition. In some cases, foul play has been suspected – and relatives of the deceased have often been met with an unbearable lack of information from the cruise lines when a passenger has gone missing. Or indeed, <a href="https://www.theguardian.com/uk/2011/nov/11/rebecca-coriam-lost-at-sea">cruise ship workers</a>: often employed on a pittance and working long hours, hired from countries far away, from families with little hope of pressing their cases. Yet the relatives of affluent passengers or western employees have been equally frustrated in finding answers after disappearances.</p> 
<aside class="element element-rich-link element--thumbnail"> 
 <p> <span>Related: </span><a href="https://www.theguardian.com/travel/2017/jul/03/air-on-board-cruise-ships-is-twice-as-bad-as-at-piccadilly-circus">Air on board cruise ships 'is twice as bad as at Piccadilly Circus'</a> </p> 
</aside> 
<p>Technology, communications and safety have all improved on ships that have the scale of small towns, complete with shopping malls, restaurants and theatres – but falls overboard vividly demonstrate their limits. Despite recent laws passed in the US that offer greater legal and police protection to their cruising citizens, for many onboard the ultimate authority – and investigators should one disappear in international waters – resides in the ship’s flag state: likely the Bahamas or Panama, or any other with accommodating tax and regulatory compliance regimes. Longstaff’s welcome rescue was a small miracle, but should best serve as a stark reminder for those who cruise: ultimately, all are at sea.</p> 
<p>• Gwyn Topham is the author of Overboard - The Stories Cruise Lines Don’t Want Told and the Guardian and Observer’s transport correspondent</p>\\
Art and design & <p>Anish Kapoor’s art installation Descent Into Limbo is a big, black hole, too deep for viewers to be able to see the bottom. Or is it merely a black circular painting? You stand on the edge of the dark nothingness, fascinated and perhaps a little tempted to reach out a foot and test it. You could call it a meditation on the sublime. Or an accident waiting to happen.</p> <aside class="element element-rich-link element--thumbnail"> <p> <span>Related: </span><a href="https://www.theguardian.com/artanddesign/2018/aug/21/holed-up-man-falls-into-art-installation-of-8ft-hole-painted-black">Holed up: man falls into art installation of 8ft hole painted black</a> </p> </aside>  <p>Descent Into Limbo was first seen in 1992, but the inevitable accident has finally come. A visitor to Kapoor’s show at the Serralves Museum in Porto, Portugal, had to be taken to hospital after ending up inside what is in fact a 2.5-metre-deep hole.</p>  <figure class="element element-image" data-media-id="2f58e70a2d61738ae75252c0a5985c6624b40e56"> <img src="https://media.guim.co.uk/2f58e70a2d61738ae75252c0a5985c6624b40e56/0\_375\_5906\_3543/1000.jpg" alt="Anish Kapoor (left) with Descent Into Limbo." width="1000" height="600" class="gu-image" /> <figcaption> <span class="element-image\_\_caption">Anish Kapoor (left) with Descent Into Limbo.</span> <span class="element-image\_\_credit">Photograph: Corbis via Getty Images</span> </figcaption> </figure>  <p>Descent Into Limbo is one of those modern artworks that seem to menace life and limb. When Doris Salcedo cut a crack through the floor of Tate Modern’s Turbine Hall in London in 2007, visitors could not resist putting a foot or a leg inside. <a href="http://news.bbc.co.uk/1/hi/entertainment/7112960.stm">Quite a few injuries resulted</a> from playful antics or people tripping up. When Carsten Höller installed giant spiral slides at the same museum in 2006, thousands took the plunge and a <a href="https://www.theguardian.com/artanddesign/2015/mar/31/giant-slides-london-south-bank-carsten-holler-hayward">few got bruised</a>. Far more seriously, Christo and Jeanne-Claude closed their 1991 installation, the Umbrellas, when high winds caused one of its huge beach umbrellas to <a href="https://www.nytimes.com/1991/10/28/us/christo-umbrella-crushes-woman.html">crush a 33-year-old California woman to death</a> and injure several others.</p> <p>We want art to be dangerous, but not that dangerous. Or do we? It has been recognised since the Romantic age that some of the most powerful imaginative experiences derive from terror, horror and awe. The 18th-century thinker Edmund Burke called this dark aesthetic “the sublime”. He observed that real, even life-threatening, danger will always trump mere pictures of horror. Today, artists teeter over that precipice. Installation art can put us at real risk, as if we were climbing a mountain or exploring a cave.</p>  <figure class="element element-image" data-media-id="3adca67f531b2ea850e5469c92380c1fe838feb8"> <img src="https://media.guim.co.uk/3adca67f531b2ea850e5469c92380c1fe838feb8/0\_176\_5339\_3204/1000.jpg" alt="Christo and Jeanne-Claude’s the Umbrellas." width="1000" height="600" class="gu-image" /> <figcaption> <span class="element-image\_\_caption">Christo and Jeanne-Claude’s the Umbrellas.</span> <span class="element-image\_\_credit">Photograph: Tonya Evatt/Sygma via Getty Images</span> </figcaption> </figure>  <p>Richard Serra’s <a href="https://www.theguardian.com/artanddesign/2016/oct/13/richard-serra-review-nj-2-gagosian-britannia-street-london">steel sculptures</a> feel the most dangerous of all. They tower over you, surround you, menace you. The danger is not illusory. Serra’s rolled steel is dangerous, especially for workers who have to install it. In 1971, a contractor called Raymond Johnson was killed during the installation of a Serra sculpture. In 1988, two workers were <a href="https://www.nytimes.com/1988/10/27/nyregion/a-16-ton-sculpture-falls-injuring-2.html">seriously injured</a> while deinstalling one of his pieces.</p> <p>It would be grotesque to say these incidents heighten the sublime qualities of Serra’s art, but it is true that those relentless steel walls give off a sense of deathly force, like looking at a dangerous cliff.</p> <p>Luckily, the victim of Kapoor’s black hole is said to be doing well in hospital. Most injuries caused by art are, mercifully, mild. Yet something in us is drawn to the void, the precipice, the crack in the floor. If art couldn’t hurt us, it wouldn’t move us.</p>\\
\bottomrule
\end{tabular}}
\end{table}

\rowcolors{2}{white}{white}

\rowcolors{2}{gray!6}{white}

\begin{table}[!h]

\caption{\label{tab:data_collection}Table containing ECONOMY  News}
\centering
\resizebox{\linewidth}{!}{
\begin{tabular}[t]{ll}
\hiderowcolors
\toprule
sectionName & body\\
\midrule
\showrowcolors
Business & <p>The British economy is showing signs of having come through the worst of its recent slowdown triggered by bad weather, after the dominant services sector expanded more quickly than expected. </p> <p>Despite the better news from the latest business survey by IHS Markit and the Chartered Institute of Procurement and Supply (Cips), the key barometer for the largest sector in the economy revealed mounting fears over <a href="https://www.theguardian.com/politics/eu-referendum">Brexit</a>. </p> <p>The latest snapshot from the <a href="https://www.markiteconomics.com/Survey/PressRelease.mvc/a39d74857f5e45d4888efd8c81627950?s=1">Markit/Cips UK Services PMI </a>showed activity in the sector, which includes banks, restaurants and hotels, recovered to a three-month high in May. However, firms responding to the survey said business volumes had continued to rise at a relatively subdued rate, citing Brexit as a key reason for the slow growth. </p> <p>The Cips group director, Duncan Brock, said: “It felt as though the sector was losing its lifeblood this month as Brexit worries continued to claw away at confidence.”</p>  <figure class="element element-interactive interactive" data-interactive="https://interactive.guim.co.uk/embed/iframe-wrapper/0.1/boot.js" data-canonical-url="https://interactive.guim.co.uk/charts/embed/jun/2018-06-05T08:58:36/embed.html" data-alt="Services PMI"> <a href="https://interactive.guim.co.uk/charts/embed/jun/2018-06-05T08:58:36/embed.html">Services PMI</a> <figcaption>Services PMI</figcaption> </figure>  <p>The reading on the PMI, where anything above 50 indicates expansion, rose to 54.0 in May from 52.8 a month earlier, beating economists’ expectations for a modest rise to 53. </p> <p>Showing greater strength for the economy than previously feared, the increase will put pressure on the <a href="https://www.theguardian.com/business/bankofenglandgovernor">Bank of England</a> to raise interest rates, potentially from as early as August. The pound reflected that optimism, rising two-thirds of a cent to \$1.3275.</p>  <figure class="element element-embed" data-alt="Guardian business email sign-up">  <iframe src="https://www.theguardian.com/email/form/plaintone/3887" height="52px" data-form-title="Sign up for Business Today" data-form-description="Get the headlines and editors' picks every weekday morning." scrolling="no" seamless frameborder="0" class="iframed--overflow-hidden email-sub\_\_iframe js-email-sub\_\_iframe js-email-sub\_\_iframe--article" data-form-success-desc="Thanks for signing up"></iframe> <figcaption>Sign up to the daily Business Today email or follow Guardian Business on Twitter at @BusinessDesk</figcaption> </figure>  <p>Threadneedle Street <a href="https://www.theguardian.com/business/2018/may/10/bank-of-england-holds-uk-interest-rates-at-05">delayed an increase for borrowing costs last month</a> after the <a href="https://www.theguardian.com/business/2018/apr/04/construction-industry-frozen-by-beast-from-the-east">“beast from the east”</a> blew the economy off course, preferring to wait and see if businesses and consumers would get back to spending and selling goods as usual. </p> <p>Service providers in the latest PMI survey cited a catch-up from the snow-related disruption of the first three months of the year, alongside sustained growth of incoming new work. However, the latest increase in overall new work was still one of the weakest seen since the summer of 2016, straight after the EU referendum. </p> <p>Chris Williamson, the chief business economist at IHS Markit, said: “With the forward-looking indicators suggesting that the economy could relapse, a rate rise is by no means assured.”</p>\\
Business & <p>Theresa May will be forced to offer further politically difficult concessions to the EU to minimise damage to the economy caused by Brexit, said one of the UK’s leading economic thinktanks.</p> <p><a href="https://www.theguardian.com/business/2018/may/04/uk-growth-prediction-2018-scaled-back-by-thinktank-niesr">The National Institute for Economic and Social Research (NIESR)</a> said Britain was gripped by an epidemic of uncertainty about the terms of its EU departure, and warned that the government would have to pay a bigger financial contribution or accept higher migration to get the deal it wanted.</p> <p>May has already lost two cabinet ministers – David Davis and Boris Johnson – after announcing <a href="https://www.theguardian.com/politics/2018/jul/12/brexit-white-paper-seeks-free-movement-for-skilled-workers-and-students">plans for Brexit </a>that involved prioritising trade in goods over services, while aiming to limit free movement of people. </p> <p>But in its latest quarterly health check on the economy, the NIESR said May’s version of a soft Brexit was still not soft enough. The government was aiming for market access similar to that enjoyed by Switzerland but with a much tougher migration system. “In our view, the government will have to make significant concessions to the EU,” it said.</p> <p>The thinktank said, that following the pick-up in activity after the sluggish start to the year, the economy was on course to grow by 1.4\% in 2018 and by 1.75\% each year thereafter. </p> <p>It added, however, that even these modest growth rates relied on the UK continuing to have close to full access to the EU market for its goods and services.</p> <p>The more limited proposals for market access in the recently published white paper would lead to a loss of output amounting to £500 per person over time, compared to a soft Brexit.</p> <p>“The loss would be around £800 under a ‘no deal’ Brexit. These estimates do not include the likely impact on productivity which could, on some estimates, double the size of the losses,” said the thinktank. </p> <p>Jagjit Chadha, director of the NIESR, said: “In the UK, uncertainty about exit from the EU seems to be limiting the development of policies to promote more inclusive growth.” Brexit, he added, was a demanding agenda that was causing political stasis. “Things are not happening that should be happening.”</p> <p>Chadha said pre-Brexit Britain was “gripped by an unusual amount of uncertainty, you might call it an epidemic of uncertainty – we haven’t worked out how to do it”.</p> <p>The NIESR said the Bank of England would need to take account of the uncertainty when setting interest rates. While supporting <a href="https://www.theguardian.com/business/2017/nov/02/bank-of-england-raises-uk-interest-rates-brexit-inflation">a quarter-point increase in official borrowing costs this week</a>, the Bank should make it clear that it was prepared to reverse the decision, the institute added. <br></p> <p>“The UK is just eight months away from the March 2019 EU exit date and the range of outcomes remains as wide as ever. We are by no means sure that there will be a deal by then, and even if there is a deal it is not clear how policymakers, businesses and households will respond to the new arrangement. And, all along, the chance of a second referendum is rising.”</p>\\
Business & <figure class="element element-atom"> 
 <gu-atom data-atom-id="interactives/2017/03/brexit-snapshot" data-atom-type="interactive"> 
  <div>
   default
  </div>
 </gu-atom> 
</figure>\\
Opinion & <p>Much of the anger in today’s politics stems from an ongoing stagnation in living standards, coupled with the sight of a small minority becoming infinitely richer. <a href="https://inews.co.uk/news/education/uk-unequal-society-europe/">Britain is one of the most unequal societies in Europe</a>, where success increasingly depends on where you live and who your parents are. <a href="https://www.theguardian.com/inequality/2018/sep/05/qa-how-unequal-is-britain-and-are-the-poor-getting-poorer">Forty-four per cent of UK wealth is owned by 10\% of households</a>, while just 9\% of wealth is owned by the bottom half. This looks set to continue, with more than half of the net increase in wealth between 2010 and 2014 <a href="https://www.ippr.org/files/2017-10/cej-wealth-in-the-21st-century-october-2017.pdf">captured by the top decile of households</a>.</p> <p>Yet although wealth inequality is twice as great as income inequality, which has on some measures actually fallen slightly in recent years, it has strangely received far less attention.</p> <p>Of course, expectations of a better life cannot be met without wealth creation. And this needs investment and technological progress through innovation, which in turn requires rewards for the risk-taking entrepreneurs who make it happen. In other words, it is far easier to create a fairer society when the size of the pie is growing. That is why my party recently set out proposals to <a href="https://www.libdems.org.uk/taxingland-notinvestment">replace business rates with a far more investment-friendly land value tax</a>. But what particularly frustrates about wealth inequality in Britain is that it is not obviously the result of productive activity, and has widened primarily thanks to asset price inflation, especially that of property. And this inequality is now so deeply rooted that relying on the doubtful existence of a trickle down effect is no longer enough, if indeed it ever was.</p> <p>Britain’s wealth gap entrenches power and privilege both between and within generations, by generating additional income for asset owners while restricting opportunities for those with little or no wealth. Through gifts and inheritances, these inequalities are passed down to future generations. Those without wealth struggle to acquire property, fund education and training, deal with unexpected income shocks, or pursue their own business and creative ideas. Worse still, wealth inequality is likely to worsen in the coming years due to weak wage growth, rising household debt, automation and falling rates of home ownership and saving. Meanwhile, the money being passed down through inheritances to a lucky minority is projected to grow.</p> <p>We must demand a better future for Britain. If we want to create a society where everyone has the chance to succeed, we need to give people the skills and resources they need and create a more level playing field. That in turn requires tackling inequality through policies that promote a fairer distribution of wealth.</p> <aside class="element element-rich-link element--supporting"> <p> <span>Related: </span><a href="https://www.theguardian.com/uk-news/2018/sep/05/thinktank-calls-for-major-overhaul-of-britains-economy">Thinktank calls for major overhaul of Britain's economy</a> </p> </aside>  <p>That is why the Liberal Democrats have today put forward a <a href="https://d3n8a8pro7vhmx.cloudfront.net/libdems/pages/43665/attachments/original/1535625215/embedpdf\_Autumn18\_Giving\_Everyone\_a\_Stake\_.pdf?1535625215">bold set of proposals</a> to give everyone a stake in our economy, which will be debated by party members at our autumn conference in Brighton next week. They follow closely on the work of the thinktanks <a href="https://www.theguardian.com/money/2018/apr/17/one-in-three-uk-millennials-will-never-own-a-home-report">The Resolution Foundation</a> and the <a href="https://www.theguardian.com/uk-news/2018/sep/05/thinktank-calls-for-major-overhaul-of-britains-economy">Institute for Public Policy Research</a>, which have reached similar conclusions on the policies needed to address Britain’s deep economic divides. They recognise the need for the country to have a serious conversation about tax; something Boris Johnson, with his thoughtless calls for Trump-style tax cuts, is desperate to avoid.</p> <p>To start with, we would overhaul inheritance tax by taxing recipients – rather than givers – on all large gifts they receive, not just those they inherit. Each person would have a generous £250,000 tax-free lifetime allowance, above which income tax rates would apply. This would close a major loophole that allows people to hand down wealth entirely untaxed if the transfer is made more than seven years before death. It would mean that unearned gifts are taxed at the same rate as earned income from employment.</p> <p>This principle – taxing wealth and work in the same way – is at the core of our proposals. We would also tax capital gains at income tax rates, and abolish separate tax-free allowances for both capital gains and dividends. All income, be it from employment or wealth, would be eligible for the personal allowance, so that those who depend solely on income from assets would not be unfairly penalised. These changes would also remove the existing incentive for the well-off to shift employment income into other forms – such as capital gains or dividends – to minimise their tax bills.</p> <p>Another area in urgent need of reform is pension tax relief. Not only does it <a href="https://www.ftadviser.com/pensions/2018/01/23/pension-tax-relief-cost-to-hit-41bn/">cost the </a><a href="https://www.ftadviser.com/pensions/2018/01/23/pension-tax-relief-cost-to-hit-41bn/">government £41</a><a href="https://www.ftadviser.com/pensions/2018/01/23/pension-tax-relief-cost-to-hit-41bn/">bn annually</a>, but the existing system of relief is deeply regressive, with <a href="https://www.resolutionfoundation.org/app/uploads/2016/03/Pension-tax-relief.pdf">63\% of all relief going to the top 15\% of taxpayers.</a> We would rebalance it by introducing a flat rate of relief on pension contributions, encouraging lower earners to save for their retirement while cutting tax breaks for the richest. We would also limit the tax-free lump sum people can withdraw from their pension pots, restricting it for those with the largest pots while leaving over three-quarters of pensioners unaffected.</p> <p>These reforms would ensure that Britain’s wealth is taxed more equitably, thus promoting a fairer distribution of wealth. Our conservative estimate is that they would raise £15bn per year, which we would put to work by boosting spending on public services, launching an ambitious programme of lifelong learning and creating a citizens’ wealth fund to invest on behalf of the country.</p> <p>As is now painfully clear, cuts to the essential public services we all rely on have gone too far, resulting in falling levels of service and difficulties in recruiting staff. An ageing population risks worsening the situation in decades to come. We would therefore devote a share of the revenue raised from higher wealth taxation to better fund our essential public services.</p> <p>Secondly, Britain’s workers risk being left behind by our changing economy if they aren’t equipped with the skills and knowledge needed to flourish in the future workplace. We believe that an ambitious programme of lifelong learning – funded in part by wealth taxation – is the only solution, with people empowered to improve their skills throughout their careers through publicly funded accounts.</p> <p>Last but not least, the citizens’ wealth fund, established using a share of the revenue raised, would enable the country as a whole to benefit from the returns to investment typically only available to the wealthy by investing in stocks, bonds and other assets. To ensure its independence, the fund would be kept at arm’s length from government and run by professional fund managers, with robust accountability measures and a strong emphasis on sustainable and ethical investment.</p> <p>Combined, these reforms would go a long way in turning fairness and opportunity from mere catchwords into a genuine reflection of people’s everyday lives.</p> <p>• Vince Cable is leader of the Liberal Democrats</p>\\
Australia news & <p>The treasurer, Scott Morrison, will flag ongoing reforms in health, education and energy to boost national productivity in response to an inquiry by the Productivity Commission to be released on Tuesday.<br></p> <p>Morrison will use a speech to the Committee for Economic Development Australia to outline the main findings of the new work on productivity – which includes recommendations to adopt patient-centred healthcare, transform teaching capacity in the education system to help workers manage the profound transitions in the labour market, and the creation of more functional cities, which could boost gross domestic product by \$29bn.</p> <p>The report is the first instalment in a series of five-yearly reviews by the Productivity Commission examining contemporary methods to boost productivity – advice that will sit alongside the intergenerational report produced by the treasury every five years.</p> <aside class="element element-rich-link element--thumbnail"> <p> <span>Related: </span><a href="https://www.theguardian.com/business/commentisfree/2017/oct/23/regulation-is-back-after-years-of-neoliberal-neglect">Regulation is back, after years of neoliberal neglect | Josh Bornstein</a> </p> </aside>  <p>According to Morrison, the new report notes, for example, that 17.5\% of Australians have mental or behavioural problems and Australia’s suicide rate “is double the rate of the best performing countries.”</p> <p>Adopting more effective preventative strategies could boost labour force participation by up to 26\%.</p> <p>According to Morrison, the new work from the productivity commission suggests the economic benefits “from a health system reboot” could be worth up to \$200bn over the next two decades.</p> <p>On education, Morrison will say on Tuesday that the commission recommends “boosting salaries in subject areas where there are teacher shortages, to attract high calibre people and stop teachers from teaching out of field – like the 30\% of high school IT teachers who have never even studied the subject.”</p> <p>The report also recommends an overhaul of the vocational educational system, including “a graduated assessment system that measures the proficiency of VET students, rather than straight competency, and creating a fair credential embracing massive open online courses (MOOCs) that lower the cost of education and make learning more accessible.”</p> <aside class="element element-rich-link"> <p> <span>Related: </span><a href="https://www.theguardian.com/business/2017/oct/19/australias-unemployment-rate-falls-to-four-year-low-of-55">Australia’s unemployment rate falls to four-year low of 5.5\%</a> </p> </aside>  <p>Many economists believe productivity gains are harder to achieve in an environment where technological change is not delivering major boosts to economies around the world.</p> <p>Morrison will argue economic reforms of the 1980s and 1990s were crucial to boosting productivity, and ushered in a generation of prosperity, but he argues “there was a lot of low hanging fruit back then” and reform is harder to achieve today than it was in previous decades.</p> <p>Morrison will note in his speech that it is politically difficult to achieve major economic reform.</p> <p>The treasurer says “the price of a generation of Australians growing up without ever having known a recession is that reform comes more stubbornly and incrementally”.</p> <p>“We also need to understand that many Australians are now far more sceptical of change,” Morrison will say Tuesday. “Whenever governments mention the word reform or productivity, they get nervous.”</p> <p>“They’ve seen this movie before. Unlike last time when economic reform was a mystery to most, this time around Australians are more alive to the costs of change as well as the benefits.”</p> <p>The treasurer will argue “the economic and political bandwidth available for change is narrower than it once was, made more difficult by the binary way change is viewed and exploited”.</p> <p>He says economic reform is translated to voters as winners and losers, in a conflict prism.</p>\\
Opinion & <p>UK interest rates have risen from 0.5\% to 0.75\% because the <a href="https://www.theguardian.com/business/monetary-policy-committee" title="">monetary policy committee</a> (MPC) believes the labour market is at full employment and wage growth is set to explode. Pull the other one, it’s got bells on. I don’t buy it.</p> <p>The pound initially rose on the news and then fell back by more than a cent against the US dollar in a sign that the markets didn’t like what they heard and didn’t believe the MPC’s claims either. The Institute of Directors came out against the increase, as have <a href="https://www.independent.co.uk/news/business/news/interest-rate-hike-latest-bank-of-england-uk-economy-mark-carney-iod-bcc-a8474416.html" title="">many economists</a> who see no basis for a rise that lowers spending power. This in the same week when the <a href="https://www.marketwatch.com/story/the-bank-of-japan-just-set-the-table-for-more-yen-weakness-analyst-2018-07-31" title="">Bank of Japan said it would loosen monetary policy further</a>. Along with the European central bank it has negative rates and is still operating quantitative easing programmes.</p> <p>Astonishingly, the decision to raise rates was a unanimous 9-0, which was a big surprise given there was essentially no hard data supportive of a rise. One smart journalist at the press conference suggested that this may well be an example of “group think” – so it’s OK if we are all wrong together. The reality is there are stronger arguments to be made for a cut in rates.</p> <p>Personal insolvencies are up; there has been a slowing in the commercial property market, which is often suggestive of a slowing economy. The number of homes on the market is up but the number of buyers is not. The housing market is slowing, and raising the cost of a mortgage will slow it further. Brexit represents a major downside risk to the UK economy as does the possibility of trade wars.</p> <p>The latest data from the Office for National Statistics shows that <a href="https://www.ons.gov.uk/economy/grossdomesticproductgdp/bulletins/gdpmonthlyestimateuk/may2018" title="">GDP growth was 0.2\% in the three months to May 2018</a>, the same rate as in the first quarter from January to March. Growth in the first quarter was <a href="http://ec.europa.eu/eurostat/tgm/table.do?tab=table\&amp;init=1\&amp;language=en\&amp;pcode=teina011\&amp;plugin=1" title="">lower than in 26 out of 28 EU countries</a>; only Romania and Estonia were worse. The UK economy’s weakness doesn’t look temporary as the MPC claims. There is also no inflation to speak of, and no sign of a big pick-up coming. The consumer price index has come down from 3\% in January to 2.4\% in June 2018.</p> <p>The MPC’s main reason for raising rates is that it believes, wrongly in my view, that wage growth is set to skyrocket. It isn’t. According to the UK’s national pay statistic average weekly earnings, total pay growth <a href="https://tradingeconomics.com/united-kingdom/wage-growth" title="">averaged at 4.3\% from January 2001 to March 2008</a>, just before the onset of recession in April 2008. From that point through to the end of 2017 wage growth averaged 1.8\%. It picked up a little in the last year and now stands at 2.5\%, down from 3.1\% in December 2017. In its last 18 forecasts in a row, the MPC wrongly forecast that wage growth was going to return to pre-recession levels of around 4\% or so. The MPC has cried wolf in its past 18 forecasts, including this one, and I see no reason to believe it is right this time. Wage growth is slowing.</p> <p>My <a href="https://www.niesr.ac.uk/publications/underemployment-and-lack-wage-pressure-uk-notes-and-contributions-national-institute" title="">research with economist David Bell</a> suggests that the UK is a long way from full employment. It is underemployment, not unemployment that is impacting wage growth, and that has not returned to pre-recession levels. When full employment is reached, wage growth will start to strengthen towards 4\%. That isn’t going to happen any time soon. The MPC has just made a major error it will have to reverse fairly quickly as bad data flows in. I have deja vu of 2008.</p> <p>• David Blanchflower is a professor of economics at Dartmouth College, New Hampshire</p>\\
\bottomrule
\end{tabular}}
\end{table}

\rowcolors{2}{white}{white}

\rowcolors{2}{gray!6}{white}

\begin{table}[!h]

\caption{\label{tab:data_collection}Table containing POLITICS  News}
\centering
\resizebox{\linewidth}{!}{
\begin{tabular}[t]{ll}
\hiderowcolors
\toprule
sectionName & body\\
\midrule
\showrowcolors
Politics & <p>When Amber Rudd agreed to appear in front of political journalists at a Westminster lunch last week, it was an opportunity to demonstrate her leadership credentials.<br></p> 
<p>But by the time the event actually came around the fallout from the Guardian’s reporting on the Windrush scandal was in full flow and she was, as she put it, “just thinking about staying in the game”.</p> 
<aside class="element element-rich-link element--thumbnail"> 
 <p> <span>Related: </span><a href="https://www.theguardian.com/politics/2018/apr/29/amber-rudd-resigns-as-home-secretary-after-windrush-scandal">Amber Rudd resigns hours after Guardian publishes deportation targets letter</a> </p> 
</aside> 
<p>She had already had a hellish week involving several appearances at the dispatch box, a brutal session at the hands of the home affairs select committee and what felt like countless apologies.</p> 
<p>But while much of the anger directed at the home secretary over the preceding days had been a result of the mishandling of the Windrush generation of migrants, it was her confusion over the rather more arcane matter of targets for deporting illegal immigrants that eventually brought her down.</p> 
<p>The key moment in Rudd’s dramatic fall from grace was when she was summoned to explain herself – and her department – in front of the committee last Wednesday.</p> 
<p>Almost as an afterthought, committee chair Yvette Cooper asked about earlier evidence from the immigration officers’ union about targets for the number of people who should be deported from the UK.</p> 
<p>“We don’t have targets for removals,” Rudd replied, kicking off the series of claims and counterclaims, leaks and denials, that eventually led to her departure.</p> 
<aside class="element element-rich-link element--thumbnail"> 
 <p> <span>Related: </span><a href="https://www.theguardian.com/politics/2018/apr/29/amber-rudd-letter-to-pm-reveals-ambitious-but-deliverable-removals-target">Amber Rudd letter to PM reveals 'ambitious but deliverable' removals target</a> </p> 
</aside> 
<p>The next day it emerged that immigration officials in her own department had been given targets after all. She was summoned to the Commons to clarify. “I was not aware of them,” she insisted.</p> 
<p>By Friday, her claims were unravelling after a secret internal Home Office document boasting of the targets in 2017 was leaked to the Guardian. Damningly, Rudd had been copied in.</p> 
<p>More than eight hours after the Guardian approached the Home Office with details of the memo – as speculation swirled around Westminster about her future – she finally responded in a series of defiant late-night tweets.</p> 
<p>The home secretary insisted she had not seen the leaked memo, even though it had been sent to her office and that she wasn’t aware of the specific removal targets. “But I accept that I should have been and I’m sorry that I wasn’t.”</p> 
<figure class="element element-atom"> 
 <gu-atom data-atom-id="ae01395c-a71b-4d04-a691-bcc2754971e1" data-atom-type="media"> 
  <div>
   <iframe frameborder="0" allowfullscreen="true" src="https://www.youtube.com/embed/T8GkqD7slFE?showinfo=0\&amp;rel=0"></iframe>
  </div>
 </gu-atom> 
</figure> 
<p>She promised to make a fresh statement to the Commons on Monday about the affair. But if Rudd thought that was the end of the matter she was wrong.</p> 
<p>On Sunday afternoon <a href="https://www.theguardian.com/politics/2018/apr/29/amber-rudd-letter-to-pm-reveals-ambitious-but-deliverable-removals-target">the Guardian published yet another leak</a>, this time revealing that in a letter she sent to Theresa May in January 2017 she told the prime minister of her target to increase deportations by 10\%.</p> 
<p>The last line of defence: that she had “an ambition” to increase deportations rather than a “target” was broken and it was clear both to Rudd and No 10 that she could not cling on any longer.</p> 
<p>May had been relying on her home secretary to protect her own legacy at the Home Office – including defending her “hostile environment” strategy – as well as her reputation. But keeping her in office became more damaging than letting her go. With her human shield now gone, the prime minister is even more exposed.</p> 
<figure class="element element-atom"> 
 <gu-atom data-atom-id="af338525-c937-45f7-a4ea-88667478c569" data-atom-type="timeline"> 
  <div>
   <div class="atom-Timeline">
    <p><i>(April 23, 2018)</i>\&nbsp;<strong> </strong></p>
    <p></p>
    <p>Rudd delivered an unprecedented apology to parliament and acknowledged that her department had “lost sight of individuals” and become “too concerned with policy”.</p>
    <p></p>
    <p><i>(April 25, 2018)</i>\&nbsp;<strong> </strong></p>
    <p></p>
    <p>Rudd apologised for failing to grasp the scale of the problem. She told the home affairs select committee: “I bitterly, deeply regret that I didn’t see it as more than individual cases gone wrong that needed addressing. I didn’t see it as a systemic issue until very recently.”</p>
    <p></p>
    <p><i>(April 26, 2018)</i>\&nbsp;<strong> </strong></p>
    <p></p>
    <p>On Thursday morning, Rudd was forced to admit officials did have targets for removals, having previously denied their existence.</p>
    <p>“The immigration arm of the Home Office has been using local targets for internal performance management. These were not published targets against which performance was assessed, but if they were used inappropriately then I am clear that this will have to change."</p>
    <p></p>
    <p><i>(April 26, 2018)</i>\&nbsp;<strong> </strong></p>
    <p></p>
    <p>On Thursday afternoon, Rudd was forced to issue a hasty clarification after appearing to leave the door open to the UK staying in a customs union with the EU.</p>
    <p>“I should have been clearer – of course when we leave the EU we will be leaving the customs union."</p>
    <p></p>
    <p><i>(April 27, 2018)</i>\&nbsp;<strong> </strong></p>
    <p></p>
    <p>In a series of late-night tweets, Rudd apologised for not being aware of documents, leaked to the Guardian, which set out immigration removal targets.\&nbsp;<br></p>
    <p>‘I wasn’t aware of specific removal targets. I accept I should have been and I’m sorry that I wasn’t. I didn’t see the leaked document, although it was copied to my office as many documents are."</p>
    <p></p>
   </div>
  </div>
 </gu-atom> 
</figure> 
<p>Rudd, an ardent remainer, impressed pro-EU colleagues in debates during the referendum campaign – with one particularly withering putdown of Boris Johnson especially memorable – and was subsequently a powerful voice in cabinet for those wanting to stay in the customs union. </p> 
<p>But while her absence around the cabinet table, with its delicate balance of remainers and Brexiters, will be keenly felt, she is also returning to the backbenches knowing exactly where all the bodies are buried.</p> 
<p>She sat on the Cabinet’s EU withdrawal “war” committee, led the government’s work on post-departure immigration policy, and was a trusted confidante for those in government, including in No 10, who wanted the lightest touch Brexit possible. </p> 
<p>Rudd has been unswervingly loyal to the prime minister so far, but she might not feel so inclined to be on the backbenches. <br></p> 
<p>Less than two weeks ago, the biggest obstacle between Rudd and No 10 seemed to be the question of whether she would hold her seat at the next election. Even at the height of the targets fiasco, many thought she would survive, not least because she retained the support of Tory backbenchers, many of whom liked her personally.</p> 
<p>Her fortitude in turning up for a television debate during last year’s general election, which May herself had refused to do, just 48 hours after her father had died won her the respect of MPs from right across the Commons. </p> 
<p>Her personal history is that of a woman who would always have intended to reach the top. Her father was a stockbroker, her brother Roland, a leading remain backer and financial PR wizard, and she was married for five years to AA Gill, the late restaurant critic, with whom she had two children. In his columns she was, famously, the Silver Spoon.<br></p> 
<p>Her backstory is typical of many women on the A-list of candidates David Cameron set up to symbolise his modernisation programme: private school, university, a brief career in banking and finance, and a reputation for networking. Richard Curtis, who recruited her to supply extras for Four Weddings and a Funeral, was impressed by how many dukes she knew.</p> 
<p>Rudd’s glamour and energy won her the marginal seat of Hastings in 2010 and 2015. After May’s arrival in Downing Street, she was one of the ministers who prospered. She proved herself suited to a project dear to May’s heart: building a female leadership team. Rudd went to the Home Office. With Johnson in the Foreign Office, May had set up her two most obvious rivals in jobs that nowadays rarely produce prime ministers.</p> 
<p>Loyalty is one of Rudd’s defining political characteristics. But she does not fawn, and she is trusted by the people who can influence her future. In the anguished team that ran the 2017 election campaign, Rudd blossomed, warm and authentic against May’s clumsy reserve. And on 9 June, it was Rudd who emerged stronger from the wreckage. </p> 
<aside class="element element-rich-link element--thumbnail"> 
 <p> <span>Related: </span><a href="https://www.theguardian.com/commentisfree/2018/apr/29/amber-rud-theresa-may-crisis-windrush">Amber Rudd has gone, but now Theresa May faces a new crisis | Isabel Hardman</a> </p> 
</aside> 
<p>Such was the bond of confidence built between Rudd and May that the idea of Rudd moving to the Treasury began to circulate. In a piece of historic symbolism, the second female prime minister would create the first female chancellor.</p> 
<p>Instead, Rudd began her second year as home secretary, a job where success is defined as stopping things from happening. For Rudd, it is even harder. No one who knows her thinks she believes in May’s cherished ambition of cutting net migration to below 100,000.</p> 
<p>Until a fortnight ago, it was generally understood Rudd was a natural liberal oppressed by the demands of the security state and the tendency of the prime minister to begin every conversation with the remark: “When I was home secretary …”</p> 
<p>Well-sourced stories emerged that she was trying to adopt a Blairite “tough on crime, tough on the causes of crime” strategy for violent crime. Yet when the strategy launched, she admitted to not having read her own research on the impact of police cuts. Now, as well as her job being taken away, it is difficult to see how her status as a leading liberal can survive the facts of Windrush.</p>\\
Politics & <p>Who are the contenders to be the next home secretary? Downing Street has said a new appointment will not be announced until later on Monday but these are the ministers and MPs who are in contention to replace Amber Rudd.</p> <aside class="element element-rich-link element--thumbnail"> <p> <span>Related: </span><a href="https://www.theguardian.com/politics/2018/apr/29/amber-rudd-resigns-as-home-secretary-after-windrush-scandal">Amber Rudd resigns hours after Guardian publishes deportation targets letter</a> </p> </aside>  <h2>Sajid Javid, communities and local government secretary</h2> <p>Javid is the favourite for the job and is already being tipped by Tory MPs. The timing could not be better from his perspective; he was on the front of the Sunday Telegraph talking about his personal anger over the treatment of the Windrush migrants, saying: “I thought that could be my mum ... my dad ... my uncle ... it could be me.” Having backed remain, Javid would keep the balance in the top jobs, though he has made it clear since the referendum he is sceptical of softer Brexit options such as remaining in the customs union.</p> <h2>Michael Gove, environment secretary</h2> <p>Gove has courted the spotlight since his return to the cabinet table, with a series of eye-catching environmental policies, including a well-publicised war on plastics. However, the key test will be whether his personal relationship with May has mended, since the two clashed during the coalition government in a furious briefing war over her performance at the Home Office when she was home secretary. Were he to get the job, it is likely to re-ignite tensions over the anti-immigration messages of the Vote Leave campaign.</p> <h2>Jeremy Hunt, health secretary</h2> <p>Hunt made it clear at the last cabinet reshuffle that he did not want to leave the Department of Health. When May attempted to move him to Business, Energy and Industrial Strategy, he convinced her to keep him in post. However, the Home Office is certainly a bigger draw and it could be the promotion to tempt him, perhaps even paving the way for an eventual run at the top job.</p> <h2>David Lidington, minister for the Cabinet Office</h2> <aside class="element element-rich-link element--thumbnail"> <p> <span>Related: </span><a href="https://www.theguardian.com/commentisfree/2018/apr/29/amber-rud-theresa-may-crisis-windrush">Amber Rudd has gone, but now Theresa May faces a new crisis | Isabel Hardman</a> </p> </aside>  <p>Lidington is only just settling in at the Cabinet Office, replacing May’s old friend Damian Green who was sacked for making misleading statements about pornography found on his office computer. In Rudd, May has lost another ally that she trusted and confidants are now getting fewer by the day in the cabinet, so Lidington could get the role if May believes the priority should be a safe pair of hands. James Brokenshire, the former Home Office minister and Northern Ireland secretary who left the cabinet due to ill health, also fits that bill.</p> <h2>Karen Bradley, Northern Ireland secretary</h2> <p>Bradley worked under May at the Home Office, giving her direct experience of the complexities of the department. The former culture secretary was moved to Northern Ireland in the latest reshuffle, currently one of the most difficult jobs in government due to Brexit tensions over the Irish border and the collapse of devolved government in Belfast. Appointing Bradley would also mean keeping the 50/50 balance of men and women in the four great offices of state.</p> <h2>Wild cards: Dominic Grieve or Nicky Morgan</h2> <p>They are unlikely appointments, but both are capable and experienced former cabinet ministers. To appoint either of them would strip the pro-remain rebels in parliament of a key voice. Both, however, may find the compromise too much, and there is mutual animosity between May and Morgan after a fallout last year over a pair of leather trousers worn by May.</p>\\
Technology & <p>Vladimir Putin was not in attendance, but his loyal lieutenants were. On 14 July last year, the Russian prime minister, <a href="https://www.theguardian.com/world/dmitry-medvedev">Dmitry Medvedev</a>, and several members of his cabinet convened in an office building on the outskirts of Moscow. On to the stage stepped a boyish-looking psychologist, <a href="http://www.michalkosinski.com/">Michal Kosinski</a>, who had been flown from the city centre by helicopter to share his research. “There was Lavrov, in the first row,” he recalls several months later, referring to Russia’s foreign minister. “You know, a guy who starts wars and takes over countries.” Kosinski, a 36-year-old assistant <a href="https://www.gsb.stanford.edu/faculty-research/faculty/michal-kosinski">professor of organisational behaviour at Stanford University</a>, was flattered that the Russian cabinet would gather to listen to him talk. “Those guys strike me as one of the most competent and well-informed groups,” he tells me. “They did their homework. They read my stuff.”</p> <p>Kosinski’s “stuff” includes groundbreaking research into technology, mass persuasion and artificial intelligence (AI) – research that inspired the creation of the political consultancy Cambridge Analytica. Five years ago, while a graduate student at Cambridge University, he showed how even benign activity on Facebook could reveal personality traits – a discovery that was later exploited by the data-analytics firm that helped put Donald Trump in the White House.</p> <p>That would be enough to make Kosinski interesting to the Russian cabinet. But his audience would also have been intrigued by his work on the use of AI to detect psychological traits. Weeks after his trip to Moscow, Kosinski published a <a href="https://www.theguardian.com/technology/2017/sep/07/new-artificial-intelligence-can-tell-whether-youre-gay-or-straight-from-a-photograph">controversial paper</a> in which he showed how face-analysing algorithms could distinguish between photographs of gay and straight people. As well as sexuality, he believes this technology could be used to detect emotions, IQ and even a predisposition to commit certain crimes. Kosinski has also used algorithms to distinguish between the faces of Republicans and Democrats, in an unpublished experiment he says was successful – although he admits the results can change “depending on whether I include beards or not”.</p>  <aside class="element element-pullquote element--supporting"> <blockquote> <p>Progress makes people uncomfortable. When the first monkeys walked, other monkeys probably said, 'This is outrageous!'</p> </blockquote> </aside>  <p>How did this 36-year-old academic, who has yet to write a book, attract the attention of the Russian cabinet? Over our several meetings in California and London, Kosinski styles himself as a taboo-busting thinker, someone who is prepared to delve into difficult territory concerning artificial intelligence and surveillance that other academics won’t. “I can be upset about us losing privacy,” he says. “But it won’t change the fact that we already lost our privacy, and there’s no going back without destroying this civilisation.”</p> <p>The aim of his research, Kosinski says, is to highlight the dangers. Yet he is strikingly enthusiastic about some of the technologies he claims to be warning us about, talking excitedly about cameras that could detect people who are “lost, anxious, trafficked or potentially dangerous. You could imagine having those diagnostic tools monitoring public spaces for potential threats to themselves or to others,” he tells me. “There are different privacy issues with each of those approaches, but it can literally save lives.”</p> <p>“Progress always makes people uncomfortable,” Kosinski adds. “Always has. Probably, when the first monkeys stopped hanging from the trees and started walking on the savannah, the monkeys in the trees were like, ‘This is outrageous! It makes us uncomfortable.’ It’s the same with any new technology.”</p> <p>***</p> <p>Kosinski has analysed thousands of people’s faces, but never run his own image through his personality-detecting models, so we cannot know what traits are indicated by his pale-grey eyes or the dimple in his chin. I ask him to describe his own personality. He says he’s a conscientious, extroverted and probably emotional person with an IQ that is “perhaps slightly above average.” He adds: “And I’m disagreeable.” What made him that way? “If you trust personality science, it seems that, to a large extent, you’re born this way.”</p> <p>His friends, on the other hand, describe Kosinski as a brilliant, provocative and irrepressible data scientist who has an insatiable (some say naive) desire to push the boundaries of his research. “Michal is like a small boy with a hammer,” one of his academic friends tells me. “Suddenly everything looks like a nail.”</p> <p>Born in 1982 in Warsaw, Kosinski inherited his aptitude for coding from his parents, both of whom trained as software engineers. Kosinski and his brother and sister had “a computer at home, potentially much earlier than western people of the same age”. By the late 1990s, as Poland’s post-Soviet economy was opening up, Kosinski was hiring his schoolmates to work for his own IT company. This business helped fund him through university, and in 2008 he enrolled in a PhD programme at Cambridge, where he was affiliated with the <a href="https://www.psychometrics.cam.ac.uk/">Psychometrics Centre</a>, a facility specialising in measuring psychological traits.</p> <p>It was around that time that he met <a href="https://www.psychometrics.cam.ac.uk/about-us/directory/david-stillwell">David Stillwell</a>, another graduate student, who had built a personality quiz and shared it with friends on Facebook. The app quickly went viral, as hundreds and then thousands of people took the survey to discover their scores according to the “Big Five” metrics: openness, conscientiousness, extraversion, agreeableness and neuroticism. When users completed the <a href="https://www.psychometrics.cam.ac.uk/productsservices/mypersonality">myPersonality</a> tests, some of which also measured IQ and wellbeing, they were given an option to donate their results to academic research.</p> <p>Kosinski came on board, using his digital skills to clean, anonymise and sort the data, and then make it available to other academics. By 2012, more than 6 million people had taken the tests – with about 40\% donating their data, creating the largest dataset of its kind.</p>  <figure class="element element-image element--thumbnail" data-media-id="db43ba56eedf63e78d125aebecaad333af19c84d"> <img src="https://media.guim.co.uk/db43ba56eedf63e78d125aebecaad333af19c84d/1431\_198\_959\_1057/907.jpg" alt="From Cesare Lombroso’s 19th-century criminal taxonomy: headshot of a “habitual thief"" width="907" height="1000" class="gu-image" /> <figcaption> <span class="element-image\_\_caption">From Cesare Lombroso’s criminal taxonomy: a ‘habitual thief’…</span> </figcaption> </figure>  <figure class="element element-image element--thumbnail" data-media-id="db43ba56eedf63e78d125aebecaad333af19c84d"> <img src="https://media.guim.co.uk/db43ba56eedf63e78d125aebecaad333af19c84d/209\_204\_962\_1045/921.jpg" alt="From Cesare Lombroso’s 19th-century criminal taxonomy: head shot of a murderer" width="921" height="1000" class="gu-image" /> <figcaption> <span class="element-image\_\_caption">…and a murderer. Photographs: Alamy</span> </figcaption> </figure>  <p>In May, New Scientist magazine revealed that the <a href="https://www.newscientist.com/article/2168713-huge-new-facebook-data-leak-exposed-intimate-details-of-3m-users/">dataset’s username and password had been accidentally left on GitHub</a>, a commonly used code-sharing website. For four years, anyone – not just authorised researchers – could have accessed the data. Before the magazine’s investigation, Kosinski had admitted to me that there were risks to their liberal approach. “We anonymised the data, and we made scientists sign a guarantee that they will not use it for any commercial reasons,” he had said. “But you just can’t really guarantee that this will not happen.” Much of the Facebook data, he added, was “de-anonymisable”. In the wake of the New Scientist story, Stillwell closed down the myPersonality project. Kosinski sent me a link to the announcement, complaining: “Twitter warriors and sensation-seeking writers made David shut down the myPersonality project.”</p> <p>During the time the myPersonalitydata was accessible, about 280 researchers used it to publish more than 100 academic papers. The most talked-about was <a href="http://www.pnas.org/content/110/15/5802">a 2013 study</a> co-authored by Kosinski, Stillwell and another researcher, that explored the relationship between Facebook “Likes” and the psychological and demographic traits of 58,000 people. Some of the results were intuitive: the best predictors of introversion, for example, were Likes for pages such as “Video Games” and “Voltaire”. Other findings were more perplexing: among the best predictors of high IQ were Likes on the Facebook pages for “Thunderstorms” and “Morgan Freeman’s Voice”. People who Liked pages for “iPod” and “Gorillaz” were likely to be dissatisfied with life.</p> <p>If an algorithm was fed with sufficient data about Facebook Likes, <a href="http://www.pnas.org/content/112/4/1036.full">Kosinski and his colleagues found</a>, it could make more accurate personality-based predictions than assessments made by real-life friends. <a href="http://www.pnas.org/content/early/2017/11/07/1710966114">In other research</a>, Kosinski and others showed how Facebook data could be turned into what they described as “an effective approach to digital mass persuasion”.</p> <p>Their research came to the attention of the <a href="https://www.theguardian.com/uk-news/2018/mar/29/cambridge-analytica-predecessor-had-access-to-secret-mod-information">SCL Group</a>, the parent company of Cambridge Analytica. In 2014, SCL tried to enlist Stillwell and Kosinski, offering to buy the myPersonality data and their predictive models. When negotiations broke down, they relied on the help of another academic in Cambridge’s psychology department – <a href="https://www.neuroscience.cam.ac.uk/directory/profile.php?Ak823">Aleksandr Kogan</a>, an assistant professor. Using his own Facebook personality quiz, and paying users (with SCL money) to take the tests, Kogan collected data on 320,000 Americans. Exploiting a loophole that allowed developers to harvest data belonging to the friends of Facebook app users (without their knowledge or consent), Kogan was able to hoover up additional data on as many as 87 million people.</p>  <figure class="element element-image" data-media-id="a4bdbc721484d1ee3db785b9b469910ba6612112"> <img src="https://media.guim.co.uk/a4bdbc721484d1ee3db785b9b469910ba6612112/299\_10\_4744\_2846/1000.jpg" alt="Headshot of whistleblower Christopher Wylie of Cambridge Analytica" width="1000" height="600" class="gu-image" /> <figcaption> <span class="element-image\_\_caption">Cambridge Analytica whistleblower Christopher Wylie who says the company tried to replicate Kosinski’s work for ‘psychological warfare’. Photograph: Getty Images</span> </figcaption> </figure>  <p><a href="https://www.theguardian.com/uk-news/2018/apr/07/christopher-wylie-why-i-broke-the-facebook-data-story-and-what-should-happen-now">Christopher Wylie, the whistleblower who lifted the lid on Cambridge Analytica’s operations </a>earlier this year, has described how the company set out to “replicate” the work done by Kosinski and his colleagues, and to turn it into an instrument of “psychological warfare”. “This is not my fault,” <a href="https://motherboard.vice.com/en\_us/article/mg9vvn/how-our-likes-helped-trump-win">Kosinski told reporters</a> from the Swiss publication Das Magazin, which was the first to make the connection between his work and Cambridge Analytica. “I did not build the bomb. I only showed that it exists.”</p> <p>Cambridge Analytica always denied using Facebook-based psychographic targeting during the Trump campaign, but the scandal over its data harvesting <a href="https://www.theguardian.com/uk-news/2018/may/02/cambridge-analytica-closing-down-after-facebook-row-reports-say">forced the company to close</a>. The saga also proved highly damaging to Facebook, whose headquarters are less than four miles from Kosinski’s base at Stanford’s Business School in Silicon Valley. The first time I enter his office, I ask him about a painting beside his computer, depicting a protester armed with a Facebook logo in a holster instead of a gun. “People think I’m anti-Facebook,” Kosinski says. “But I think that, generally, it is just a wonderful technology”.</p> <p>Still, he is disappointed in the Facebook CEO, Mark Zuckerberg, who, when he <a href="https://www.theguardian.com/technology/2018/apr/11/mark-zuckerbergs-testimony-to-congress-the-key-moments">testified before US Congress in April</a>, said he was trying to find out “whether there was something bad going on at Cambridge University”. Facebook, Kosinski says, was well aware of his research. He shows me emails he had with employees in 2011, in which they disclosed they were “using analysis of linguistic data to infer personality traits”. In 2012, the same employees filed a patent, showing how personality characteristics could be gleaned from Facebook messages and status updates.</p> <p>Kosinski seems unperturbed by the furore over Cambridge Analytica, which he feels has unfairly maligned psychometric micro-targeting in politics. “There are negative aspects to it, but overall this is a great technology and great for democracy,” he says. “If you can target political messages to fit people’s interests, dreams, personality, you make those messages more relevant, which makes voters more engaged – and more engaged voters are great for democracy.” But you can also, I say, use those same techniques to discourage your opponent’s voters from turning out, which is bad for democracy. “Then every politician in the US is doing this,” Kosinski replies, with a shrug. “Whenever you target the voters of your opponent, this is a voter-suppression activity.”</p> <p>Kosinski’s wider complaint about the Cambridge Analytica fallout, he says, is that it has created “an illusion” that governments can protect data and shore up their citizens’ privacy. “It is a lost war,” he says. “We should focus on organising our society in such a way as to make sure that the post-privacy era is a habitable and nice place to live.”</p> <p>***</p> <p>Kosinski says he never set out to prove that AI could predict a person’s sexuality. He describes it as a chance discovery, something he “stumbled upon”. The lightbulb moment came as he was sifting through Facebook profiles for another project and started to notice what he thought were patterns in people’s faces. “It suddenly struck me,” he says, “introverts and extroverts have completely different faces. I was like, ‘Wow, maybe there’s something there.’”</p> <p>Physiognomy, the practice of determining a person’s character from their face, has a history that stretches back to ancient Greece. But its heyday came in the 19th century, when the Italian anthropologist Cesare Lombroso published his famous taxonomy, which declared that “nearly all criminals” have “jug ears, thick hair, thin beards, pronounced sinuses, protruding chins, and broad cheekbones”. The analysis was rooted in a deeply racist school of thought that held that criminals resembled “savages and apes”, although Lombroso presented his findings with the precision of a forensic scientist. Thieves were notable for their “small wandering eyes”, rapists their “swollen lips and eyelids”, while murderers had a nose that was “often hawklike and always large”.</p> <p>Lombroso’s remains are still on display <a href="http://museolombroso.unito.it/index.php/en/">in a museum in Turin</a>, besides the skulls of the hundreds of criminals he spent decades examining. Where Lombroso used calipers and craniographs, Kosinski has been using neural networks to find patterns in photos scraped from the internet.</p>  <aside class="element element-pullquote element--supporting"> <blockquote> <p>Kosinski received emails from people confused about their sexuality who hoped he'd run their photo through his algorithm</p> </blockquote> </aside>  <p>Kosinski’s research dismisses physiognomy as “a mix of superstition and racism disguised as science” – but then argues it created a taboo around “studying or even discussing the links between facial features and character”. There is growing evidence, he insists, that links between faces and psychology exist, even if they are invisible to the human eye; now, with advances in machine learning, such links can be perceived. “We didn’t have algorithms 50 years ago that could spot patterns,” he says. “We only had human judges.”</p> <p>In a <a href="https://osf.io/zn79k/">paper published last year</a>, Kosinski and a Stanford computer scientist, Yilun Wang, reported that a machine-learning system was able to distinguish between photos of gay and straight people with a high degree of accuracy. They used 35,326 photographs from dating websites and what Kosinski describes as “off-the-shelf” facial-recognition software.</p> <p>Presented with two pictures – one of a gay person, the other straight – the algorithm was trained to distinguish the two in 81\% of cases involving images of men and 74\% of photographs of women. Human judges, by contrast, were able to identify the straight and gay people in 61\% and 54\% of cases, respectively. When the algorithm was shown five facial images per person in the pair, its accuracy increased to 91\% for men, 83\% for women. “I was just shocked to discover that it is so easy for an algorithm to distinguish between gay and straight people,” Kosinski tells me. “I didn’t see why that would be possible.”</p>  <figure class="element element-image" data-media-id="4e33c14b2d8eba1960cb68754fce99ef4692d2d1"> <img src="https://media.guim.co.uk/4e33c14b2d8eba1960cb68754fce99ef4692d2d1/0\_1190\_6192\_7066/876.jpg" alt="Psychologist Michal Kosinski" width="876" height="1000" class="gu-image" /> <figcaption> <span class="element-image\_\_caption">‘I did not build the bomb. I only showed it exists.’</span> <span class="element-image\_\_credit">Photograph: Jason Henry for the Guardian</span> </figcaption> </figure>  <p>Neither did many other people, and there was an immediate backlash when the research – dubbed “AI gaydar” – was previewed in the Economist magazine. Two of America’s most prominent LGBTQ organisations demanded that Stanford distance itself from what they called its professor’s “dangerous and flawed research”. Kosinski received a deluge of emails, many from people who told him they were confused about their sexuality and hoped he would run their photo through his algorithm. (He declined.) There was also anger that Kosinski had conducted research on a technology that could be used to persecute gay people in countries such as Iran and Saudi Arabia, where homosexuality is punishable by death.</p> <p>Kosinski says his critics missed the point. “This is the inherent paradox of warning people against potentially dangerous technology,” he says. “I stumbled upon those results, and I was actually close to putting them in a drawer and not publishing – because I had a very good life without this paper being out. But then a colleague asked me if I would be able to look myself in the mirror if, one day, a company or a government deployed a similar technique to hurt people.” It would, he says, have been “morally wrong” to bury his findings.</p> <p>One vocal critic of that defence is the Princeton professor <a href="https://psych.princeton.edu/person/alexander-todorov">Alexander Todorov</a>, who has conducted some of the most widely cited research into faces and psychology. He argues that Kosinski’s methods are deeply flawed: the patterns picked up by algorithms comparing thousands of photographs may have little to do with facial characteristics. In <a href="https://medium.com/@blaisea/do-algorithms-reveal-sexual-orientation-or-just-expose-our-stereotypes-d998fafdf477">a mocking critique</a> posted online, Todorov and two AI researchers at Google argued that Kosinski’s algorithm could have been responding to patterns in people’s makeup, beards or glasses, even the angle they held the camera at. Self-posted photos on dating websites, Todorov points out, project a number of non-facial clues.</p> <p>Kosinski acknowledges that his machine learning system detects unrelated signals, but is adamant the software also distinguishes between facial structures. His findings are consistent with the prenatal hormone theory of sexual orientation, he says, which argues that the levels of androgens foetuses are exposed to in the womb help determine whether people are straight or gay. The same androgens, Kosinski argues, could also result in “gender-atypical facial morphology”. “Thus,” he writes in his paper, “gay men are predicted to have smaller jaws and chins, slimmer eyebrows, longer noses and larger foreheads... The opposite should be true for lesbians.”</p>  <aside class="element element-pullquote element--supporting"> <blockquote> <p>If you basically accept we’re just computers, then computers are not guilty of crime</p> </blockquote> </aside>  <p>This is where Kosinski’s work strays into biological determinism. While he does not deny the influence of social and environmental factors on our personalities, he plays them down. At times, what he says seems eerily reminiscent of Lombroso, who was critical of the idea that criminals had “free will”: they should be pitied rather than punished, the Italian argued, because – like monkeys, cats and cuckoos – they were “programmed to do harm”.</p> <p>“I don’t believe in guilt, because I don’t believe in free will,” Kosinski tells me, explaining that a person’s thoughts and behaviour “are fully biological, because they originate in the biological computer that you have in your head”. On another occasion he tells me, “If you basically accept that we’re just computers, then computers are not guilty of crime. Computers can malfunction. But then you shouldn’t blame them for it.” The professor adds: “Very much like: you don’t, generally, blame dogs for misbehaving.”</p> <p>Todorov believes Kosinski’s research is “incredibly ethically questionable”, as it could lend a veneer of credibility to governments that might want to use such technologies. He points to a <a href="https://arxiv.org/pdf/1611.04135v1.pdf">paper that appeared online two years ago</a>, in which Chinese AI researchers claimed they had trained a face-recognition algorithm to predict – with 90\% accuracy – whether someone was a convicted criminal. The research, which used Chinese government identity photographs of hundreds of male criminals, was not peer-reviewed, and was <a href="https://medium.com/@blaisea/physiognomys-new-clothes-f2d4b59fdd6a">torn to shreds by Todorov</a>, who warned that “developments in artificial intelligence and machine learning have enabled scientific racism to enter a new era”.</p> <p>Kosinski has a different take. “The fact that the results were completely invalid and unfounded, doesn’t mean that what they propose is also wrong,” he says. “I can’t see why you would not be able to predict the propensity to commit a crime from someone’s face. We know, for instance, that testosterone levels are linked to the propensity to commit crime, and they’re also linked with facial features – and this is just one link. There are thousands or millions of others that we are unaware of, that computers could very easily detect.”</p> <p>Would he ever undertake similar research? Kosinski hesitates, saying that “crime” is an overly blunt label. It would be more sensible, he says, to “look at whether we can detect traits or predispositions that are potentially dangerous to an individual or society – like aggressive behaviour”. He adds: “I think someone has to do it… Because if this is a risky technology, then governments and corporations are clearly already using it.”</p> <p>***</p> <p>But when I press Kosinski for examples of how psychology-detecting AI is being used by governments, he repeatedly falls back on an obscure Israeli startup, <a href="https://www.faception.com/">Faception</a>. The company provides software that scans passports, visas and social-media profiles, before spitting out scores that categorise people according to several personality types. On its website, Faception lists eight such <a href="https://www.faception.com/our-technology">classifiers</a>, including “White-Collar Offender”, “High IQ”, “Paedophile” and “Terrorist”. Kosinski describes the company as “dodgy” – a case study in why researchers who care about privacy should alert the public to the risks of AI. “Check what Faception are doing and what clients they have,” he tells me during an animated debate over the ethics of his research.</p> <p>I call Faception’s chief executive, Shai Gilboa, who used to work in Israeli military intelligence. He tells me the company has contracts working on “homeland security and public safety” in Asia, the Middle East and Europe. To my surprise, he then tells me about a research collaboration he conducted two years ago. “When you look in the academia market you’re looking for the best researchers, who have very good databases and vast experience,” he says. “So this is the reason we approached Professor Kosinski.”</p> <p>But when I put this connection to Kosinski, he plays it down: he claims to have met Faception to discuss the ethics of facial-recognition technologies. “They came [to Stanford] because they realised what they are doing has potentially huge negative implications, and huge risks.” Later, he concedes there was more to it. He met them “maybe three times” in Silicon Valley, and was offered equity in the company in exchange for becoming an adviser (he says he declined).</p> <p>Kosinski denies having collaborated on research, but admits Faception gave him access to its facial-recognition software. He experimented with Facebook photos in the myPersonality dataset, he says, to determine how effective the Faception software was at detecting personality traits. He then suggested Gilboa talk to Stillwell about purchasing the myPersonality data. (Stillwell, Kosinski says, declined.)</p> <p>He bristles at my suggestion that these conversations seem ethically dubious. “I will do a lot of this,” he says. “A lot of startup people come here and they don’t offer you any money, but they say, ‘Look, we have this project, can you advise us?’” Turning down such a request would have made him “an arrogant prick”.</p> <p>He gives a similar explanation for his trip to Moscow, which he says was arranged by <a href="https://sberbank-university.ru/en/">Sberbank Corporate University</a> as an “educational day” for Russian government officials. The university is a subsidiary of Sberbank, a state-owned bank sanctioned by the EU; its chief executive, Russia’s former minister for economic development, is close to Putin. What was the purpose of the trip? “I didn’t really understand the context,” says Kosinski. “They put me on a helicopter, flew me to a place, I came on the stage. On the helicopter I was given a briefing about who was going to be in the room. Then I gave a talk, and we talked about how AI is changing society. And then they sent me off.”</p> <aside class="element element-rich-link element--thumbnail"> <p> <span>Related: </span><a href="https://www.theguardian.com/technology/2018/jun/13/mexico-us-border-wall-surveillance-artificial-intelligence-technology">'Surveillance society': has technology at the US-Mexico border gone too far?</a> </p> </aside>  <p>The last time I see Kosinski, we meet in London. He becomes prickly when I press him on Russia, pointing to its dire record on gay rights. Did he talk about using facial-recognition technology to detect sexuality? Yes, he says – but this talk was no different from other presentations in which he discussed the same research. (A couple of days later, Kosinski tells me he has checked his slides; in fact, he says, he didn’t tell the Russians about his “AI gaydar”.)</p> <p>Who else was in the audience, aside from Medvedev and Lavrov? Kosinski doesn’t know. Is it possible he was talking to a room full of Russian intelligence operatives? “That’s correct,” he says. “But I think that people who work for the surveillance state, more than anyone, deserve to know that what they are doing is creating real risk.” He tells me he is no fan of Russia, and stresses there was no discussion of spying or influencing elections. “As an academic, you have a duty to try to counter bad ideas and spread good ideas,” he says, adding that he would talk to “the most despicable dictator out there”.</p> <p>I ask Kosinski if anyone has tried to recruit him as an intelligence asset. He hedges. “Do you think that if an intelligence agency approaches you they say: ‘Hi, I’m the CIA’?” he replies. “No, they say, ‘Hi, I’m a startup, and I’m interested in your work – would you be an adviser?’ That definitely happened in the UK. When I was at Cambridge, I had a minder.” He tells me about a British defence expert he suspected worked for the intelligence services who took a keen interest in his research, inviting him to seminars attended by officials in military uniforms.</p> <p>In one of our final conversations, Kosinski tells me he shouldn’t have talked about his visit to Moscow, because his hosts asked him not to. It would not be “elegant” to mention it in the Guardian, he says, and besides, “it is an irrelevant fact”. I point out that he already left a fairly big clue on Facebook, where he posted an image of himself onboard a helicopter with the caption: “Taking off to give a talk for Prime Minister Medvedev.” He later changed his privacy settings: the photo was no longer public, but for “friends only”.</p> <p>• Comments on this piece are premoderated to ensure the discussion remains on the topics raised by the article.</p> <p>Commenting on this piece? If you would like your comment to be considered for inclusion on Weekend magazine’s letters page in print, please email <a href="mailto:weekend@theguardian.com">weekend@theguardian.com</a>, including your name and address (not for publication).</p>\\
Politics & <p>Amber Rudd’s position as home secretary looked increasingly precarious on Sunday night after the Guardian published in full for the first time a <a href="https://www.theguardian.com/politics/2018/apr/29/amber-rudd-faces-new-pressure-over-immigration-targets">letter</a> that showed she had personally set a target for an increase in the enforced deportation of immigrants. </p> 
<p>The publication of the letter came as the home secretary prepared to face Parliament again on Monday, this time to explain why she appeared to mislead the home affairs select committee last week by denying any knowledge of targets.</p> 
<aside class="element element-rich-link element--thumbnail"> 
 <p> <span>Related: </span><a href="https://www.theguardian.com/commentisfree/2018/apr/29/amber-rud-theresa-may-crisis-windrush">Amber Rudd has gone, but now Theresa May faces a new crisis | Isabel Hardman</a> </p> 
</aside> 
<p>After her former junior minister confirmed that she had “ambitions” to increase the numbers being deported, the Guardian <a href="https://www.theguardian.com/politics/2018/apr/29/amber-rudd-letter-to-pm-reveals-ambitious-but-deliverable-removals-target">published in full Rudd’s four-page letter</a> to Theresa May in which she sets an “ambitious but deliverable” target to increase deportations by 10\%. </p> 
<p>The contents of the letter, some of which were first reported by the Guardian as the pressure mounted on Rudd and May over the Windrush scandal, became even more relevant on Friday when she responded to another leaked memo by saying she was unaware of “specific removal targets”. </p> 
<aside class="element element-rich-link element--thumbnail"> 
 <p> <span>Related: </span><a href="https://www.theguardian.com/politics/2018/apr/29/amber-rudd-letter-to-pm-reveals-ambitious-but-deliverable-removals-target">Amber Rudd letter to PM reveals 'ambitious but deliverable' removals target</a> </p> 
</aside> 
<p>Rudd has claimed she did not set, see or approve any targets for removals. But Home Office sources said it was “shamefaced nonsense” to claim the department had not been set specific targets in this area, or that these have not been regularly discussed at the highest levels.</p> 
<p>Over the past week, she has been forced to apologise, express regret or clarify her position five times over what she knew about the Windrush affair and the subsequent row about deportations.</p> 
<figure class="element element-atom"> 
 <gu-atom data-atom-id="af338525-c937-45f7-a4ea-88667478c569" data-atom-type="timeline"> 
  <div>
   <div class="atom-Timeline">
    <p><i>(April 23, 2018)</i>\&nbsp;<strong> </strong></p>
    <p></p>
    <p>Rudd delivered an unprecedented apology to parliament and acknowledged that her department had “lost sight of individuals” and become “too concerned with policy”.</p>
    <p></p>
    <p><i>(April 25, 2018)</i>\&nbsp;<strong> </strong></p>
    <p></p>
    <p>Rudd apologised for failing to grasp the scale of the problem. She told the home affairs select committee: “I bitterly, deeply regret that I didn’t see it as more than individual cases gone wrong that needed addressing. I didn’t see it as a systemic issue until very recently.”</p>
    <p></p>
    <p><i>(April 26, 2018)</i>\&nbsp;<strong> </strong></p>
    <p></p>
    <p>On Thursday morning, Rudd was forced to admit officials did have targets for removals, having previously denied their existence.</p>
    <p>“The immigration arm of the Home Office has been using local targets for internal performance management. These were not published targets against which performance was assessed, but if they were used inappropriately then I am clear that this will have to change."</p>
    <p></p>
    <p><i>(April 26, 2018)</i>\&nbsp;<strong> </strong></p>
    <p></p>
    <p>On Thursday afternoon, Rudd was forced to issue a hasty clarification after appearing to leave the door open to the UK staying in a customs union with the EU.</p>
    <p>“I should have been clearer – of course when we leave the EU we will be leaving the customs union."</p>
    <p></p>
    <p><i>(April 27, 2018)</i>\&nbsp;<strong> </strong></p>
    <p></p>
    <p>In a series of late-night tweets, Rudd apologised for not being aware of documents, leaked to the Guardian, which set out immigration removal targets.\&nbsp;<br></p>
    <p>‘I wasn’t aware of specific removal targets. I accept I should have been and I’m sorry that I wasn’t. I didn’t see the leaked document, although it was copied to my office as many documents are."</p>
    <p></p>
   </div>
  </div>
 </gu-atom> 
</figure> 
<p>Home Office sources have told the Guardian that Immigration Enforcement has been working all year to reach the target of 12,800 enforced returns in 2017-18.</p> 
<p>They have been bracing themselves to acknowledge to ministers that the agency has failed to do so. To meet the goal, it needed to deport 250 people a week, but it has only been able to remove about 225 a week.</p> 
<p>“At the Home Office we work in a target culture,” said a source on Sunday. “The civil service is completely target-based. That’s all we do. It is shamefaced nonsense for Amber Rudd to say otherwise.”</p> 
<aside class="element element-rich-link element--thumbnail"> 
 <p> <span>Related: </span><a href="https://www.theguardian.com/uk-news/2018/apr/29/windrush-citizens-share-their-relief-at-being-listened-to">Windrush citizens share their relief at being listened to</a> </p> 
</aside> 
<p>In an interview on The Andrew Marr Show that appeared to deepen her woes, the <a href="https://www.theguardian.com/uk-news/2018/apr/29/tory-chairman-brandon-lewis-says-he-discussed-deportations-with-amber-rudd">former home office minister Brandon Lewis confirmed on Sunday</a> that he had discussed attempts to increase the number of government deportations with Rudd while a minister in her department. </p> 
<p>The Conservative party chairman told Marr he had talked to Rudd about “ambitions” to increase the number of people deported from Britain but they had not discussed specific targets.<br></p> 
<aside class="element element-rich-link element--thumbnail"> 
 <p> <span>Related: </span><a href="https://www.theguardian.com/help/ng-interactive/2017/mar/17/contact-the-guardian-securely">Contact the Guardian securely</a> </p> 
</aside> 
<p>Lewis’s claims appeared to contradict Rudd’s evidence to the home affairs select committee last Wednesday, when she was asked baldly when targets for removals were set. Rudd told the committee: “We do not have targets for removals.”</p> 
<p>Lewis said that while he had worked with her “on a weekly basis” about their efforts to increase the numbers of illegal immigrants being removed, they had never discussed “particular numbers” in the way that was suggested at the home affairs committee on Wednesday. </p> 
<p>“Yes, I did talk to the home secretary about that and the overall work that we were doing and the overall ambition to see an increase in numbers, but not on the detailed numbers and targets,” he said. </p> 
<p>Diane Abbott MP, the shadow home secretary, said the “Tories’ shameful attempts to cover up their mess” must end. </p> 
<aside class="element element-rich-link element--thumbnail"> 
 <p> <span>Related: </span><a href="https://www.theguardian.com/politics/2018/apr/29/windrush-row-five-questions-amber-rudd-has-yet-to-answer">Windrush row: five questions Amber Rudd has yet to answer</a> </p> 
</aside> 
<p>“Theresa May has sent minister after minister out to protect her cruel legacy, misleading parliament and the public in the process. This chaos has gone on for far too long. It’s time for Rudd to go and for the government to rethink its whole approach,” she said. </p> 
<p>The latest furore was sparked on Friday when the <a href="https://viewer.gutools.co.uk/politics/2018/apr/27/amber-rudd-was-told-about-migrant-removal-targets-leak-reveals">Guardian published details from a separate confidential memo that was sent to Rudd</a> in June last year.</p> 
<p>Prepared by Hugh Ind, the director general of Immigration Enforcement in the Home Office, it picked up on the new policy outlined by Rudd in her letter to Theresa May. </p> 
<p>The document stated that his agency had “set a target of achieving 12,800 enforced returns in 2017/18 … this will move us along the path towards the 10\% increased performance on enforced returns which we promised the home secretary earlier this year”.</p> 
<p>While Rudd has denied seeing the six-page briefing note, it was also sent to at least eight of the Home Office’s most senior officials, including the senior director of national and international operations in Immigration Enforcement and the director general of the Passport Office.</p> 
<p>Downing Street has said Rudd continues to enjoy the “full confidence” of May and she has received public support from both senior ministers and Tory backbenchers. <br></p> 
<p>The home secretary insisted on Friday night that she had not seen the leaked memo revealed by the Guardian on Friday “although it was copied to my office, as many documents are”.</p> 
<p>She repeated her claim that she “wasn’t aware of specific removal targets”, adding: “I accept I should have been and I’m sorry that I wasn’t.”</p> 
<p>She promised to make a fresh statement to MPs on Monday about the affair, and concluded: “As home secretary I will work to ensure that our immigration policy is fair and humane.”</p> 
<aside class="element element-rich-link element--thumbnail"> 
 <p> <span>Related: </span><a href="https://www.theguardian.com/uk-news/2018/apr/29/windrush-citizens-share-their-relief-at-being-listened-to">Windrush citizens share their relief at being listened to</a> </p> 
</aside> 
<p>More than 200 MPs have written to May urging her to enshrine in law the promises made to those affected – including a commitment to resolve their immigration status as quickly as possible. <br></p> 
<p>The letter, co-ordinated by Labour backbencher David Lammy, was signed by MPs from Labour, the Liberal Democrats, the SNP, Plaid Cymru and the Greens as well as one Conservative, Anne-Marie Morris. <br></p> 
<p>A Home Office spokesman said the commitments could all be carried through under existing immigration laws. </p> 
<p>“Detailed policy guidance and amendments to existing secondary legislation, where necessary, will be brought forward as soon as possible – and in some areas, we will want to consult with the communities affected first to ensure it best meets their needs,” the spokesman said.</p>\\
Life and style & <p>Walking through town at night, the week after <a href="https://www.theguardian.com/world/international-womens-day">International Women’s Day</a>, and the shop windows screamed empowering messages. Mannequins’ T-shirts were lit from behind, a headless army subtitled in Impact font. STRONG WOMAN. FIERCE. HEAR ME ROAR. GIRL CREW. FEARLESS FEMALE. FEMINIST. With every bus stop, I got more irritated.</p> <p>But first let me stress that the way brands supported International Women’s Day was amazing. Please, never imagine for a second that I would want to undermine the important political work they’re doing with their buy-one-get-one-free yoga classes, and their hand-cream samples, and their phone cases that say “Cheeky Grrrl” in glitter. I’d heard of the myriad promotions businesses were planning – indeed over the past month I’ve kept a special folder in my inbox into which press releases from zoos, beer companies, juice bars and fashion labels celebrating women by selling them hats or salad cream automatically fell – but the scale of support truly overwhelmed me. McDonald’s turned their M over on the internet. So it was a W. Which was really powerful. Though, admittedly, by the time I’d finished the complimentary thimble of prosecco handed to me outside a pilates studio, and considered next door’s International Women’s Day discount on a bikini wax, the sheen had worn off a little. I passed a Tesco Metro, and in my weary wokeness, read it as Tesco \#MeToo.</p>  <aside class="element element-pullquote element--supporting"> <blockquote> <p>Show me a feisty woman and I’ll show you someone who finds it hard to be alone</p> </blockquote> </aside>  <p>Why did all this annoy me so much? It’s not just the branded feminism. It’s a bit that, but not all that. It’s not just the use of feminism as a marketing opportunity, plonking it in the window of H\&amp;M scrubbed clean of all guts and pain. It’s not just the commodification of it, the relentless selling of stuff, the viscous flood of “content” that leaves you feeling slightly filthy as, halfway through learning about trafficked women or the rise of backstreet abortions, you find yourself seriously considering the purchase of a new moisturising leg oil that smells of vanilla. It’s not just that. It’s not even simply the fact of the messaging, though “Fearless”, “Fierce”, “Feisty” – it all makes me shrivel. The condescension.</p> <p>Who decided we couldn’t be scared? Why all this interminable energy? To be female, even to be feminist, does not require full-time perkiness. In fact, in my experience quite a lot of the time to be a woman is also to be a bit cross, with a slight headache, occasionally overwhelmed. But I suppose that’s less snappy on a T-shirt. We don’t all have to be incredible, or inspiring – most of us just want to be accepted as equal. Show me a fearless woman and I’ll show you someone who has pressed their fear into a crack inside them like a prayer in a wall, or who carries it hidden in their palm, a small thorn made of experience. Show me a feisty woman and I’ll show you someone who finds it hard to be alone. These are not essential qualities. They sit alongside the less marketable ones, always.</p> <p>It’s the same with the messaging that promises women we are all beautiful. Are we? Or are most of us fairly weatherworn and sullen, and thrown by the meaning of the word. If every woman is beautiful, then what is its value? And, spoken in the context of female empowerment, what does the insistence that everyone is beautiful tell us about beauty itself? Perhaps the intention is that we will read it on a magazine advert and then relax totally, all that anxiety about weight and skin and age now dissipated to allow concentration on overthrowing the patriarchy and making our flats smell nice. Instead, its effect is that we learn “beauty” is core to our feminine experience, ingrained and important. If you don’t feel beautiful then it’s nothing to do with the myriad reminders that you need to work harder to pass as female, it’s your own silly-billy fault.</p> <p>OK, but here is the thing that annoys me the most as I trundle through town with a face on. It’s fairly common practice now that to wear, say, a Ramones T-shirt, you should at the very least own an album and be able to sing along to their hits. If you wear a T-shirt that says PRIYA’S HEN PARTY 2001 then those piqued enough by the penis acronym on the back should expect no less than a good half an hour of anecdotes about what you got up to at Center Parcs that fateful weekend in June.</p> <p>Similarly, if you wear a T-shirt saying FEMINIST, then the expectation should be that you have performed some action to further women’s rights, that you have marched, voted, campaigned. That the company selling it has invested in women’s projects. That the wearer will at the very least be able to sing along to feminism’s greatest hits – reproductive rights, sexual violence, equal pay. Otherwise, what is it other than political appropriation?</p> <p>The irony, of course, is that the feminist slogan that really rolls off the tongue and would look fabulous on a T-shirt, is the simplest: “Deeds not words”, babes.</p> <p><em>Email Eva at </em><a href="mailto:e.wiseman@observer.co.uk"><em>e.wiseman@observer.co.uk</em></a><em>or follow her on Twitter </em><a href="https://twitter.com/EvaWiseman?ref\_src=twsrc\%5Egoogle\%7Ctwcamp\%5Eserp\%7Ctwgr\%5Eauthor"><em>@EvaWiseman</em></a></p>\\
Australia news & <figure class="element element-interactive interactive" data-interactive="https://interactive.guim.co.uk/2017/01/essential-poll/boot.js" data-canonical-url="https://interactive.guim.co.uk/2017/01/essential-poll/boot.js" data-alt="key=1E4CB1H7V\_\_TuOam3PW8zX9eIiaYUrQM39qPgWvrhK6I"> </figure>\\
\bottomrule
\end{tabular}}
\end{table}

\rowcolors{2}{white}{white}

\rowcolors{2}{gray!6}{white}

\begin{table}[!h]

\caption{\label{tab:data_collection}Table containing SCIENCE  News}
\centering
\resizebox{\linewidth}{!}{
\begin{tabular}[t]{ll}
\hiderowcolors
\toprule
sectionName & body\\
\midrule
\showrowcolors
Books & <p>As science journalist Peter Brannen points out, life is extremely fragile, a “thin glaze of interesting chemistry on an otherwise unremarkable, cooling ball of stone”. So fragile, in fact, that in the planet’s history there have been five mass extinctions, when nearly all life has been wiped out. The question hanging over this book is whether the current most dominant species on the planet is about to cause a <a href="https://www.theguardian.com/environment/2017/jul/10/earths-sixth-mass-extinction-event-already-underway-scientists-warn" title="">sixth mass extinction</a>.</p>  <figure class="element element-image element--thumbnail" data-media-id="0cd8c5d4c0d8ecab75da68ff2210fa0cb1594793"> <img src="https://media.guim.co.uk/0cd8c5d4c0d8ecab75da68ff2210fa0cb1594793/193\_0\_1472\_1471/1000.jpg" alt="Peter Brannen." width="1000" height="1000" class="gu-image" /> <figcaption> <span class="element-image\_\_caption">Peter Brannen.</span> <span class="element-image\_\_credit">Photograph: Matt Misisco</span> </figcaption> </figure>  <p>To answer this, Brannen takes us back millions of years: “to see the world through the lens of geology is to see the world for the first time”. His evocative prose brings the “incomprehensible eternities” of ancient history vividly alive as he meets the scientists who are piecing together the events of these violent periods when life became impossible. As one says, ominously for our own era: “When there are severe, rapid changes in the carbon cycle, it doesn’t end well.” A remarkable journey into the deep past that has much to teach us about the future of our planet.</p> <p>• <em>The Ends of the World: Volcanic Apocalypses, Lethal Oceans and our Quest to Understand Earth’s Past Mass Extinctions </em>is published by Oneworld. To order a copy for £9.45 (RRP <strong> </strong>£10.99) go to <a href="https://guardianbookshop.com/ends-of-the-world-608527.html?utm\_source=editoriallink\&amp;utm\_medium=merch\&amp;utm\_campaign=article" title="">guardianbookshop.com</a> or call 0330 333 6846. Free UK p\&amp;p over £10, online orders only. Phone orders min p\&amp;p of £1.99.</p>\\
Education & <p>University technical colleges – part of the <a href="https://www.theguardian.com/politics/2018/sep/23/ban-anonymous-accounts-angela-rayner-tells-social-media-firms">free schools </a><a href="https://www.theguardian.com/politics/2018/sep/23/ban-anonymous-accounts-angela-rayner-tells-social-media-firms">changes</a> pushed through by Michael Gove – have been described as ineffective and unpopular by a report that found more than half their students dropped out.</p> <p>Of those who remained at UTCs, many made poor progress, with even previously high-achieving students performing less well in their exams, according to the <a href="https://epi.org.uk/">Education Policy Institute</a>.</p> <p>About 60 UTCs have opened since 2011, after being championed by the Conservative Lord Baker and the then prime minister, David Cameron, enrolling students aged 14 to 18 and designed to encourage the study of science, technology and engineering.</p> <aside class="element element-rich-link element--thumbnail"> <p> <span>Related: </span><a href="https://www.theguardian.com/education/2015/sep/22/university-technical-colleges-five-years-on-the-jurys-still-out">University technical colleges: five years on, the jury's still out</a> </p> </aside>  <p>But despite official encouragement and lavish funding, they have failed to generate enthusiasm among parents, and 10 have subsequently <a href="https://www.theguardian.com/education/2017/feb/07/greater-manchester-university-technical-college-closes-three-years">closed or converted</a> into conventional schools.</p> <p>David Laws, the EPI’s executive chairman, said after spending “hundreds of millions of pounds” on UTCs, the Department for Education (DfE) should halt any further expansion until their effectiveness has been reviewed.</p> <p>Baker, a former education secretary who chairs the Baker Dearing Trust, which promotes UTCs, accused EPI researchers of ignoring evidence.</p> <p>“EPI start with their conclusion that a 14-18 institution cannot fit into an 11-18 system and then use statistics to support that,” he said.</p> <p>“It is a pity that they did not take up Baker Dearing’s offer to visit several of our 50 UTCs and speak to teachers, students and parents.”</p> <p>The EPI found many UTCs struggled to recruit students, and failed to retain the majority of those who did enrol. More than half of all UTC students left between the ages of 16 and 17 after taking GCSEs, while more continued to quit before finishing key stage five at the age of 18.</p> <p>One in five UTCs were rated as inadequate by Ofsted inspectors, the EPI found, while a further 40\% were rated as requiring improvement – well above the national average for mainstream schools in England.</p> <p>Julian Gravatt, the deputy chief executive of the <a href="https://www.aoc.co.uk/">Association of Colleges</a>, said the report showed UTCs “are an experiment that hasn’t worked”.</p> <p>“Given the high level of support given to them by the DfE and the capital funding allocated by the Treasury, this is obviously depressing,” he said.</p> <p>The analysis also found UTC students’ GCSE results were almost a grade lower than their peers at secondary schools. “Significantly, this poor progress is particularly acute for high attainers, who make over a grade’s less progress than high attainers in all state-funded schools,” the EPI noted.</p> <p>The National Education Union said the report backed up its research, which found Black Country university technical college in Walsall cost more than £11m between its opening in 2011 and closure in 2015, with 158 students enrolled out of a planned 480.</p> <p>Another UTC in Burnley cost £10m but closed three years after opening in 2013, with 113 students enrolled despite plans for 800.</p> <p>The EPI did note several benefits from UTCs, including that they offer a wider range of technical subjects such as computer science than other schools.</p> <p>The report concluded that existing UTCs should be repurposed as 16-18 colleges offering post-GCSE technical<strong> </strong>qualifications, such as the <a href="https://www.theguardian.com/education/2017/mar/08/t-levels-aim-to-improve-technical-education-and-improve-uk-productivity">government’s promised T-</a><a href="https://www.theguardian.com/education/2017/mar/08/t-levels-aim-to-improve-technical-education-and-improve-uk-productivity">levels</a>.</p> <p>But Gravatt said such a change needed careful consideration. “The 16-to-18 sector of education is already a chaotic and underfunded market,” he said.</p> <p>A DfE spokesperson said UTCs were an important part of England’s diverse education system.</p> <p>“Our most recent data shows that when young people leave a UTC, they are headed in the right direction – with twice as many key stage four students beginning an apprenticeship compared to the national average,” they said.</p>\\
Science & <h2>This week’s biggest stories</h2> <p>This weeks headlines have been full of our ancestors, most prominently the discovery that <a href="https://www.theguardian.com/science/2018/feb/22/neanderthals-not-humans-were-first-artists-on-earth-experts-claim">Neanderthals painted on cave walls in Spain 65,000 years ago</a> – tens of thousands of years before the arrival of modern humans. Some say this made them the first artists on Earth, but Guardian art critic <a href="https://www.theguardian.com/artanddesign/2018/feb/23/neanderthals-cave-art-spain-astounding-discovery-humbles-every-human">Jonathan Jones has some interesting points to make about that claim</a>. Neanderthals aside, an intriguing theory that <a href="https://www.theguardian.com/science/2018/feb/20/homo-erectus-may-have-been-a-sailor-and-able-to-speak"><em>Homo erectus</em> may have been a sailor and able to speak</a> has been put forward. Researchers also say that ancient DNA reveals that the <a href="https://www.theguardian.com/science/2018/feb/21/arrival-of-beaker-folk-changed-britain-forever-ancient-dna-study-shows">arrival of the Beaker folk changed Britain forever</a>. Genetic analysis has shown that at least 90\% of the ancestry of Britons was replaced by a wave of migrants, who arrived about 4,500 years ago. Also unpicking the past were some Australian scientists, who say that if you want to know about <a href="https://www.theguardian.com/science/2018/feb/22/want-to-know-about-t-rex-chase-an-ibis-around-a-track-scientists-say"><em>T rex</em><em>’s</em> locomotion then watching how an ibis moves</a> might just be the key. Another team of antipodean researchers are using <a href="https://www.theguardian.com/science/2018/feb/21/tasmanian-tiger-joey-3d-scans-may-unlock-evolutionary-mystery">3D scans to try to unlock the evolutionary history of the Tasmanian tiger</a>. Shedding light in a totally different way were scientists who have used synthetic bioluminescent molecules to make <a href="https://www.theguardian.com/science/2018/feb/22/scientists-make-cells-glow-so-brightly-they-can-be-seen-outside-the-body">brain cells glow so brightly they can be seen outside the body</a>. Used only in animals so far, the luminescence allows researchers to track individual cells in animals with unprecedented accuracy. The future’s bright, it seems ...</p> <p><strong><a href="https://www.theguardian.com/science">More news from Guardian Science</a> | <a href="https://www.theguardian.com/science/2016/jun/07/sign-up-for-lab-notes-the-guardians-weekly-science-update">Sign up to Lab notes</a></strong><br></p> <p>\_\_\_</p> <h2>Archaeological controversy</h2>  <figure class="element element-image" data-media-id="0918876bccbe7e990465d994683509299a93ce47"> <img src="https://media.guim.co.uk/0918876bccbe7e990465d994683509299a93ce47/187\_0\_942\_565/500.jpg" alt="Clovis points from the Rummells-Maske Site." width="500" height="300" class="gu-image" /> <figcaption> <span class="element-image\_\_caption">Clovis points from the Rummells-Maske Site.</span> <span class="element-image\_\_credit">Photograph: Billwhittaker at English Wikipedia via Wikimedia Commons</span> </figcaption> </figure>  <p>Last month’s release of <a href="http://www.cbc.ca/natureofthings/episodes/ice-bridge">The Ice Bridge</a>, an episode in the Canadian Broadcasting Corporation series <a href="http://www.cbc.ca/natureofthings/">The Nature of Things</a> has once again revived public discussion of a controversial idea about how the Americas were peopled known as the <a href="https://www.theguardian.com/science/1999/nov/28/archaeology.uknews">“Solutrean hypothesis”</a>. This idea suggests a European origin for the peoples who made the Clovis tools, the first recognized stone tool tradition in the Americas. One of the experts appearing on the documentary was Jennifer Raff, who writes for our blog The Past and The Curious and is a geneticist who specialises in the study of human variation among contemporary and ancient populations. She <a href="https://www.theguardian.com/science/2018/feb/21/rejecting-the-solutrean-hypothesis-the-first-peoples-in-the-americas-were-not-from-europe">shares her thoughts about it – and why she see the ideas portrayed within the documentary as unsettling, unwise, and scientifically implausible</a>. </p> <h2>\_\_\_</h2> <h2>Straight from the lab – top picks from our experts on the blog network</h2>  <figure class="element element-image" data-media-id="b2bfeb110a9c55498aa7c00dae14bf8e0bbbd1ed"> <img src="https://media.guim.co.uk/b2bfeb110a9c55498aa7c00dae14bf8e0bbbd1ed/0\_357\_3960\_2376/1000.jpg" alt="Looking at the nuts and bolts of creature creation." width="1000" height="600" class="gu-image" /> <figcaption> <span class="element-image\_\_caption">Looking at the nuts and bolts of creature creation.</span> <span class="element-image\_\_credit">Photograph: Ronald Grant Archive</span> </figcaption> </figure>  <p><a href="https://www.theguardian.com/science/blog/2018/feb/22/how-to-make-a-monster-whats-the-science-behind-shelleys-frankenstein">How to make a monster: what’s the science behind Shelley’s Frankenstein?</a> | Notes \&amp; Theories</p> <blockquote class="quoted"> <p>In terms of the technical aspects of building a creature from scraps, many people focus on the collecting of the raw materials and reanimation stages. It’s understandable as there are many great stories about <a href="https://books.google.co.uk/books?id=NEuthk74yG0C\&amp;dq=death+dissection+and+the+destitute\&amp;hl=en\&amp;sa=X\&amp;ved=0ahUKEwjhlLLsuKjZAhWIK8AKHcBBAt0Q6AEIJTAA">grave-robbers and dissection rooms</a> as well as <a href="http://www.exclassics.com/newgate/ng464.htm">electrical experiments that were performed on recently executed murderers</a>. But there quite a few stages between digging up dead bodies and reanimating a creature. The months of tedious and fiddly surgery to bring everything together are often glossed over, but what virtually no one mentions is how difficult it would have been to keep the bits and pieces in a suitable state of preservation while Victor worked on his creation. Making a monster takes time, and bodies rot very quickly.</p> </blockquote> <p><a href="https://www.theguardian.com/science/2018/feb/21/the-new-specimen-forcing-a-radical-rethink-of-archaeopteryx">The new specimen forcing a radical rethink of Archaeopteryx</a> | Lost Worlds Revisited</p> <blockquote class="quoted"> <p>The new specimen represents a largely intact skeleton lying on its left side. The stone slab it is preserved in was found in a number of pieces, resulting in some bone loss along the vertebral column. No feathers are preserved. The skull has been dislocated from the body and rotated downwards and back, giving the specimen a characteristic look that is very different from the classical <a href="https://blogs.scientificamerican.com/laelaps/the-secret-of-the-dinosaur-death-pose/">death pose</a> with the head and neck arching backwards towards the tail. The stone slab also contains a beautiful ammonite, for those who prefer their fossils spineless.</p> </blockquote> <p><a href="https://www.theguardian.com/science/life-and-physics/2018/feb/20/how-much-mass-does-the-w-boson-have">How much mass does the W boson have?</a> | Life and Physics</p> <blockquote class="quoted"> <p>The mass of the W boson is a key parameter of the Standard Model of particle physics, the framework we use to describe all the fundamental forces and particles<sup>1</sup>. The mass itself comes from the Brout-Englert-Higgs mechanism, as does the mass of all fundamental particles in the Standard Model. But the W is more intimately intertwined than most with the <a href="https://www.theguardian.com/science/higgs-boson">Higgs boson</a>, the last and crucial particle of the Standard Model discovered in 2012. Because of this, precise measurements of the W mass have a powerful influence on the possible ways the Standard Model might be extended to explain some of the things it misses. Every percentage shaved off the uncertainty consigns more theoretical possibilities to the scrapheap.</p> </blockquote> <p><em><sup>1</sup>Except gravity, but we tend to ignore that.</em></p> <p><strong><a href="https://www.theguardian.com/science/series/science-blog-network">Visit the Science blog network</a></strong></p> <p>\_\_\_<br></p> <h2>Eye on science – this week’s top images</h2>  <figure class="element element-image" data-media-id="5ddee6f0c9641c41078ecc545305dea7509ff15e"> <img src="https://media.guim.co.uk/5ddee6f0c9641c41078ecc545305dea7509ff15e/0\_0\_3600\_2401/1000.jpg" alt="Meghann Riepenhoff Ecotone \#250 (Bainbridge Island, WA 12.28.17, Draped in Mixed Precipitation) Unique dynamic cyanotype" width="1000" height="667" class="gu-image" /> <figcaption> <span class="element-image\_\_caption">Meghann Riepenhoff Ecotone \#250 (Bainbridge Island, WA 12.28.17, Draped in Mixed Precipitation) Unique dynamic cyanotype</span> <span class="element-image\_\_credit">Photograph: Courtesy of EUQINOMprojects</span> </figcaption> </figure>  <p><a href="https://www.theguardian.com/artanddesign/gallery/2018/feb/23/meghann-riepenhoff-cyanotypes-tidal-patterns">This lovely gallery</a> shows the work of US artist Meghann Riepenhoff. Fascinated by the nature of humans’ relationship to an impermanent landscape, Riepenhoff’s photographic process captures tidal patterns made by ocean waves, sand and marine life. </p>\\
Science & <p>The longest lunar eclipse of the century (so far) <a href="https://www.theguardian.com/science/2018/jul/22/longest-lunar-eclipse-century-uk-celestial-thrill">will take place across Friday night and Saturday morning</a>, as the moon is totally eclipsed by the Earth for one hour and 43 minutes. During this time, people around the world will be able to see a “blood moon”, as the Earth’s satellite turns red.</p> 
<figure class="element element-atom"> 
 <gu-atom data-atom-id="c75a6f07-f9c8-494e-94b6-8328b05b5c95" data-atom-type="qanda"> 
  <div>
   <div class="atom-Qanda">
    <p></p>
    <p>If you have good photos of the moon during Friday’s event we’d like to see them. Be sure to tell us where you took your image, and any other information you think useful. <br></p>
    <p>You can share your photos by filling in <a href="https://guardiannewsandmedia.formstack.com/forms/lunar\_eclipse\_blood\_moon\_photos">this encrypted form</a>. One of our journalists may be in touch and we will consider some of your responses in our reporting. You can read terms of service\&nbsp;<a href="https://www.theguardian.com/help/terms-of-service">here</a>.</p>
    <p></p>
   </div>
  </div>
 </gu-atom> 
</figure> 
<h2>What is a blood moon?</h2> 
<p>A “blood moon” is a name given to the moon during a lunar eclipse. A lunar eclipse occurs when the Earth passes between the sun and the moon, casting the moon into shadow. </p> 
<p>Unlike with a solar eclipse, when the moon passes between the Earth and sun, blocking the sun’s light, the moon is not “turned dark” during an eclipse, but instead appears to turn red.</p> 
<aside class="element element-rich-link element--thumbnail"> 
 <p> <span>Related: </span><a href="https://www.theguardian.com/science/2018/jul/22/starwatch-red-marvel-that-is-a-lunar-eclipse">Starwatch: red marvel that is a lunar eclipse</a> </p> 
</aside> 
<p>“If the Earth was a big smooth ball with no atmosphere that would be the end of the subject, it would just go dark, like with a new moon,” said Chris Tinney, professor in the physics department at the University of New South Wales. “But because there is light scattered through the atmosphere of the Earth, some of the sun’s light gets bounced around the edge of the Earth to hit the moon.”</p> 
<p>Because blue and violet wavelengths are scattered more than red and orange ones, more of the red wavelengths reach the moon, making the moon appear red. </p> 
<h2>How rare is it?</h2> 
<p>Lunar eclipses are less common than solar eclipses, with a maximum of three occurring in any given location per year, though some years there can be none. However, each lunar eclipse is visible from more than half the Earth. </p> 
<p>If this eclipse is rare, it is because it will last for so long. The moon will be in the Earth’s shadow for four hours and totally eclipsed for one hour and 43 minutes, which is just short of the theoretical limit of a lunar eclipse (one hour and 47 minutes). The eclipse will last for so long on Friday night because the moon will be passing through the centre of the Earth’s shadow. </p> 
<h2>Where and when can I see it?</h2> 
<p>The best view of the eclipse will be from east Africa, the Middle East, across to India and the westernmost tip of China. But there still be reasonable views for people in the rest of Africa, Europe, other parts of Asia, Australia and the eastern tip of South America. </p> 
<aside class="element element-rich-link element--thumbnail"> 
 <p> <span>Related: </span><a href="https://www.theguardian.com/science/2016/nov/15/supermoonfail-internets-worst-photographs-of-the-perigee-full-moon">\#Supermoonfail: internet's worst photographs of the perigee full moon</a> </p> 
</aside> 
<p>North America and Greenland look to be the only places that will miss out entirely.</p> 
<p>In Australia: the moon will start getting red at 4:30am AEST on Saturday morning, with the total eclipse occurring between 5:30 and 6:30am, before the moon sets at 6:55am. The moon will be close to the horizon, so make sure to look west-south-west.</p> 
<p>In the UK: the partial eclipse will begin at 8:30pm, the total eclipse will occur between 9:20pm BST on Friday, with the moon visible to the south-east, until 10:13pm.</p> 
<p>In east Africa: the partial eclipse will begin at 9:30pm EAT, with the moon completely red between 10:30pm and 12:13am. This region will have one of the best views of the eclipse.</p> 
<p>In India: the total eclipse will begin at 1am IST, finishing at 2:43am.</p> 
<h2>Tips for watching the blood moon</h2> 
<p>“The best place to see it is out in the country away from lights,” said Tinney. “If you’re living in [a city] then there’s a lot of light pollution from the night sky, so the contrast between the moon and the sky won’t be as great.” </p> 
<p>It is safe to look at the moon during an eclipse.</p> 
<p>Timeanddate.com has a good <a href="https://www.timeanddate.com/eclipse/in/india/delhi">eclipse tracker</a>, which allows you to enter your location and find out when best to see the eclipse.</p> 
<p>For those in areas that will miss out on the blood moon, the Royal Observatory at Greenwich plans to stream live pictures of the event.</p>\\
World news & <p>Arms manufacturers are spending millions of pounds a year promoting their brands in Britain’s schools, the <em>Observer</em> has learned.</p> <p>The companies, which between them have sold tens of billions of pounds of weapons to overseas governments, including those with poor human rights records, sponsor a series of school events at which their brands are prominently on display. In addition, they issue teaching materials for use in classrooms that promote the defence sector, sponsor competitions and award prizes.</p> <p>One company even deployed a high-profile children’s television presenter to promote its activities in a school, while another developed a missile simulator for pupils to “play with”. Critics accuse the companies of trying to “normalise their appalling business” in the minds of the young, but the body representing the defence sector says such an approach is vital if the UK is to produce a future generation of engineers.</p>  <aside class="element element-pullquote element--supporting"> <blockquote> <p>The fact that companies that arm and support human rights abusing regimes are targeting children is extremely concerning</p> <footer> <cite>Andrew Smith, CAAT</cite> </footer> </blockquote> </aside>  <p>“When these companies are promoting themselves to children they are not talking about the deadly impact their weapons are having,” said Andrew Smith of <a href="https://www.caat.org.uk/">Campaign Against Arms Trade</a>. “Many of these companies have profited from war and fuelled atrocities around the world. Schools are vital to our society and should never be used as commercial vehicles for arms companies. It is time for arms companies to be kicked out of the classroom.”</p> <p>BAE Systems, Europe’s largest arms company whose fighter jets are currently being used by Saudi forces in Yemen – where there have been large numbers of strikes on civilian buildings – visited 420 schools across the UK last year and prepared lesson plans for children as young as seven.</p> <p>The company promotes <a href="https://twitter.com/Roadshow\_Team">its roadshows on Twitter </a>and other social media. One event included an appearance<a href="https://www.baesystemseducationprogramme.com/roadshow.php"> by CBeebies television presenter Maddie Moate </a>who, according to BAE, was there to “join in the fun and take a few ‘selfies’ for her own personal collection”.</p> <p>In <a href="https://www.slideshare.net/assocpm/jonathan-waite-slides">an online presentation, BAE</a> states that it spends tens of millions of pounds a year on reaching pupils as young as four. Among worksheets issued to schoolchildren were some encouraging them to think about how BAE’s special camouflage system could have “significant advantages on the battlefield” by allowing tanks to become invisible to hostile thermal imaging systems.</p> <p>Another sheet encourages pupils to look at the company’s past initiatives to find out “more about how shapes of aeroplanes, ships, submarines and tanks have changed over the years”.</p>  <figure class="element element-image element--supporting" data-media-id="8f9f0ee37fb86b0d06a1f22d01d7adc75b186f95"> <img src="https://media.guim.co.uk/8f9f0ee37fb86b0d06a1f22d01d7adc75b186f95/423\_196\_1836\_2293/801.jpg" alt="CBeebies presenter Maddie Moate." width="801" height="1000" class="gu-image" /> <figcaption> <span class="element-image\_\_caption">CBeebies presenter Maddie Moate has appeared at one of BAE Systems’ roadshows.</span> <span class="element-image\_\_credit">Photograph: Ken McKay/ITV/REX/Shutterstock</span> </figcaption> </figure>  <p>Since 2005, 213,000 young people have seen a BAE roadshow, according to the company. BAE also claims to have 845 “ambassadors” – comprised mainly of school governors across Britain.</p> <p>But its activities have proved controversial. Last year there were complaints from parents whose <a href="http://www.thenational.scot/news/15539169.Parents\_\_fury\_as\_defence\_giant\_BAE\_delivers\_pupil\_workshop/">children were taken out of classes at Glasgow Gaelic School to attend a BAE event</a>.</p> <p>A spokeswoman for BAE defended its interest in schools. “As a world leader in advanced engineering and technology, our education and skills activities inspire the next generation of engineers to help address the critical skills gap,” she said. “We invest in a diverse portfolio of programmes aimed at encouraging more young people to study STEM (science, technology, engineering, maths) subjects, which is vital for the UK economy.”</p> <p>Raytheon, the fourth-largest arms company in the world, which has sold bombs and missiles to Israel and Saudi Arabia and whose weapons have been used in Yemen, runs an annual competition across the UK for pupils to build model drones. The US company’s website says it supports science and technology programmes “<a draggable="true" href="https://www.raytheon.com/uk/news/feature/countdown-change">in primary schools, secondary schools, universities and colleges</a>”.</p> <p>Thales, the world’s 10th-largest arms company, whose customers include Saudi Arabia, the United Arab Emirates, Egypt and Kazakhstan, has its<a draggable="true" href="https://www.thalesgroup.com/en/global/presence/europe/united-kingdom/stem-education-and-outreach/adventures-raybot-faybot"> own mascots, Raybot and Faybot,</a> who are used to promote the French company’s education tools.</p> <p>It produces teaching resources and lesson plans for teachers, sponsors the <a href="https://www.thebigbangfair.co.uk/whats-on/activities-2018/thales/">Big Bang Fair</a> and regional events across the country, and has designed a missile simulator as “a new activity for children to play with that related to our work and would help inspire them to consider engineering for a future career”.</p> <p>French MBDA, whose missiles are also being used by Saudi forces in Yemen, runs a “robot rumble” competition where pupils compete to design and build a robot. Each robot is put through <a draggable="true" href="https://www.mbdacareers.co.uk/news-media/mbda-blog/get-ready-rumble/">“a shake test to represent the tests MBDA put their missiles through”</a>, according to the promotional website.</p>  <aside class="element element-pullquote element--supporting"> <blockquote> <p>Events and challenges run by industry in partnership with schools help to inspire the next generation of engineers</p> <footer> <cite>Paul Everitt, ADS chief executive</cite> </footer> </blockquote> </aside>  <p>Leonardo, an Italian company which makes naval artillery and armoured vehicles, “actively supports education and skills development <a href="http://www.uk.leonardocompany.com/about-us/profile">through partnerships with schools, colleges and universities throughout the country</a>, investing substantially in school engagement and supporting Science, Technology, Engineering and Maths (STEM) curriculums”.</p> <p>And Rolls-Royce, whose military aircraft engines service “160 customers in 103 countries”, sponsors a Cub scientist activity badge for the Scout Association.</p> <p>Paul Everitt, chief executive of ADS, the trade body which represents defence contractors, said it was important for the sector to engage with schools.</p> <p>“UK defence companies play a vital role in local communities, providing high-value, long-term jobs with rewarding career paths for those choosing apprenticeships, graduate or post-graduate routes,” he said. “The UK has a national shortage of engineers. Events and challenges run by industry in partnership with schools help to inspire the next generation of engineers and boost access to careers in an innovative and technologically advanced sector. In their engagement with schools, defence companies focus on encouraging pupils to study maths and sciences.”</p> <p>But Smith called on schools to sever their links with arms companies. “The fact that companies that arm and support human-rights-abusing regimes are targeting such young children is extremely concerning,” he said. “Arms companies aren’t targeting schools because they care about education. They are doing it because they want to improve their reputations and normalise their appalling business.”</p>\\
Technology & <p>James Murdoch is the favourite to take over the chairmanship of volatile electric-carmaker <a href="https://www.theguardian.com/technology/tesla">Tesla</a> from its embattled founder, <a href="https://www.theguardian.com/technology/elon-musk">Elon Musk</a>, according to reports.</p> <p>The son of newspaper, satellite TV and movie studio tycoon <a href="https://www.theguardian.com/media/rupert-murdoch">Rupert Murdoch</a> is now the lead candidate for the Tesla chair, the Financial Times reported on Wednesday, citing two people briefed on the discussions.</p> <p>Musk agreed he would step down as chairman by mid-November but remain as chief executive officer, after he and Tesla reached a <a href="https://www.theguardian.com/technology/2018/sep/29/elon-musk-tesla-40m-sec-case-tweets">settlement</a> with US financial watchdog the Securities and Exchange Commission (SEC) last month in which they agreed to pay \$20m each to financial regulators. The billionaire had claimed in early August to have secured funding to take Tesla private, sending markets scrambling.</p> <p>There then followed much concern about his health and stability after he talked about sleeping little because of the high stress of the job, and <a href="https://www.theguardian.com/technology/2018/sep/07/tesla-chief-elon-musk-smokes-marijuana-on-live-web-show">he smoked a joint during a </a><a href="https://www.theguardian.com/technology/2018/sep/07/tesla-chief-elon-musk-smokes-marijuana-on-live-web-show">podcast interview</a> amid swirling speculation about drug use.</p> <p>James Murdoch is currently chief executive of <a href="https://www.theguardian.com/media/21st-century-fox">21st Century Fox</a>, but will leave the role when the entertainment giant completes the sale of the majority of its assets to Disney, and will be succeeded by his brother Lachlan. He joined Tesla’s board last year as a non-executive director and has reportedly said he wants the job of chair.</p> <p>The Tesla board is understood to have not yet made a final decision about Musk’s successor and may still appoint an external candidate.</p> <p>On Tuesday, Murdoch resigned from the board of Sky plc, the owner of Sky News, as part of Comcast’s takeover of the company, beating 21st Century Fox in an intense bidding war. He is also contemplating starting a technology investment fund.</p> <p>“The Tesla chairman job is perfect for James,” a person briefed on the discussions told the FT. “He’s working on this fund and will be sitting next to Elon … he’s going to get access to so much deal flow.”</p> <p>However, Musk denied the Financial Times story <a href="https://twitter.com/elonmusk/status/1050164164805545984">in a response to a tweet about the report</a>, writing: “This is incorrect.”</p>  <figure class="element element-tweet" data-canonical-url="https://twitter.com/elonmusk/status/1050164164805545984">  <blockquote class="twitter-tweet"><p lang="en" dir="ltr">This is incorrect</p>\&mdash; Elon Musk (@elonmusk) <a href="https://twitter.com/elonmusk/status/1050164164805545984?ref\_src=twsrc\%5Etfw">October 10, 2018</a></blockquote>  </figure>  <p>Silicon Valley-based Musk created the electric carmaker and aerospace company <a href="https://www.theguardian.com/science/spacex">SpaceX</a>, which has been <a href="https://www.theguardian.com/science/gallery/2018/oct/09/spacex-launches-and-lands-falcon-9-rocket-on-california-coast-in-pictures">launching</a> space rockets in an attempt to win commercial space transportation business in future – and realising Musk’s talk of colonising Mars.</p> <p><a href="https://www.theguardian.com/technology/tesla">Tesla</a> is wildly successful by many measures, but has been under pressure over profits, keeping up with demand and <a href="https://www.theguardian.com/technology/2018/sep/10/tesla-workers-union-elon-musk">conditions for workers</a>.</p> <p>Its shares have struggled amid the recent controversy, trading around \$259.52 at the time of writing, around a 27\% fall since Musk tweeted his intention to take the firm private at \$420 a share.</p> <aside class="element element-rich-link element--thumbnail"> <p> <span>Related: </span><a href="https://www.theguardian.com/technology/2018/oct/04/elon-musk-sec-twitter">Musk mocks 'Shortseller Enrichment Commission' after SEC settlement</a> </p> </aside>  <p>The SEC filed a suit accusing Musk of fraud in relation to the tweets, alleging his go-private plan, which he abandoned weeks later, had no basis in fact. The tech billionaire accepted a \$20m fine and agreed to step down as chairman as part of the settlement. </p> <p>Tesla and 21st Century Fox did not immediately respond to a request for comment.</p> <p>Shares of Tesla on Wednesday afternoon pared losses to trade down 1.6\% at \$258.50.</p>\\
\bottomrule
\end{tabular}}
\end{table}

\rowcolors{2}{white}{white}

\rowcolors{2}{gray!6}{white}

\begin{table}[!h]

\caption{\label{tab:data_collection}Table containing HEALTH  News}
\centering
\resizebox{\linewidth}{!}{
\begin{tabular}[t]{ll}
\hiderowcolors
\toprule
sectionName & body\\
\midrule
\showrowcolors
Society & <p>Babies and mothers died after a health regulator failed to act against midwives suspected of providing dangerously poor care, despite the police raising concerns about their conduct, a damning report has concluded.</p> <p>The Nursing and Midwifery Council (NMC) did nothing for two years about information supplied by Cumbria police on maternity staff at Furness general hospital, an inquiry found. </p> <p>The NMC’s failure to instigate disciplinary proceedings against midwives at the hospital allowed them to carry on practising when they represented a danger to pregnant women and their offspring, according to the Professional Standards Authority (PSA). </p> <aside class="element element-rich-link element--thumbnail"> <p> <span>Related: </span><a href="https://www.theguardian.com/politics/2018/mar/25/jeremy-hunt-to-unveil-plan-for-women-to-have-same-midwives-through-pregnancy">Hunt to unveil plan for women to have same midwives through pregnancy</a> </p> </aside>  <p>Up to 19 babies and mothers died at the hospital between 2004 and 2012 as a result of mistakes by staff of its maternity unit, in one of the biggest patient care scandals involving an NHS trust in England. </p> <p>A previous inquiry into the deaths concluded that 13 of the infants and women would have lived if they had received better care. The scandal involved six neonatal deaths, 10 stillbirths and three deaths of mothers at the hospital, which is one of five run by Morecambe Bay hospitals NHS foundation trust. </p> <p>The PSA’s review of the NMC’s handling of the scandal was commissioned by the health secretary, Jeremy Hunt. Cumbria police told the review: “We were really concerned that reports of the same midwives [of whom] we had the cases sitting in front of us were still practising at the hospital.”</p> <p>The PSA, which supervises medical regulators, found that the NMC was not prompted to do anything by the police reports, despite their seriousness. </p> <p>“In our view there was scope for the NMC to investigate the wider fitness to practise of the midwives concerned and the police expected them to do so at the time the information was sent,” the 80-page report says. “We saw no evidence that the NMC considered doing so. This was an opportunity missed, given that some of the midwives identified by the police were subsequently involved in adverse events at [Furness general hospital].”</p> <p>The NMC has admitted that its handling of cases involving midwives from the trust was unacceptable and has said it is “truly sorry”. Jackie Smith, the regulator’s chief executive since 2012, announced her resignation on Monday. </p> <p>Bill Kirkup, who led <a href="https://www.theguardian.com/society/2015/mar/03/furness-hospitals-lethal-mix-of-failings-led-to-deaths-of-12-report-says">the inquiry into the scandal that reported in 2015</a>, has criticised the NMC’s “lamentable failure” over its decision to clear two midwives of misconduct relating to the death in 2008 of Joshua Titcombe. He died nine days after his birth after midwives failed to spot he had a serious infection. </p> <p>In a scathing joint statement, Joshua’s father, James Titcombe, and two other bereaved parents, Liza Brady and Carl Hendrickson, said the report exposed “the truly shocking scale of the NMC’s failure to respond properly to the serious concerns and detailed information provided to them”. </p> <p>They said: “We were particularly horrified that even when Cumbria police directly raised significant issues, the NMC effectively ignored the information for almost two years. Whilst this was going on, serious incidents involving registrants [midwives] under investigation continued, meaning lives were undoubtedly put at risk. Avoidable tragedies continued to happen that could well have been prevented.”</p> <p>They also slated the NMC for being “defensive, legalistic and in some cases grossly misleading in their responses to families and others” and for its “culture of denial and reputational management”. </p> <p>The PSA’s report also reveals that the NMC mishandled its dealings with bereaved families, had very poor record-keeping and did not pass on to the PSA material that the latter’s inquiry team then found elsewhere. </p> <p>Smith, the outgoing chief executive, said: “Since 2014 we’ve made significant changes to improve the way we work and as the report recognises, we’re now a very different organisation. The changes we’ve made put vulnerable witnesses and families affected by failings in care at the heart of our work. But we know that there is much more to do.”</p>\\
Society & <p>There is a “shocking” divide in dental health standards between north and south and rich and poor, a new report says.</p> <p>The report by the Nuffield Trust and the Health Foundation found a “consistent gap” between the dental health of the rich and poor, with people from the most deprived backgrounds twice as likely to be admitted to hospital in need of dental work than those better off.</p> <p>The report, which analyses publicly available data on dental health outcomes, said that there was a pattern of evidence that dental health is better in the south and east than in the north of England. </p> <p>While the authors noted that dental health is improving in general, they added that without action to decrease inequalities, progress in dental health will come to a halt. “As a nation, our dental health is improving,” said Prof John Appleby, the Nuffield Trust’s director of research. “But it is shocking that your income or where you live can still determine your dental health.”</p> <aside class="element element-rich-link element--thumbnail"> <p> <span>Related: </span><a href="https://www.theguardian.com/society/2017/oct/31/rotten-state-childrens-teeth-england-under-10-hospital-treatment-tooth-decay">Rotten state of children's teeth in England exposed in hospital data</a> </p> </aside>  <p>The findings showed that 14\% of people from deprived backgrounds had been hospitalised in need of dental work, against 7\% of the better off. It also said that 18\% of parents with children eligible for free school meals found it difficult to find an NHS dentist in 2013, compared with 11\% of parents whose children were not.</p> <p>Tooth decay remains the number one cause of child hospital admissions in the country. Eighty-three per cent of five-year-olds in the richest regions of the country had healthy teeth, compared with 70\% in the poorest parts in 2014-15. </p> <p>The report said that dental charges had risen steadily since 2010, with costs rising by over 6\% in the past two years, over and above inflation, while the amount of money spent on NHS <a href="https://www.theguardian.com/society/dentists">dentistry</a> had been reduced by up to 15\% since 2010-11. It called for dentists to be more integrated in wider healthcare action, arguing that they were “perfectly placed” to help tackle problems linked to poor oral health such as obesity, excessive alcohol consumption and smoking. </p> <p>It also argued that the new NHS dental contract, currently under review, should include dentistry in its plans to tackle poor overall health because focusing on the wider determinants of poor dental health could help tackle these inequalities.</p> <p>Henrik Overgaard-Nielsen, chairman of general dental practice at the British Dental Association, criticised the government for supporting a contract that set limits on patient numbers and left the most disadvantaged people without access. He said: “These divides between north and south, rich and poor, expose the myth of universal access to NHS dentistry. We have a discredited system that funds dental care for barely half the population and the patients that lose out are all too often the ones that need us most.”</p> <p>A Department of Health spokeswoman said: “Improving oral health, particularly in children, is a key priority for this government, and we want everyone to be able to access an NHS dentist wherever they are. NHS England’s Starting Well programme is working in 13 areas across the country, specifically targeting children who are not visiting a dentist, to prevent poor oral health.”</p>\\
Opinion & <p>A study published this week shows that air pollution has an alarming effect <a href="https://www.theguardian.com/environment/2018/aug/27/air-pollution-causes-huge-reduction-in-intelligence-study-reveals">on our cognitive abilities</a>. Shocking as this conclusion is in itself, the report joins a long list of research linking toxic air to serious health problems, and demonstrates the devastating consequences air pollution has on people living in towns and cities across the UK.</p> <p>As a doctor working in A\&amp;E, I was shocked by the number of children I treated this summer whose health was harmed by air pollution. I saw children choking with asthma and struggling to breathe.</p>  <aside class="element element-pullquote element--supporting"> <blockquote> <p>It means holding the car companies pushing diesels on to our roads to account for their part</p> </blockquote> </aside>  <p>I feel I have a duty to speak up about issues that have serious impacts on people’s health.</p> <p>Air pollution is a public health crisis. Long-term exposure to diesel fumes can be carcinogenic, especially for children. Prolonged exposure to air pollution has also been known to cause asthma in otherwise healthy children, and to <a href="https://www.telegraph.co.uk/journalists/laura-donnelly/11953613/Air-pollution-stunting-childrens-lungs-study-finds.html">permanently stunt children’s lung growth</a> by up to 10\%. That can have lifelong health implications for children growing up in our cities.</p> <p>On days with heavy air pollution there are things you can do to protect yourself, including staying away from busy roads, avoiding heavy cardiovascular exercise, and keeping children inside to play. Crucially, avoid driving if you can, as it will only make the pollution problem worse – and you’ll be at least as exposed to pollution as if you were walking or cycling.</p> <aside class="element element-rich-link element--thumbnail"> <p> <span>Related: </span><a href="https://www.theguardian.com/environment/2018/aug/27/air-pollution-causes-huge-reduction-in-intelligence-study-reveals">Air pollution causes ‘huge’ reduction in intelligence, study reveals</a> </p> </aside>  <p>But ordinary people shouldn’t need to change their behaviour or keep their children inside to protect their lungs. Instead, we need to clean up our dirty air for good. Across the UK many of our cities have illegally high levels of air pollution. A Greenpeace Unearthed investigation found that <a href="https://unearthed.greenpeace.org/2017/04/04/air-pollution-nurseries/">more than 1,000 nurseries nationwide</a> are close to illegally polluted roads. No one should live in a polluted city, and government must take responsibility for tackling this health emergency.</p> <p>Diesel fuel is one of the biggest sources of pollution in many towns and cities across the UK. That means we need to introduce clean air zones to protect people in some of the worst polluted areas, and to move away from polluting diesel cars altogether. And it means holding the car companies pushing diesels on to our roads to account for their part in causing the UK’s air pollution crisis. This is why a group of health professionals joined Greenpeace last week to <a href="https://www.theguardian.com/environment/2018/aug/20/anti-pollution-activists-stage-protest-at-volkswagens-uk-headquarters">shut down the headquarters of VW</a>, the company selling the most diesel cars in the UK.</p> <aside class="element element-rich-link element--thumbnail"> <p> <span>Related: </span><a href="https://www.theguardian.com/environment/2018/aug/29/impact-air-pollution-environment-inequality-expert-community">Do you have expertise in the rising impact of air pollution?</a> </p> </aside>  <p>As well as this, our cities badly need better infrastructure for active travel, so that people can safely walk and cycle, and green, reliable and affordable public transport that means people can actually choose to leave their cars at home.</p> <p>As our knowledge of the link between air pollution and ill health grows, so does the weight of our responsibility towards those affected by it, especially the next generation. We won’t be able to tell them that we didn’t know, or that we didn’t have a solution. If we fail, it’ll be because we didn’t want to succeed.</p> <p>• Dr Guddi Singh is a paediatrician and a member of <a href="https://doctorsagainstdiesel.uk/">Doctors Against Diesel</a></p>\\
Cities & <p>Mbandaka, once prosperous and charming, with electricity and running water 24 hours a day, hasn’t moved on since the 1970s. Quite the opposite – this city, situated at the confluence of the Congo and Ruki rivers, has declined. Rainwater gnaws at the tarmac of its roads due to lack of drainage; houses collapse into the sodden ground of its estates; its socioeconomic life exudes poverty and misery, all the more glaring beside the affluent minority. Deadly malnutrition is creeping into many families. In late May, news that Ebola had <a href="https://www.theguardian.com/world/2018/may/17/drc-ebola-outbreak-in-new-phase-with-case-in-north-west-city-officials">reached Mbandaka</a> added an edge of fear to this dismal portrait. The latest outbreak had already killed nearly 30 across the Democratic Republic of the Congo (DRC), and reaching a city of 1.2 million threatened a serious escalation.<br></p>  <figure class="element element-embed" data-alt="Ebola management">  <iframe class="fenced" srcdoc="\&lt;html\&gt;\&lt;head\&gt;\&lt;/head\&gt;\&lt;body\&gt;\&lt;blockquote class=\&quot;instagram-media\&quot; data-instgrm-captioned data-instgrm-permalink=\&quot;https://www.instagram.com/p/BjcBth2grbA/\&quot; data-instgrm-version=\&quot;8\&quot; style=\&quot; background:\#FFF; border:0; border-radius:3px; box-shadow:0 0 1px 0 rgba(0,0,0,0.5),0 1px 10px 0 rgba(0,0,0,0.15); margin: 1px; max-width:658px; padding:0; width:99.375\%; width:-webkit-calc(100\% - 2px); width:calc(100\% - 2px);\&quot;\&gt;\&lt;div style=\&quot;padding:8px;\&quot;\&gt; \&lt;div style=\&quot; background:\#F8F8F8; line-height:0; margin-top:40px; padding:50.0\% 0; text-align:center; width:100\%;\&quot;\&gt; \&lt;div style=\&quot; background:url(data:image/png;base64,iVBORw0KGgoAAAANSUhEUgAAACwAAAAsCAMAAAApWqozAAAABGdBTUEAALGPC/xhBQAAAAFzUkdCAK7OHOkAAAAMUExURczMzPf399fX1+bm5mzY9AMAAADiSURBVDjLvZXbEsMgCES5/P8/t9FuRVCRmU73JWlzosgSIIZURCjo/ad+EQJJB4Hv8BFt+IDpQoCx1wjOSBFhh2XssxEIYn3ulI/6MNReE07UIWJEv8UEOWDS88LY97kqyTliJKKtuYBbruAyVh5wOHiXmpi5we58Ek028czwyuQdLKPG1Bkb4NnM+VeAnfHqn1k4+GPT6uGQcvu2h2OVuIf/gWUFyy8OWEpdyZSa3aVCqpVoVvzZZ2VTnn2wU8qzVjDDetO90GSy9mVLqtgYSy231MxrY6I2gGqjrTY0L8fxCxfCBbhWrsYYAAAAAElFTkSuQmCC); display:block; height:44px; margin:0 auto -44px; position:relative; top:-22px; width:44px;\&quot;\&gt;\&lt;/div\&gt;\&lt;/div\&gt; \&lt;p style=\&quot; margin:8px 0 0 0; padding:0 4px;\&quot;\&gt; \&lt;a href=\&quot;https://www.instagram.com/p/BjcBth2grbA/\&quot; style=\&quot; color:\#000; font-family:Arial,sans-serif; font-size:14px; font-style:normal; font-weight:normal; line-height:17px; text-decoration:none; word-wrap:break-word;\&quot; target=\&quot;\_blank\&quot;\&gt;@msf\_rdcongo renforce son intervention pour contenir l’épidémie d’Ebola dans la province d’Equateur. À Bikoro où l’on travaille depuis quelques semaines avec le Ministère de la Santé et d’autres partenaires, nos équipes ont commencé la vaccination d’essai contre Ebola. Les personnels de santé, souvent plus exposés au virus, sont les premiers concernés par cette vaccination. La deuxième étape consiste à identifier les membres de la famille, les voisins, les collègues et les amis des patients afin d’obtenir leur consentement pour être vaccinés. Cette vaccination est accompagnée d\&amp;\#39;autres mesures de protection pour contrôler l\&amp;\#39;épidémie. \_\_\_\_\_\_\_\_\_\_\_\_\_\_\_\_\_\_\_\_\_\_\_\_\_\_\_\_\_\_\_\_\_\_\_\_\_\_\_ \#MÉDICAL \#ONG \#médecinssansfrontières \#Ebola \#Ebolaoutbreak \#Equateur \#Bikoro \#Mbandaka \#RDC \#MSF\&lt;/a\&gt;\&lt;/p\&gt; \&lt;p style=\&quot; color:\#c9c8cd; font-family:Arial,sans-serif; font-size:14px; line-height:17px; margin-bottom:0; margin-top:8px; overflow:hidden; padding:8px 0 7px; text-align:center; text-overflow:ellipsis; white-space:nowrap;\&quot;\&gt;A post shared by \&lt;a href=\&quot;https://www.instagram.com/msf\_rdcongo/\&quot; style=\&quot; color:\#c9c8cd; font-family:Arial,sans-serif; font-size:14px; font-style:normal; font-weight:normal; line-height:17px;\&quot; target=\&quot;\_blank\&quot;\&gt; MSF RDCongo\&lt;/a\&gt; (@msf\_rdcongo) on \&lt;time style=\&quot; font-family:Arial,sans-serif; font-size:14px; line-height:17px;\&quot; datetime=\&quot;2018-05-31T10:08:44+00:00\&quot;\&gt;May 31, 2018 at 3:08am PDT\&lt;/time\&gt;\&lt;/p\&gt;\&lt;/div\&gt;\&lt;/blockquote\&gt; \&lt;script async defer src=\&quot;//www.instagram.com/embed.js\&quot;\&gt;\&lt;/script\&gt;\&lt;/body\&gt;\&lt;/html\&gt;"></iframe> </figure>  <h2>Panic management</h2> <p>The faithful pack bars and churches alike, talking of Ebola and of the provincial government that has done nothing since arriving in power five months ago. Others bemoan the terrible state of the Mbandaka-Bikoro road, which is currently obstructing efforts to reach sites over which the disease has cast a fatal shadow. This Ebola outbreak was born around Bikoro, a market town 130km to the south; the handful of cases in the city are people who have travelled from this territory, which explains why panic is relatively restrained here<strong tabindex="-1">. </strong></p> <p>On the other hand, the city has seen a massive influx of manpower to combat the disease – Africans, Europeans, Congolese doctors, lab assistants, nurses, researchers, awareness campaigners – and money is pouring in. The hotel signs all read: “No vacancies!” Experts and the authorities are working relentlessly to devise containment and vaccination strategies from the local government headquarters and the Iyonda reception centre, 15km from the city centre. Panic is receding, though the fightback against the spread of the outbreak continues on the ground.</p>  <figure class="element element-embed" data-alt="Equator sign">  <iframe class="fenced" srcdoc="\&lt;html\&gt;\&lt;head\&gt;\&lt;/head\&gt;\&lt;body\&gt;\&lt;blockquote class=\&quot;instagram-media\&quot; data-instgrm-captioned data-instgrm-permalink=\&quot;https://www.instagram.com/p/BgeGX\_GAkH8/\&quot; data-instgrm-version=\&quot;8\&quot; style=\&quot; background:\#FFF; border:0; border-radius:3px; box-shadow:0 0 1px 0 rgba(0,0,0,0.5),0 1px 10px 0 rgba(0,0,0,0.15); margin: 1px; max-width:658px; padding:0; width:99.375\%; width:-webkit-calc(100\% - 2px); width:calc(100\% - 2px);\&quot;\&gt;\&lt;div style=\&quot;padding:8px;\&quot;\&gt; \&lt;div style=\&quot; background:\#F8F8F8; line-height:0; margin-top:40px; padding:50.0\% 0; text-align:center; width:100\%;\&quot;\&gt; \&lt;div style=\&quot; background:url(data:image/png;base64,iVBORw0KGgoAAAANSUhEUgAAACwAAAAsCAMAAAApWqozAAAABGdBTUEAALGPC/xhBQAAAAFzUkdCAK7OHOkAAAAMUExURczMzPf399fX1+bm5mzY9AMAAADiSURBVDjLvZXbEsMgCES5/P8/t9FuRVCRmU73JWlzosgSIIZURCjo/ad+EQJJB4Hv8BFt+IDpQoCx1wjOSBFhh2XssxEIYn3ulI/6MNReE07UIWJEv8UEOWDS88LY97kqyTliJKKtuYBbruAyVh5wOHiXmpi5we58Ek028czwyuQdLKPG1Bkb4NnM+VeAnfHqn1k4+GPT6uGQcvu2h2OVuIf/gWUFyy8OWEpdyZSa3aVCqpVoVvzZZ2VTnn2wU8qzVjDDetO90GSy9mVLqtgYSy231MxrY6I2gGqjrTY0L8fxCxfCBbhWrsYYAAAAAElFTkSuQmCC); display:block; height:44px; margin:0 auto -44px; position:relative; top:-22px; width:44px;\&quot;\&gt;\&lt;/div\&gt;\&lt;/div\&gt; \&lt;p style=\&quot; margin:8px 0 0 0; padding:0 4px;\&quot;\&gt; \&lt;a href=\&quot;https://www.instagram.com/p/BgeGX\_GAkH8/\&quot; style=\&quot; color:\#000; font-family:Arial,sans-serif; font-size:14px; font-style:normal; font-weight:normal; line-height:17px; text-decoration:none; word-wrap:break-word;\&quot; target=\&quot;\_blank\&quot;\&gt;on the \#equator \#equatorrock \#mbandaka \#congo \#drc\&lt;/a\&gt;\&lt;/p\&gt; \&lt;p style=\&quot; color:\#c9c8cd; font-family:Arial,sans-serif; font-size:14px; line-height:17px; margin-bottom:0; margin-top:8px; overflow:hidden; padding:8px 0 7px; text-align:center; text-overflow:ellipsis; white-space:nowrap;\&quot;\&gt;A post shared by @\&lt;a href=\&quot;https://www.instagram.com/babasteve/\&quot; style=\&quot; color:\#c9c8cd; font-family:Arial,sans-serif; font-size:14px; font-style:normal; font-weight:normal; line-height:17px;\&quot; target=\&quot;\_blank\&quot;\&gt; babasteve\&lt;/a\&gt; on \&lt;time style=\&quot; font-family:Arial,sans-serif; font-size:14px; line-height:17px;\&quot; datetime=\&quot;2018-03-18T15:53:47+00:00\&quot;\&gt;Mar 18, 2018 at 8:53am PDT\&lt;/time\&gt;\&lt;/p\&gt;\&lt;/div\&gt;\&lt;/blockquote\&gt; \&lt;script async defer src=\&quot;//www.instagram.com/embed.js\&quot;\&gt;\&lt;/script\&gt;\&lt;/body\&gt;\&lt;/html\&gt;"></iframe> </figure>  <h2>Mbandaka in numbers</h2> <p><strong>460 sq km</strong> – Surface area of the city</p> <p><strong>4km – </strong>Distance between town hall and the equator</p> <p><strong>93.6\% – </strong>Percentage of the population living on less than \$1 a day</p> <p><strong>2\% –</strong> Percentage with access to drinkable water </p> <p><strong>370 – </strong>Hectare size of the <a href="http://www.bgci.org/worldwide/article/0037/">Eala botanical gardens</a>, founded in 1900 </p>  <figure class="element element-embed" data-alt="Eala">  <iframe class="fenced" srcdoc="\&lt;html\&gt;\&lt;head\&gt;\&lt;/head\&gt;\&lt;body\&gt;\&lt;blockquote class=\&quot;instagram-media\&quot; data-instgrm-captioned data-instgrm-permalink=\&quot;https://www.instagram.com/p/BVtmJDQAE64/\&quot; data-instgrm-version=\&quot;8\&quot; style=\&quot; background:\#FFF; border:0; border-radius:3px; box-shadow:0 0 1px 0 rgba(0,0,0,0.5),0 1px 10px 0 rgba(0,0,0,0.15); margin: 1px; max-width:658px; padding:0; width:99.375\%; width:-webkit-calc(100\% - 2px); width:calc(100\% - 2px);\&quot;\&gt;\&lt;div style=\&quot;padding:8px;\&quot;\&gt; \&lt;div style=\&quot; background:\#F8F8F8; line-height:0; margin-top:40px; padding:50\% 0; text-align:center; width:100\%;\&quot;\&gt; \&lt;div style=\&quot; background:url(data:image/png;base64,iVBORw0KGgoAAAANSUhEUgAAACwAAAAsCAMAAAApWqozAAAABGdBTUEAALGPC/xhBQAAAAFzUkdCAK7OHOkAAAAMUExURczMzPf399fX1+bm5mzY9AMAAADiSURBVDjLvZXbEsMgCES5/P8/t9FuRVCRmU73JWlzosgSIIZURCjo/ad+EQJJB4Hv8BFt+IDpQoCx1wjOSBFhh2XssxEIYn3ulI/6MNReE07UIWJEv8UEOWDS88LY97kqyTliJKKtuYBbruAyVh5wOHiXmpi5we58Ek028czwyuQdLKPG1Bkb4NnM+VeAnfHqn1k4+GPT6uGQcvu2h2OVuIf/gWUFyy8OWEpdyZSa3aVCqpVoVvzZZ2VTnn2wU8qzVjDDetO90GSy9mVLqtgYSy231MxrY6I2gGqjrTY0L8fxCxfCBbhWrsYYAAAAAElFTkSuQmCC); display:block; height:44px; margin:0 auto -44px; position:relative; top:-22px; width:44px;\&quot;\&gt;\&lt;/div\&gt;\&lt;/div\&gt; \&lt;p style=\&quot; margin:8px 0 0 0; padding:0 4px;\&quot;\&gt; \&lt;a href=\&quot;https://www.instagram.com/p/BVtmJDQAE64/\&quot; style=\&quot; color:\#000; font-family:Arial,sans-serif; font-size:14px; font-style:normal; font-weight:normal; line-height:17px; text-decoration:none; word-wrap:break-word;\&quot; target=\&quot;\_blank\&quot;\&gt;<U+0001F333>Jardin Botanique d\&amp;\#39;Eala <U+0001F333> Many species of trees found here were used to create to botanical garden in Kisantu <U+0001F1E8><U+0001F1E9> \#mbandaka \#drcongo \#africa \#congoriver \#botanicalgardens \#ealabotanicalgarden\&lt;/a\&gt;\&lt;/p\&gt; \&lt;p style=\&quot; color:\#c9c8cd; font-family:Arial,sans-serif; font-size:14px; line-height:17px; margin-bottom:0; margin-top:8px; overflow:hidden; padding:8px 0 7px; text-align:center; text-overflow:ellipsis; white-space:nowrap;\&quot;\&gt;A post shared by \&lt;a href=\&quot;https://www.instagram.com/bilou\_1708/\&quot; style=\&quot; color:\#c9c8cd; font-family:Arial,sans-serif; font-size:14px; font-style:normal; font-weight:normal; line-height:17px;\&quot; target=\&quot;\_blank\&quot;\&gt; Blain Valentin\&lt;/a\&gt; (@bilou\_1708) on \&lt;time style=\&quot; font-family:Arial,sans-serif; font-size:14px; line-height:17px;\&quot; datetime=\&quot;2017-06-24T06:34:38+00:00\&quot;\&gt;Jun 23, 2017 at 11:34pm PDT\&lt;/time\&gt;\&lt;/p\&gt;\&lt;/div\&gt;\&lt;/blockquote\&gt; \&lt;script async defer src=\&quot;//www.instagram.com/embed.js\&quot;\&gt;\&lt;/script\&gt;\&lt;/body\&gt;\&lt;/html\&gt;"></iframe> </figure>  <h2>History in 100 words</h2> <p>Mbandaka, founded by <a href="https://en.wikipedia.org/wiki/Henry\_Morton\_Stanley">Henry Morton Stanley</a> in 1883 as Équateurville, was born thanks to migration imposed by the Congo’s Belgian colonists, which displaced native people across the country. The majority of the city’s population are ethnically <a href="https://en.wikipedia.org/wiki/Mongo\_people">Mongo</a>, widely called “Ngele ea ntando”, or “from downstream”. In fact they largely came from upstream, travelling down the Tshuapa, Busira and Ruki rivers to end up in Mbandaka, then called Coquilhatville after the Belgian governor Camille-Aimé Coquilhat. They set themselves up largely in the Bokala neighbourhood against the river on the city’s east side. A separate stream of migrants – from the Eleku, Mpama, Nunu and other tribes – came down the river from Mampoko to occupy Bongondjo, Bolenge and other districts. The city, though it never became the Congolese capital it was once earmarked as, was an administrative centre from the late 19th century onwards.</p> <h2>Best building</h2> <p>Built to proclaim the DRC’s independence in June 1960, and hosting Mbandaka’s first black governor, Laurent Eketebi, the city’s public administration building is without equal across the country (apart from the one in Kananga): two wings, four floors, set in a picturesque garden of acacia and cypress.</p>  <figure class="element element-embed" data-alt="Buildings in Mbandaka">  <iframe class="fenced" srcdoc="\&lt;html\&gt;\&lt;head\&gt;\&lt;/head\&gt;\&lt;body\&gt;\&lt;blockquote class=\&quot;instagram-media\&quot; data-instgrm-captioned data-instgrm-permalink=\&quot;https://www.instagram.com/p/BKu7zRqho0z/\&quot; data-instgrm-version=\&quot;8\&quot; style=\&quot; background:\#FFF; border:0; border-radius:3px; box-shadow:0 0 1px 0 rgba(0,0,0,0.5),0 1px 10px 0 rgba(0,0,0,0.15); margin: 1px; max-width:658px; padding:0; width:99.375\%; width:-webkit-calc(100\% - 2px); width:calc(100\% - 2px);\&quot;\&gt;\&lt;div style=\&quot;padding:8px;\&quot;\&gt; \&lt;div style=\&quot; background:\#F8F8F8; line-height:0; margin-top:40px; padding:33.33333333333333\% 0; text-align:center; width:100\%;\&quot;\&gt; \&lt;div style=\&quot; background:url(data:image/png;base64,iVBORw0KGgoAAAANSUhEUgAAACwAAAAsCAMAAAApWqozAAAABGdBTUEAALGPC/xhBQAAAAFzUkdCAK7OHOkAAAAMUExURczMzPf399fX1+bm5mzY9AMAAADiSURBVDjLvZXbEsMgCES5/P8/t9FuRVCRmU73JWlzosgSIIZURCjo/ad+EQJJB4Hv8BFt+IDpQoCx1wjOSBFhh2XssxEIYn3ulI/6MNReE07UIWJEv8UEOWDS88LY97kqyTliJKKtuYBbruAyVh5wOHiXmpi5we58Ek028czwyuQdLKPG1Bkb4NnM+VeAnfHqn1k4+GPT6uGQcvu2h2OVuIf/gWUFyy8OWEpdyZSa3aVCqpVoVvzZZ2VTnn2wU8qzVjDDetO90GSy9mVLqtgYSy231MxrY6I2gGqjrTY0L8fxCxfCBbhWrsYYAAAAAElFTkSuQmCC); display:block; height:44px; margin:0 auto -44px; position:relative; top:-22px; width:44px;\&quot;\&gt;\&lt;/div\&gt;\&lt;/div\&gt; \&lt;p style=\&quot; margin:8px 0 0 0; padding:0 4px;\&quot;\&gt; \&lt;a href=\&quot;https://www.instagram.com/p/BKu7zRqho0z/\&quot; style=\&quot; color:\#000; font-family:Arial,sans-serif; font-size:14px; font-style:normal; font-weight:normal; line-height:17px; text-decoration:none; word-wrap:break-word;\&quot; target=\&quot;\_blank\&quot;\&gt;Room with a view on the Congo river.  Equateur, DR Congo. On assignment for @france24 and @Unicef  \#congoriver \#Mbandaka \#DRC \#equateur\&lt;/a\&gt;\&lt;/p\&gt; \&lt;p style=\&quot; color:\#c9c8cd; font-family:Arial,sans-serif; font-size:14px; line-height:17px; margin-bottom:0; margin-top:8px; overflow:hidden; padding:8px 0 7px; text-align:center; text-overflow:ellipsis; white-space:nowrap;\&quot;\&gt;A post shared by \&lt;a href=\&quot;https://www.instagram.com/thomas.nicolon/\&quot; style=\&quot; color:\#c9c8cd; font-family:Arial,sans-serif; font-size:14px; font-style:normal; font-weight:normal; line-height:17px;\&quot; target=\&quot;\_blank\&quot;\&gt; Thomas Nicolon\&lt;/a\&gt; (@thomas.nicolon) on \&lt;time style=\&quot; font-family:Arial,sans-serif; font-size:14px; line-height:17px;\&quot; datetime=\&quot;2016-09-24T09:17:48+00:00\&quot;\&gt;Sep 24, 2016 at 2:17am PDT\&lt;/time\&gt;\&lt;/p\&gt;\&lt;/div\&gt;\&lt;/blockquote\&gt; \&lt;script async defer src=\&quot;//www.instagram.com/embed.js\&quot;\&gt;\&lt;/script\&gt;\&lt;/body\&gt;\&lt;/html\&gt;"></iframe> </figure>  <h2>Mbandaka in sound and vision<br></h2>       <figure class="element element-video" data-canonical-url="http://www.youtube.com/watch?v=PEGNTQIRPsw"                                                                        >  <iframe height="259" width="460" src="https://www.youtube.com/embed/PEGNTQIRPsw?wmode=opaque\&feature=oembed" frameborder="0" allowfullscreen ></iframe>  </figure>   <p>Take a slideshow tour of interiors from Mbandaka’s colonial houses, with audio of those occupying them.</p>  <figure class="element element-embed" data-alt=".">  <iframe class="fenced" srcdoc="\&lt;html\&gt;\&lt;head\&gt;\&lt;/head\&gt;\&lt;body\&gt;\&lt;blockquote class=\&quot;instagram-media\&quot; data-instgrm-captioned data-instgrm-permalink=\&quot;https://www.instagram.com/p/hCDYz4F9GV/\&quot; data-instgrm-version=\&quot;8\&quot; style=\&quot; background:\#FFF; border:0; border-radius:3px; box-shadow:0 0 1px 0 rgba(0,0,0,0.5),0 1px 10px 0 rgba(0,0,0,0.15); margin: 1px; max-width:658px; padding:0; width:99.375\%; width:-webkit-calc(100\% - 2px); width:calc(100\% - 2px);\&quot;\&gt;\&lt;div style=\&quot;padding:8px;\&quot;\&gt; \&lt;div style=\&quot; background:\#F8F8F8; line-height:0; margin-top:40px; padding:50\% 0; text-align:center; width:100\%;\&quot;\&gt; \&lt;div style=\&quot; background:url(data:image/png;base64,iVBORw0KGgoAAAANSUhEUgAAACwAAAAsCAMAAAApWqozAAAABGdBTUEAALGPC/xhBQAAAAFzUkdCAK7OHOkAAAAMUExURczMzPf399fX1+bm5mzY9AMAAADiSURBVDjLvZXbEsMgCES5/P8/t9FuRVCRmU73JWlzosgSIIZURCjo/ad+EQJJB4Hv8BFt+IDpQoCx1wjOSBFhh2XssxEIYn3ulI/6MNReE07UIWJEv8UEOWDS88LY97kqyTliJKKtuYBbruAyVh5wOHiXmpi5we58Ek028czwyuQdLKPG1Bkb4NnM+VeAnfHqn1k4+GPT6uGQcvu2h2OVuIf/gWUFyy8OWEpdyZSa3aVCqpVoVvzZZ2VTnn2wU8qzVjDDetO90GSy9mVLqtgYSy231MxrY6I2gGqjrTY0L8fxCxfCBbhWrsYYAAAAAElFTkSuQmCC); display:block; height:44px; margin:0 auto -44px; position:relative; top:-22px; width:44px;\&quot;\&gt;\&lt;/div\&gt;\&lt;/div\&gt; \&lt;p style=\&quot; margin:8px 0 0 0; padding:0 4px;\&quot;\&gt; \&lt;a href=\&quot;https://www.instagram.com/p/hCDYz4F9GV/\&quot; style=\&quot; color:\#000; font-family:Arial,sans-serif; font-size:14px; font-style:normal; font-weight:normal; line-height:17px; text-decoration:none; word-wrap:break-word;\&quot; target=\&quot;\_blank\&quot;\&gt;Mbandaka avant la pluie \#congo \#equateur \#mbandaka \#sceneville\&lt;/a\&gt;\&lt;/p\&gt; \&lt;p style=\&quot; color:\#c9c8cd; font-family:Arial,sans-serif; font-size:14px; line-height:17px; margin-bottom:0; margin-top:8px; overflow:hidden; padding:8px 0 7px; text-align:center; text-overflow:ellipsis; white-space:nowrap;\&quot;\&gt;A post shared by \&lt;a href=\&quot;https://www.instagram.com/congo\_cd/\&quot; style=\&quot; color:\#c9c8cd; font-family:Arial,sans-serif; font-size:14px; font-style:normal; font-weight:normal; line-height:17px;\&quot; target=\&quot;\_blank\&quot;\&gt; Oliver Meisenberg\&lt;/a\&gt; (@congo\_cd) on \&lt;time style=\&quot; font-family:Arial,sans-serif; font-size:14px; line-height:17px;\&quot; datetime=\&quot;2013-11-22T21:31:07+00:00\&quot;\&gt;Nov 22, 2013 at 1:31pm PST\&lt;/time\&gt;\&lt;/p\&gt;\&lt;/div\&gt;\&lt;/blockquote\&gt; \&lt;script async defer src=\&quot;//www.instagram.com/embed.js\&quot;\&gt;\&lt;/script\&gt;\&lt;/body\&gt;\&lt;/html\&gt;"></iframe> </figure>  <h2>Biggest urban risk</h2> <p>There’s a tendency to blame Mbandaka’s decline on the political authorities, especially former president <a href="http://theconversation.com/from-mobutu-to-kabila-the-drc-is-paying-a-heavy-price-for-autocrats-at-its-helm-79455">Mobutu Sese Soko and his allies</a>. But the simple truth is that everyone around him busy enriching themselves invested in other provinces without looking out for Mbandaka. Meanwhile, locals put their time into minor trading activities, whereas people in Kivu and Bandundu made big real-estate investments. Lack of electricity is the other reason for Mbandaka’s stunted development. Because of the north-south conflicts in DRC, Mobutu didn’t have the option of hooking the city to the grid until the arrival of electricity in Bandundu in 1993. Mbandaka’s people are still waiting.</p> <h2>Grand designs</h2> <p>Colonial urban planning was excellent: first-rate building lots with well-thought-out layouts that avoided marshy ground. But this legacy hasn’t been respected since: for the last 20 years, we’ve seen the parcelling out of state land. Private villas are sprouting like mushrooms, without planning controls, though many are undeniably beautiful buildings. More than a thousand plots are affected, on which people are even starting to squabble over access roads. The local council has just established an inquiry into the free-for-all, and evictions and demolitions may follow.</p>  <figure class="element element-embed" data-alt=".">  <iframe class="fenced" srcdoc="\&lt;html\&gt;\&lt;head\&gt;\&lt;/head\&gt;\&lt;body\&gt;\&lt;blockquote class=\&quot;instagram-media\&quot; data-instgrm-captioned data-instgrm-permalink=\&quot;https://www.instagram.com/p/xRG3mIgRY2/\&quot; data-instgrm-version=\&quot;8\&quot; style=\&quot; background:\#FFF; border:0; border-radius:3px; box-shadow:0 0 1px 0 rgba(0,0,0,0.5),0 1px 10px 0 rgba(0,0,0,0.15); margin: 1px; max-width:658px; padding:0; width:99.375\%; width:-webkit-calc(100\% - 2px); width:calc(100\% - 2px);\&quot;\&gt;\&lt;div style=\&quot;padding:8px;\&quot;\&gt; \&lt;div style=\&quot; background:\#F8F8F8; line-height:0; margin-top:40px; padding:50\% 0; text-align:center; width:100\%;\&quot;\&gt; \&lt;div style=\&quot; background:url(data:image/png;base64,iVBORw0KGgoAAAANSUhEUgAAACwAAAAsCAMAAAApWqozAAAABGdBTUEAALGPC/xhBQAAAAFzUkdCAK7OHOkAAAAMUExURczMzPf399fX1+bm5mzY9AMAAADiSURBVDjLvZXbEsMgCES5/P8/t9FuRVCRmU73JWlzosgSIIZURCjo/ad+EQJJB4Hv8BFt+IDpQoCx1wjOSBFhh2XssxEIYn3ulI/6MNReE07UIWJEv8UEOWDS88LY97kqyTliJKKtuYBbruAyVh5wOHiXmpi5we58Ek028czwyuQdLKPG1Bkb4NnM+VeAnfHqn1k4+GPT6uGQcvu2h2OVuIf/gWUFyy8OWEpdyZSa3aVCqpVoVvzZZ2VTnn2wU8qzVjDDetO90GSy9mVLqtgYSy231MxrY6I2gGqjrTY0L8fxCxfCBbhWrsYYAAAAAElFTkSuQmCC); display:block; height:44px; margin:0 auto -44px; position:relative; top:-22px; width:44px;\&quot;\&gt;\&lt;/div\&gt;\&lt;/div\&gt; \&lt;p style=\&quot; margin:8px 0 0 0; padding:0 4px;\&quot;\&gt; \&lt;a href=\&quot;https://www.instagram.com/p/xRG3mIgRY2/\&quot; style=\&quot; color:\#000; font-family:Arial,sans-serif; font-size:14px; font-style:normal; font-weight:normal; line-height:17px; text-decoration:none; word-wrap:break-word;\&quot; target=\&quot;\_blank\&quot;\&gt;Sunset barge \#Mbandaka \#DRC \#DRCongo \#FleuveCongo\&lt;/a\&gt;\&lt;/p\&gt; \&lt;p style=\&quot; color:\#c9c8cd; font-family:Arial,sans-serif; font-size:14px; line-height:17px; margin-bottom:0; margin-top:8px; overflow:hidden; padding:8px 0 7px; text-align:center; text-overflow:ellipsis; white-space:nowrap;\&quot;\&gt;A post shared by \&lt;a href=\&quot;https://www.instagram.com/whuslim/\&quot; style=\&quot; color:\#c9c8cd; font-family:Arial,sans-serif; font-size:14px; font-style:normal; font-weight:normal; line-height:17px;\&quot; target=\&quot;\_blank\&quot;\&gt; Wesley Mollentze\&lt;/a\&gt; (@whuslim) on \&lt;time style=\&quot; font-family:Arial,sans-serif; font-size:14px; line-height:17px;\&quot; datetime=\&quot;2014-12-31T10:12:26+00:00\&quot;\&gt;Dec 31, 2014 at 2:12am PST\&lt;/time\&gt;\&lt;/p\&gt;\&lt;/div\&gt;\&lt;/blockquote\&gt; \&lt;script async defer src=\&quot;//www.instagram.com/embed.js\&quot;\&gt;\&lt;/script\&gt;\&lt;/body\&gt;\&lt;/html\&gt;"></iframe> </figure>  <h2>What’s next for the city?</h2> <p>The future isn’t necessarily dark. Mbandaka might rise again if the locals can join forces with investors who know how to take risks and set up processing units such as sawmills and cocoa-bean-roasting, fish-salting and livestock-feed facilities, as well as developing poultry farms instead of waiting for eggs to arrive from Kinshasa. The Eala botanical gardens are a wonder with great tourist potential, and let’s not forget the little islands and fishing camps that decorate the many branches of the Congo river. But the city’s first priority is warehouses and floating storage for holding and dispatching goods. This might encourage traders to stop in Mbandaka and stock up there rather thanheading elsewhere.</p> <h2>Close zoom</h2> <p>On the ground in Mbandaka in late May, <a href="https://www.economist.com/middle-east-and-africa/2018/05/24/the-ebola-outbreak-in-congo-can-probably-be-contained">the Economist painted an optimistic picture</a> of a country getting to grips with Ebola. Take a trip from the<a href="http://lab.pulitzercenter.org/on-the-congo/"> city along the Congo in a <em>pirogue</em></a>, a boat carved from the Tola tree, in this photo essay.</p> <ul> <li>Peter Gbiako is director of Radio Mwana in Mbandaka.<br></li> </ul> <p><em>Do you live in </em><em>Mbandaka</em><em>? What key facts, figures and cultural highlights have we missed? Share your stories below</em></p> <p><em>Follow Guardian Cities on <a href="https://twitter.com/guardiancities">Twitter</a>, <a href="https://www.facebook.com/guardiancities">Facebook</a> and <a href="https://www.instagram.com/guardiancities/?hl=en">Instagram</a> to join the discussion, and </em><em><a href="https://www.theguardian.com/cities">explore our archive here</a></em></p>\\
Opinion & <p>A new Fitbit wristband called <a href="http://uk.businessinsider.com/fitbit-ace-kids-fitness-tracker-features-release-date-photos-2018-3?r=US\&amp;IR=T" title="">Fitbit Ace</a> has been launched for children over eight. It will feature “reminders” for them to get active, undertake family step-challenges and also monitor sleep patterns. Thankfully, the calorie-counting device found on normal Fitbits has been disabled, but is that enough?</p> <p>Bearing in mind the UK’s child obesity problem – according to figures from NHS Direct, <a href="https://www.theguardian.com/politics/2016/aug/20/will-the-governments-new-childhood-obesity-strategy-have-any-effect" title="">a third of children</a> between two and 15 are overweight or obese – some people may feel that the Fitbit Ace could only be a boon. However, that’s debatable. Children are already bombarded with harmful messages about body image. Overweight kids are teased. Normal-sized girls feel that they should be on strict diets. Even young boys are succumbing to anorexia. Do children need what amounts to a “fat-shaming toy”?</p> <p>While many teenagers have owned Fitbits, the difference was that they were officially perceived as adult products, set apart from the child’s world. Now there’s this designated child-fitness “toy” pumping out information that’s arguably too complex for a young mind to properly process.</p> <p>While the Fitbit Ace is monitoring physical progress, who is monitoring the child’s reactions and emotions? After all, motivating children to get fit is just one important aspect; another would be stopping them from going too far with what may feel like a new playground craze.</p> <p>What may seem like a harmless motivational gimmick could lead an immature mind into disturbing, obsessive behaviour. <a href="https://www.theguardian.com/lifeandstyle/2015/sep/26/orthorexia-eating-disorder-clean-eating-dsm-miracle-foods" title="">Orthorexia</a> (the type of anorexia that masks itself as a health and fitness obsession) is a very real issue that should not be encouraged in anybody, never mind children. The good news is that there are already devices widely available to encourage most children to be healthier and fitter – they’re called parents.</p> <p>• Barbara Ellen is an <em>Observer</em> columnist</p>\\
US news & <div id="block-59540ec9e4b0b0b640125189" class="block is-key-event" data-block-contributor=""> <p class="block-time published-time"> <time datetime="2017-06-28T20:57:28.051Z">9.57pm <span class="timezone">BST</span></time> </p>   <h2 class="block-title">1. 'Great, great surprise’</h2>  <div class="block-elements">  <p>Donald Trump predicted victory for Republicans trying to salvage healthcare legislation in the Senate, and the president pushed back at the notion that he has no idea what’s in the bill and is not capable of assessing it.</p> <p><a href="https://www.theguardian.com/us-news/2017/jun/28/womens-healthcare-republican-senate-bill">Why are 13 men in charge?</a></p>  <figure class="element element-image" data-media-id="0a8e8bab6d50f11d9cfa4ee8fea1d45842590695"> <img src="https://media.guim.co.uk/0a8e8bab6d50f11d9cfa4ee8fea1d45842590695/0\_3\_3500\_2101/1000.jpg" alt="Guess which hand." width="1000" height="600" class="gu-image" /> <figcaption> <span class="element-image\_\_caption">Guess which hand.</span> <span class="element-image\_\_credit">Photograph: Yuri Gripas/Reuters</span> </figcaption> </figure> </div>   </div> <div id="block-59540de8e4b021a5988a7218" class="block is-summary" data-block-contributor=""> <p class="block-time published-time"> <time datetime="2017-06-28T20:57:16.829Z">9.57pm <span class="timezone">BST</span></time> </p>   <h2 class="block-title">2. Incredibly unpopular</h2>  <div class="block-elements">  <p>Three polls gauged public approval of the Senate bill. </p> <p>The results:</p> <p>USA Today/Suffolk: 12\%</p> <p>Marist: 17\%</p> <p>Quinnipiac: 15\%</p> <p>Average approval: 14.67\%</p>  <figure class="element element-image element--thumbnail" data-media-id="329ab50b013bcb54960d3e090145ea7772774d60"> <img src="https://media.guim.co.uk/329ab50b013bcb54960d3e090145ea7772774d60/0\_56\_3533\_2119/1000.jpg" alt="Arrested on Capitol Hill." width="1000" height="600" class="gu-image" /> <figcaption> <span class="element-image\_\_caption">Arrested on Capitol Hill.</span> <span class="element-image\_\_credit">Photograph: Drew Angerer/Getty Images</span> </figcaption> </figure> </div>   </div> <div id="block-59540dcbe4b0b0b640125184" class="block is-summary" data-block-contributor=""> <p class="block-time published-time"> <time datetime="2017-06-28T20:57:04.137Z">9.57pm <span class="timezone">BST</span></time> </p>   <h2 class="block-title">'It's gonna be great'</h2>  <div class="block-elements">  <blockquote class="quoted"> <p>Healthcare is working along very well … I think you’re gonna have a great, great surprise. It’s gonna be great.</p> </blockquote> <p>– Donald Trump</p> </div>   </div> <div id="block-59540db7e4b091f4f12bf49e" class="block is-summary" data-block-contributor=""> <p class="block-time published-time"> <time datetime="2017-06-28T20:56:54.752Z">9.56pm <span class="timezone">BST</span></time> </p>   <h2 class="block-title">McConnell twists arms</h2>  <div class="block-elements">  <p>Senate majority leader Mitch McConnell was observed greeting at least five senators in his office who had said no to the original version of the Senate healthcare bill. Will he find the votes?</p> </div>   </div> <div id="block-59540dabe4b0b0b640125183" class="block is-summary" data-block-contributor=""> <p class="block-time published-time"> <time datetime="2017-06-28T20:56:44.249Z">9.56pm <span class="timezone">BST</span></time> </p>   <h2 class="block-title">Does not</h2>  <div class="block-elements">  <figure class="element element-tweet" data-canonical-url="https://twitter.com/paulkrugman/status/880099762900529153">  <blockquote class="twitter-tweet"><p lang="en" dir="ltr">It\&\#39;s totally plausible that Trump has no idea that the health bill is the opposite of everything he promised voters. Who would tell him?</p>\&mdash; Paul Krugman (@paulkrugman) <a href="https://twitter.com/paulkrugman/status/880099762900529153">June 28, 2017</a></blockquote>  </figure> </div>   </div> <div id="block-59540d9ce4b091f4f12bf49c" class="block is-summary" data-block-contributor=""> <p class="block-time published-time"> <time datetime="2017-06-28T20:56:38.334Z">9.56pm <span class="timezone">BST</span></time> </p>   <h2 class="block-title">Does so</h2>  <div class="block-elements">  <figure class="element element-tweet" data-canonical-url="https://twitter.com/realDonaldTrump/status/880017678978736129">  <blockquote class="twitter-tweet"><p lang="en" dir="ltr">Some of the Fake News Media likes to say that I am not totally engaged in healthcare. Wrong, I know the subject well \&amp; want victory for U.S.</p>\&mdash; Donald J. Trump (@realDonaldTrump) <a href="https://twitter.com/realDonaldTrump/status/880017678978736129">June 28, 2017</a></blockquote>  </figure> </div>   </div> <div id="block-59540d8ae4b0b0b640125182" class="block is-summary" data-block-contributor=""> <p class="block-time published-time"> <time datetime="2017-06-28T20:56:29.254Z">9.56pm <span class="timezone">BST</span></time> </p>   <h2 class="block-title">Essential health benefits and you</h2>  <div class="block-elements">  <figure class="element element-tweet" data-canonical-url="https://twitter.com/daveweigel/status/880130715400142852">  <blockquote class="twitter-tweet"><p lang="en" dir="ltr">Sen. Cassidy, pushing back on the idea that maternity care can be dropped from EHB: \&quot;Women can\&\#39;t get pregnant without sperm.\&quot;</p>\&mdash; Dave Weigel (@daveweigel) <a href="https://twitter.com/daveweigel/status/880130715400142852">June 28, 2017</a></blockquote>  </figure> </div>   </div> <div id="block-59540d29e4b091f4f12bf499" class="block is-summary" data-block-contributor=""> <p class="block-time published-time"> <time datetime="2017-06-28T20:56:16.060Z">9.56pm <span class="timezone">BST</span></time> </p>   <h2 class="block-title">3. Trump to hit \$35,000-a-plate fundraiser</h2>  <div class="block-elements">  <p>The president was to attend a fundraiser for his 2020 re-election effort on Wednesday night. Dinner tickets cost \$35,000 apiece. The event was to be held at Trump’s Washington DC hotel. </p> <p><a href="https://twitter.com/jeffzeleny/status/880147200726192128">Check out an invite</a></p>  <figure class="element element-image element--thumbnail" data-media-id="be66d0e93d5e5c9301032b1ac332302835daa224"> <img src="https://media.guim.co.uk/be66d0e93d5e5c9301032b1ac332302835daa224/0\_0\_5277\_3166/1000.jpg" alt="No parking unless you have 35K." width="1000" height="600" class="gu-image" /> <figcaption> <span class="element-image\_\_caption">No parking unless you have 35K.</span> <span class="element-image\_\_credit">Photograph: J David Ake/AP</span> </figcaption> </figure> </div>   </div> <div id="block-59540d18e4b021a5988a7216" class="block is-key-event" data-block-contributor=""> <p class="block-time published-time"> <time datetime="2017-06-28T20:55:55.986Z">9.55pm <span class="timezone">BST</span></time> </p>   <h2 class="block-title">4. Support the Guardian</h2>  <div class="block-elements">  <p>It might take just a minute to catch up on the latest politics news. But good journalism takes time and costs money. If you like the Guardian’s politics coverage, please make a contribution. Thanks for reading!</p> <p><a href="https://contribute.theguardian.com/uk?countryGroup=us\&amp;INTCMP=gdnwb\_copts\_editorial\_memco\_theminute\_copy">Make a contribution today</a></p> </div>   </div> <div id="block-59540cbee4b0b0b64012517c" class="block is-key-event" data-block-contributor=""> <p class="block-time published-time"> <time datetime="2017-06-28T20:55:50.225Z">9.55pm <span class="timezone">BST</span></time> </p>   <h2 class="block-title">5. States investigate Trump lawyer's nonprofit</h2>  <div class="block-elements">  <p>Authorities in two states are looking into a nonprofit led by Trump attorney Jay Sekulow, after the Guardian reported it had steered tens of millions of dollars to the attorney, his family and their businesses.</p> <p><a href="https://www.theguardian.com/us-news/2017/jun/27/trump-lawyer-jay-sekulow-obamacare-repeal-christian-nonprofit">Read Jon Swaine’s coverage</a></p>  <figure class="element element-image" data-media-id="bfd31e522b394df1307f680631ad009fbfd5059c"> <img src="https://media.guim.co.uk/bfd31e522b394df1307f680631ad009fbfd5059c/342\_25\_2658\_1596/1000.jpg" alt="Evangelist." width="1000" height="600" class="gu-image" /> <figcaption> <span class="element-image\_\_caption">Evangelist.</span> <span class="element-image\_\_credit">Photograph: Chicago Tribune/MCT via Getty Images</span> </figcaption> </figure> </div>   </div> <div id="block-59540ca8e4b091f4f12bf494" class="block is-summary" data-block-contributor=""> <p class="block-time published-time"> <time datetime="2017-06-28T20:55:37.105Z">9.55pm <span class="timezone">BST</span></time> </p>   <h2 class="block-title">6. Secret Spice strikes again</h2>  <div class="block-elements">  <figure class="element element-tweet" data-canonical-url="https://twitter.com/brianstelter/status/880117404495425536">  <blockquote class="twitter-tweet"><p lang="en" dir="ltr">Stop me if you\&\#39;ve heard this one before-- but the <a href="https://twitter.com/WhiteHouse">@WhiteHouse</a> is prohibiting live video and audio of today\&\#39;s press briefing.</p>\&mdash; Brian Stelter (@brianstelter) <a href="https://twitter.com/brianstelter/status/880117404495425536">June 28, 2017</a></blockquote>  </figure> </div>   </div> <div id="block-59540c93e4b021a5988a7214" class="block is-summary" data-block-contributor=""> <p class="block-time published-time"> <time datetime="2017-06-28T20:54:59.503Z">9.54pm <span class="timezone">BST</span></time> </p>   <h2 class="block-title">... and another thing:</h2>  <div class="block-elements">  <figure class="element element-tweet" data-canonical-url="https://twitter.com/\_colleenmurphy\_/status/880107537324019713">  <blockquote class="twitter-tweet"><p lang="en" dir="ltr">Folks, <a href="https://t.co/FzxiBgY8ma">pic.twitter.com/FzxiBgY8ma</a></p>\&mdash; Colleen Murphy (@\_colleenmurphy\_) <a href="https://twitter.com/\_colleenmurphy\_/status/880107537324019713">June 28, 2017</a></blockquote>  </figure> </div>   </div> <div id="block-59540c7fe4b021a5988a7213" class="block is-summary" data-block-contributor=""> <p class="block-time published-time"> <time datetime="2017-06-28T20:54:52.283Z">9.54pm <span class="timezone">BST</span></time> </p>   <h2 class="block-title">... and another thing:</h2>  <div class="block-elements">  <figure class="element element-tweet" data-canonical-url="https://twitter.com/politico/status/880138505690767360">  <blockquote class="twitter-tweet"><p lang="en" dir="ltr"><U+0001F4F7> The Obamas are on vacation in Indonesia. Here are 13 photos from their family trip: <a href="https://t.co/LuToQ9nJDe">https://t.co/LuToQ9nJDe</a> <a href="https://t.co/BFUMN7h35R">pic.twitter.com/BFUMN7h35R</a></p>\&mdash; POLITICO (@politico) <a href="https://twitter.com/politico/status/880138505690767360">June 28, 2017</a></blockquote>  </figure> </div>   </div> <div id="block-59540c64e4b091f4f12bf48f" class="block is-summary" data-block-contributor=""> <p class="block-time published-time"> <time datetime="2017-06-28T20:54:45.369Z">9.54pm <span class="timezone">BST</span></time> </p>   <h2 class="block-title">... and another thing:</h2>  <div class="block-elements">  <figure class="element element-tweet" data-canonical-url="https://twitter.com/Bencjacobs/status/880124731579666432">  <blockquote class="twitter-tweet"><p lang="en" dir="ltr">Thomas Homan, head of ICE, on separating immigrant families, \&quot;families of US citizens get separated every day\&quot;</p>\&mdash; Ben Jacobs (@Bencjacobs) <a href="https://twitter.com/Bencjacobs/status/880124731579666432">June 28, 2017</a></blockquote>  </figure> </div>   </div>\\
\bottomrule
\end{tabular}}
\end{table}

\rowcolors{2}{white}{white}

\rowcolors{2}{gray!6}{white}

\begin{table}[!h]

\caption{\label{tab:data_collection}Table containing ART  News}
\centering
\resizebox{\linewidth}{!}{
\begin{tabular}[t]{ll}
\hiderowcolors
\toprule
sectionName & body\\
\midrule
\showrowcolors
Football & <p>Sometimes straightforward virtues are the best. In a Premier League that at times seems to have all but given up anything resembling traditional defending, there was something almost comforting about a clash between two sides who play in such a familiar, unpretentious way. This was a reminder of simpler virtues, a world in which the greatest aspiration is to be compact, and produced a sort of mutually assured self-neutralisation, a game in which flair was all but absent and, where it did exist, confined to a tiny sliver on the flanks. That the one goal came from a set piece was entirely appropriate.</p> <p><a href="https://www.theguardian.com/football/2017/sep/24/brighton-hove-albion-newcastle-united-premier-league-match-report" title="">The free-kick that produced the goal</a> five minutes into the second half was in part a result of the Brighton left-back Markus Suttner pushing forward and linking with Tomer Hemed on the left, which always looked the most likely source of a breakthrough for Brighton. There seemed a fairly clear plan from the start to isolate DeAndre Yedlin, the Newcastle right-back, against Solly March. It was the 23-year-old’s cross, after Newcastle had been opened up by a burst from Anthony Knockaert, that led to the Pascal Gross shot that cannoned to safety off Knockaert and then his cross-shot, cutting inside, that drew an awkward sprawling save from Rob Elliot.</p> <aside class="element element-rich-link"> <p> <span>Related: </span><a href="https://www.theguardian.com/football/2017/sep/24/brighton-hove-albion-newcastle-united-premier-league-match-report">Tomer Hemed’s strike sinks Newcastle as Brighton seal much-needed win</a> </p> </aside>  <p>In a tight first half, in which both sides were predictably compact and fairly narrow, most of Newcastle’s best openings similarly came down their left, Christian Atsu drifting infield to create chances for both Joselu and Ayoze Pérez. That, perhaps, should not come as too much of a surprise: both Chris Hughton and Rafa Benítez have a clearly defined way of playing that seeks, first and foremost, to maintain discipline and shape. Hughton acknowledged the importance of “solid platforms”.</p> <p>Only eight players – four from each side – started here who had started the Championship meeting between the clubs at the end of February, when Newcastle rather fortuitously came from behind to win with two late goals, yet the shape of the game was very similar.</p> <p>From the home side there was a 4-4-2 or 4-4-1-1, from the away team a 4-2-3-1, both seeking to deny the opposition space, both looking to their left-winger for creativity. Knockaert also had his moments but the organisation of both sides was such that, unless a player beat a couple of opponents, there was no space and the result was a game that flickered without ever quite catching light. Pérez, who might have been able to open the game up, was to all intents starved of possession and so his influence was limited throughout.</p> <p>But Newcastle are not quite a clone of last season, the change in their approach exemplified by the fact that Jonjo Shelvey has not been able to force his way back into the side after being sent off for treading on Dele Alli’s ankle against Tottenham on the opening weekend of the season. Shelvey was the creative fulcrum last season, his capacity to weight balls over the top for Dwight Gayle to run on to a key feature of the promotion campaign. But with Joselu replacing Gayle as the out-and-out striker, there has been far less need for that type of ball.</p> <p>Joselu may not be the most clinical finisher but he is an all-round striker and he is certainly not as quick as Gayle. But he is far more adept at holding the ball up and playing with his back to goal, giving Newcastle a more varied approach.</p> <p>That is the theory anyway but tactical schemes rarely exist in abstract isolation from other considerations and the fact is that Mikel Merino, the Spanish 21-year-old on loan from Borussia Dortmund, is a more reliable figure than Shelvey and probably just a better footballer. Nobody on the pitch attempted more passes than him and nobody had a better pass completion rate.</p> <p>Shelvey did come on midway through the second half, replacing Isaac Hayden, with Gayle joining him three minutes later, but by then Brighton had bunkered down, sitting deep so there was no space behind the back four for Gayle to exploit. He was such a peripheral figure that he had only two touches in the 20 minutes he was on the pitch. Partnering Shelvey and Merino, perhaps, might be a way of bringing a little more creativity to the midfield, although the risk is a greater openness and less protection for the back four.</p> <p>Benítez acknowledged it had been a “frustrating” afternoon but it was not particularly different from the games at Huddersfield and Swansea. Both of those matches were tight and hard-fought, lacking much in the way of flow and decided by a single goal. Brighton, similarly, have not conceded more than two in any game since March. Compact, solid, narrow: both sides will have a lot more matches this season settled by a single set play.</p>\\
Art and design & <p>“OK, so you are looking at The Persistence of Memory by Salvador Dalí, arguably his most famous and recognisable work,” says Tony Ellwood, director of the National Gallery of Victoria (NGV).</p> <p>We are only about one-third of the way through a guided tour of the new Winter Masterpieces exhibition, MoMA at NGV, but with promise of “the Dalí” the crowd packs in a bit tighter, gets a bit quieter and grows a bit more excited.</p> <p>The painting is immediately recognisable – it’s the one with all the melting clocks. I’ve seen prints of it on share-house walls, on tea towels, as the cover picture of Facebook events, even replicated in 3D as functioning clocks that drape over shelves at 90-degree angles. </p> <ul> <li><a href="https://www.theguardian.com/info/2015/dec/08/daily-email-au">Sign up to receive the top stories in Australia every day at noon</a><br></li> </ul>  <figure class="element element-embed" data-alt="Sign up to receive the top stories in Australia every day at noon">  <iframe src="https://www.theguardian.com/email/form/plaintone/4150" height="52" data-form-title="Sign up for The Guardian Today - Australia edition" data-form-description="Get our daily email of editors' picks and the biggest headlines each afternoon." data-form-campaign-code="AU\_signup\_page" scrolling="no" seamless frameborder="0" class="iframed--overflow-hidden email-sub\_\_iframe js-email-sub\_\_iframe js-email-sub\_\_iframe--article" data-form-success-desc="We will send you our picks of the most important headlines tomorrow morning."></iframe> </figure>  <p>“It’s one of those images that you think you know because it’s been reproduced so often,” says Glenn Lowry, director of the Museum of Modern Art (MoMA) in New York. “So you have the idea that actually it’s quite large.”</p> <p>In reality, the painting is small – much smaller than I would have thought. But the surprise of its size doesn’t change the impact of seeing it in person. The colours are vibrant, more so than any reproduction has been able to depict, and despite the image being so very familiar, there’s still something remarkable about seeing it in real life. It feels new, different. </p>  <figure class="element element-image" data-media-id="18904a0de6a1655e418a447c55f1cefd233613e9"> <img src="https://media.guim.co.uk/18904a0de6a1655e418a447c55f1cefd233613e9/0\_0\_2048\_1365/1000.jpg" alt="Moma at the NGV" width="1000" height="667" class="gu-image" /> <figcaption> <span class="element-image\_\_caption">‘Despite the image being so very familiar, there’s still something remarkable about seeing it in real life.’</span> <span class="element-image\_\_credit">Photograph: Tom Ross</span> </figcaption> </figure>  <p>This is a recurring theme throughout the exhibition, made up of over 230 works on loan from MoMA. You are brought face to face with familiar scenes and shapes, and repeatedly have your expectations challenged.<br></p> <p>It is a remarkable achievement for the NGV – for many of the pieces, this is the first time they’ve left the gallery. And there are an impressive amount of “big names” on show. Kahlo. Mondrian. Matisse. Van Gogh. Dalí. These are the A-listers of the art world – and this is a chance to “meet” your idols.</p> <aside class="element element-rich-link element--thumbnail"> <p> <span>Related: </span><a href="https://www.theguardian.com/artanddesign/2018/jun/11/outdoor-public-art-across-us-marilyn-monroe-snowmen">A hot dog bus and a snowman in June: the best in US public art this summer</a> </p> </aside>  <p>The works are divided up into eight separate rooms, which progress roughly chronologically and are loosely grouped by theme. It is the first time the NGV has dedicated its entire ground floor to one exhibition. </p> <p>Each room has its “rock-star” piece, the one that people flock to first before stepping back and taking in the rest. In one room a man to my right hisses “It’s a Pollock” in excitement. “Go look!” In another I’m so distracted by Frida Kahlo’s Self-Portrait With Cropped Hair (1940), I completely miss a Picasso. </p> <p>In an era where it is easier than ever to replicate art, where many of the things we are seeing in this gallery are things we’ve seen before, there’s an inherent challenge for these works to live up to their reputations. But at the same time, that is one of the main reasons people will come to see them – to see how the legend stacks up to reality. </p>  <figure class="element element-image" data-media-id="a3c2d38237302bb459e50709434a159c582b45eb"> <img src="https://media.guim.co.uk/a3c2d38237302bb459e50709434a159c582b45eb/0\_277\_4518\_2711/1000.jpg" alt="A photo being taken of Frida Kahlo." width="1000" height="600" class="gu-image" /> <figcaption> <span class="element-image\_\_caption">‘Look with your eyes not through a screen,’ hissed one man.</span> <span class="element-image\_\_credit">Photograph: Elizabeth Flux</span> </figcaption> </figure>  <p>For me, seeing Piet Mondrian’s Composition in Red, Blue, and Yellow (1930) was simultaneously thrilling and hollow. It’s a work that inspired Yves Saint Laurent and countless replicas (and Etsy shops) in the decades since. I’ve loved it so much from afar, the real thing never stood a chance. But even so, the opportunity to be just inches away from it is not something I would trade.</p> <p>For every piece that can’t meet an impossible standard however, there’s one that surpasses expectation. It’s brighter, perhaps. Livelier. Or there’s a detail you only take in after seeing it in real life – in “the Dalí” for example, you can get close enough to see that the “monstrous fleshy figure”, as it is officially described, is a side profile of the artist’s face, complete with eyelashes. In between the “rock-stars”, too, there are new favourites-to-be.</p> <p>Amateur photos of these artworks on display are going to fill our newsfeeds. There will be so many selfies with the Van Gogh that greets you at the entrance, with the Warhol Marilyn Monroe series, with Roy Lichtenstein’s iconic Drowning Girl (1963) that you’ll feel like you’ve seen the exhibition 10 times. One man tsk-ed as everyone struggled to capture their own image of “the Dalí”, shaking his head and saying under his breath “look with your eyes not through a screen”.</p> <p>But this is part of the experience too – at least in moderation. My phone is filled with photos that no one but me will want to see. It’s because these images are so easy to access, to copy, to buy a cheap print of to slap on your wall, that we want proof – these pictures say I was there, that I saw these works in person. </p> <p>The exhibition is so large that to get to the second half we need to pass through the gift shop. On the wall there is a print of “the Dalí” for sale. It’s bigger than the original – and just a little bit duller.</p> <p><em>• <a href="https://www.ngv.vic.gov.au/exhibition/moma-at-ngv/">MoMA at NGV</a>: 130 Years of Modern and Contemporary Art is showing at the National Gallery of Victoria until 7 October</em></p>\\
Opinion & <p>Apart from the hard right political trajectory of the government’s Brexit stance, what worries me most is that Theresa May’s team appears to be ignoring almost every one of the rules of negotiation. These are in the DNA of the best British diplomats, who are famed across the world for their creative professionalism – as I witnessed at first hand as a minister.</p> <p>There was yet another warning from the EU’s chief negotiator, Michel Barnier, on Thursday that there is still no clarity on the UK’s policy, and that the stalemate over the UK’s financial commitments could drag on for months. This shows just how precarious the position is. You don’t have to agree with Barnier’s negotiating hand to accept that he needs to understand what ours is.</p>  <aside class="element element-pullquote element--supporting"> <blockquote> <p>[The other side] needs to know what makes you tick: your family, hobbies, favourite football team</p> </blockquote> </aside>  <p>Worryingly, the government seems not to have grasped the first principle of negotiation: personal relationships. Although enormously time-consuming to develop, they pay massive dividends. And it’s elementary: you are more likely to get a sympathetic response from someone who is on friendly terms with you, who understands not just why you are adopting a policy position, but also what makes you tick: your family, hobbies, favourite football team. I’ve been in the middle of tense standoffs, trying to fathom what may be pure brinkmanship as against a genuine bottom line, when some humour – “come on X, give me a break, at least your football team won last night” – has led to a rapprochement. Personality clashes can otherwise lead to stubbornness, and block progress and cooperation.</p> <p>When I was a minister, I used the global enthusiasm for football continuously. Finding out that the vice president of Ghana (whom I had never met) was a Manchester United fan, and presenting him in Accra with an Alex Ferguson-signed team shirt, created an immediate bond between us. At a tricky time of transition from military coup to democracy, he was thrilled.</p> <p>The former IRA commander <a href="https://www.theguardian.com/commentisfree/2017/mar/24/the-guardian-view-on-martin-mcguinnesss-funeral-enlarging-the-definition-of-us" title="">Martin McGuinness</a>, I was astonished to glean from him in one of our early meetings in 2005, was an ardent England cricket fan. We often talked about the Ashes. Discovering what you have in common with someone helps ease the way to resolving differences.</p>  <figure class="element element-image" data-media-id="bfd2e9ffc78aaeb0ce67f50ff397a250fbe4ea03"> <img src="https://media.guim.co.uk/bfd2e9ffc78aaeb0ce67f50ff397a250fbe4ea03/0\_50\_2000\_1200/1000.jpg" alt="Ian Paisley and Gerry Adams around a triangular table as the Northern Ireland deal is announced in March 2007." width="1000" height="600" class="gu-image" /> <figcaption> <span class="element-image\_\_caption">Ian Paisley and Gerry Adams around a triangular table as the Northern Ireland deal is announced in March 2007.</span> <span class="element-image\_\_credit">Photograph: Paul Faith/AP</span> </figcaption> </figure>  <p>By contrast, May and David Davis, flying quickly in and out to deliver pre-scripted lines for the Brexiteer tabloids back home, appear friendless in Brussels – which is astonishing when they are confronting the biggest challenge for our country since the second world war. They don’t seem to have grasped that European leaders need to be negotiating partners, not enemies. And this is nothing to do with being “soft on Brussels”. The best way to get a good settlement is to have the other side want to help deliver it.</p> <p>And putting in the legwork really does pay dividends. After I’d had endless dinners, lunches, breakfasts and one-to-ones during European convention talks in Brussels in 2002-03, Germany suddenly replaced their lowly delegate with foreign minister Joschka Fischer. “I’m here because you are here!” he said to me cheerfully. The convention outcome was reported in the media across Europe as a British triumph.</p> <p>Rule number two is to be clear. Instead, the government position papers last month were oblique and opaque waffle. The paper on whether the Irish border can remain open – so critical to the peace process – was full of pious platitudes with a dogmatic objection to continued UK membership of the customs union and the single market, which provides the only obvious solution.</p> <p>The third rule is trust. You will never make friends with everyone. But you don’t have to agree with your opponents, or even like them, to build a relationship of mutual understanding – which takes time, empathy and patience. Tragically, trust and empathy between the principal figures in Brussels and London appears non-existent. Despite the ritual Barnier-Davis courtesies, to say their relationship appears frosty is to put it charitably.</p> <p>It is the absence of trust that has led to Northern Ireland’s suspension of self-government. May’s salvation deal with the DUP, coupled with her perplexing unwillingness to get personally involved, makes it almost impossible for the government to play the essential role of honest broker, trusted by both sides. That, plus Brexit, creates a real risk of the settlement falling apart.</p> <aside class="element element-rich-link element--thumbnail"> <p> <span>Related: </span><a href="https://www.theguardian.com/politics/2017/sep/28/brexit-negotiations-could-take-months-to-progress-to-next-phase-says-barnier">Brexit talks could take months to progress to next phase, says Barnier</a> </p> </aside>  <p>The fourth rule is to think creatively and laterally. The night before the announcement of the Northern Ireland deal in March 2007, it almost collapsed: not over the substance, but a table. As a matter of principle Ian Paisley would not face the world’s media on the same side as Gerry Adams who, as a matter of\&\#xa0;principle, insisted on doing so. I wanted them close for the historic photo of the bitter old enemies together. Impasse. Disaster.</p> <p>Finally, with the hours ticking by, we came up with a solution: building overnight a new triangular table. In the morning, Paisley sat on one side, Adams\&\#xa0;on the other, with honour satisfied and the world oblivious to the preceding drama.</p> <p>Anyone can see there will be similarly fraught moments ahead in Brussels – with 27 member states (not including the UK) it’s bound to be complicated. It’s time for May and Davis to read up on the arts of successful British negotiation. Otherwise there will be no deal – unless of course that is what they, and backseat driver Boris Johnson, really want.</p> <p>• Peter Hain was Labour MP for Neath from 1991 to 2015. He has held the posts of secretary of state for Wales, for work and pensions, and for Northern Ireland. He was also minister for Africa, and a leading anti-apartheid campaigner. His latest book is Back to the Future of Socialism</p>\\
Art and design & <p>A huge painting of Venice’s Grand Canal by Francesco Guardi has had a temporary export bar placed on it by ministers in an attempt to prevent it leaving the UK.</p> <p>‘Rialto Bridge with the Palazzo dei Camerlenghi’ was owned by successive generations of the Guinness family before it was sold last summer to an overseas buyer at <a href="http://www.christies.com/features/Guardi-Rialto-Bridge-with-the-Palazzo-dei-Camerlenghi-8389-3.aspx?sc\_lang=en\#!\#FID-8389">a Christie’s auction</a> for £26.2m.</p> <p>The export bar is to allow time for a UK buyer to come forward to match the asking price which, with VAT of £591,000, is just shy of £26.8m.</p> <p>The arts minister John Glen said he hoped the 2-metre wide painting could be kept in the UK “where it can be appreciated and admired by future generations for many years to come.” He called it “magnificent” and a “true masterpiece that encapsulates the vibrant atmosphere and light of 18th-century Venice”.</p> <p>Guardi was one of the greatest Venetian view painters of the 18th century and the huge painting is considered his masterpiece. Busy with commerce and gondolas it shows familar landmarks such as the Rialto Bridge, the fruit market and the <a href="https://www.venetoinside.com/hidden-treasures/post/the-stone-high-reliefs-of-palazzo-dei-camerlenghi-in-rialto/">Palazzo dei Camerlenghi</a>.</p> <p>The decision to place a temporary export bar was based on a recommendation by the Reviewing Committee on the Export of Works of Art and Objects of Cultural Interest (RCEWA).</p> <p>Aidan Weston-Lewis, a member of the committee, said: “Commissioned by a British visitor to Venice in the late 1760s, it has remained in the UK ever since and has frequently been on public display. Its departure from these shores would be a regrettable loss.”</p> <p>The search is on for a UK buyer but, in truth, the chances of so much money being paid for it are slim.</p> <p>In 2011, <a href="https://www.theguardian.com/artanddesign/2011/oct/17/export-bar-francesco-guardi-painting">a similar export bar was placed on the painting’s companion piece ‘Venice, a View of the Rialto Bridge, Looking North, from the Fondamenta del Carbon’</a>. It sold for a record £26.7m at Sotheby’s. No UK buyer was found.</p> <p>The painting was commissioned in around 1768 by the English MP and Jamaican sugar plantation owner Chaloner Arcedeckne. It remained in his family until bought by Edward Guinness, the 1st Earl of Iveagh – <a href="http://www.english-heritage.org.uk/visit/places/kenwood/history-stories-kenwood/history/collections">many of whose paintings are on free public display at Kenwood in north London</a> – and was passed down through the family. </p> <p>A decision on an export licence has been deferred until July to allow time for a potential buyer to come forward.</p>\\
Travel & <p>Havana will be 500 years old next year. I have been coming here for 30 years, and while I have experienced just a small part of the city’s history it has had an outsized impact on my life. I fell hopelessly in love on the spot – hooked by its openhearted people and weathered beauty. No one looks you more in the eyes than the Habaneros. It is the city of stares, of sincerity and of kindness. Come here to be mesmerised by the eclectic architecture, with a surprise around every corner; to sway to the Cuban rhythms in surround sound as you stroll; to be inspired by cutting-edge contemporary art; and to taste the creative cuisine of the next gastronomic capital of the world. And whatever you do, explore beyond the borders of beautiful Habana Vieja (Old Havana), to discover the city’s distinct neighbourhoods.</p> <h2><strong>Malecón 663</strong></h2>  <figure class="element element-image" data-media-id="dd99de8a874f25a65925dbe73d1336fdf2a7801f"> <img src="https://media.guim.co.uk/dd99de8a874f25a65925dbe73d1336fdf2a7801f/0\_192\_6143\_3687/1000.jpg" alt="Malecón 663 Boutique Hotel, Havana" width="1000" height="600" class="gu-image" /> </figure>  <p>Opened last year, Havana’s most striking boutique hotel is the brainchild of Sandra Exposito and her husband Orlandito Mengual. It has four guestrooms, each by a different local designer and in different styles: retro eclectic, art deco, vintage 1950s and contemporary. There is a cocktail and tapas bar on the ground floor, as well as a shop selling artisanal soaps and one-off jewellery. The roof terrace has great views of the Malecón, Havana’s famous promenade. <br>•<em> D</em><em>oubles from £135 B\&amp;B, </em><a href="http://malecon663.com/en/home"><em>malecon663.com</em></a></p> <h2>Niels Reyes </h2>  <figure class="element element-image element--showcase" data-media-id="a9f40f9766ca97278fe4b0206da568a651ccc909"> <img src="https://media.guim.co.uk/a9f40f9766ca97278fe4b0206da568a651ccc909/0\_14\_6720\_4032/1000.jpg" alt="Niels Reyes, Havana, Cuba" width="1000" height="600" class="gu-image" /> </figure>  <p>Niels Reyes is one of the most exciting artists on the roaring Cuban art scene, prized by collectors from New York to Beijing. Up close, his large-scale portraits look abstract but from further away they speak volumes about the human condition. After graduating from Havana’s Instituto Superior de Arte in 2006, Reyes was soon exhibiting around the world. He is represented in Cuba by <a href="https://www.facebook.com/Galer\%C3\%ADa-ARTIS-718-275301936001695/">Galería Artis 718</a> (Avenida 7, corner Calle 18, Miramar) and <a href="https://www.facebook.com/collagehabana/">Collage Habana</a> (San Rafael \#103, between Consulado and Industria, Centro Habana). <br>• <em>Niels Reyes is on <a href="https://www.facebook.com/niels.reyescadalso">Facebook</a></em></p> <h2><strong>Nazdarovie</strong></h2>  <figure class="element element-image" data-media-id="1e7c6fab20f52151693cf6fa6240b7dc521bcef6"> <img src="https://media.guim.co.uk/1e7c6fab20f52151693cf6fa6240b7dc521bcef6/0\_87\_1280\_768/1000.jpg" alt="128 Nazdorovie Putinka, Havana" width="1000" height="600" class="gu-image" /> </figure>  <p>Nazdarovie is a Soviet-style <em>paladar </em>(privately owned restaurant) set up in 2014 by Gregory Biniowsky, a Slavo-Canadian lawyer who had lived in Cuba for two decades, and wanted to bring a flavour of the motherland to thousands of Soviet expats. Russian and Ukrainian chefs cook favourites (from £5) such as borscht with <em tabindex="-1">smetana</em> (sour cream), savoury<em tabindex="-1"> varenikis</em> (dumplings), chicken kiev, traditional beef <em tabindex="-1">stroganoff</em>, and Russian black bread. They also make cocktails such as black or white Russians with Stolichnaya or Putinka vodka. Customers include ambassadors from former Soviet republics, and the place is decorated with old propaganda posters and huge matryoshka dolls. There’s also memorabilia from some of the many Cubans who studied in the USSR; the CCCP hockey shirt by the bar is my donation from 10 years in Moscow.<br tabindex="-1">• <em tabindex="-1">Malecón \#25, 2nd floor, between Paseo del Prado and Carcel</em>, <em tabindex="-1">Centro Habana</em>, <em tabindex="-1">open daily noon-midnight. </em><a draggable="true" href="http://nazdarovie-havana.com/"><em tabindex="-1">nazdarovie-havana.com</em></a></p> <h2><strong>El Bosque de La Habana</strong></h2>  <figure class="element element-image element--showcase" data-media-id="2f017125a56a3318b474638890b5c3086014bb15"> <img src="https://media.guim.co.uk/2f017125a56a3318b474638890b5c3086014bb15/0\_159\_4032\_2419/1000.jpg" alt="El Bosque, Havana" width="1000" height="600" class="gu-image" /> </figure>  <p>El Bosque (the forest), between the neighbourhoods of Vedado and Miramar, is often called the lungs of Havana, where 300-year-old banyan, carob and jagüey trees are draped in vines that make them look like green monsters. It was laid out in the 1930s by French landscape architect Jean-Claude Nicolas Forestier (designer of the Eiffel Tower’s Champs-de-Mars gardens) as a “fusion of nature, architecture and city”. After the 1959 revolution, it was expanded to include Parque Almendares, with its playground, pony rides, mini-golf, lake and amphitheatre for marionette shows. This is where a new generation of Habaneros come for bike rides along the Almendares river or to barbecue and party.</p> <h2>Belview ArtCafé</h2>  <figure class="element element-image" data-media-id="1bc770cc2c7f3f3d904dc0920d2866936765cd0c"> <img src="https://media.guim.co.uk/1bc770cc2c7f3f3d904dc0920d2866936765cd0c/0\_108\_6626\_3975/1000.jpg" alt="Belview Art Cafe, Havana, Cuba" width="1000" height="600" class="gu-image" /> </figure>  <p>Every last detail of Belview ArtCafé is carefully thought out, from the world globes over the bar to the photographs by German photographer <a href="http://www.creutzmann.com/">Sven Creutzmann</a>, who has lived in Havana for nearly 30 years and is married to owner Clara Veliz. Sit on the sleek 1957 Bel Air sofa or one of the padded Thonet chairs and order one of the house specialities (around £3) such as pulled pork crepe, sweet potato fries with tzatziki or spaghetti amatriciana. For dessert, there are brownies with ice-cream or a crepe with banana and chocolate cream. On Saturdays, it sells homemade bread including black, sourdough, ciabatta, brioche and challah, as well as croissants and other pastries. <br>• <em>Calle 6 \#412, corner 19, Vedado, closed Mon. </em><a href="https://laesquinadelpan.com/"><em>laesquinadelpan.com</em></a></p> <h2><strong>Salchipizza</strong></h2>  <figure class="element element-image" data-media-id="65fe171ab2eb09c599cc7a503ac05f14bd5988f4"> <img src="https://media.guim.co.uk/65fe171ab2eb09c599cc7a503ac05f14bd5988f4/0\_403\_853\_853/500.jpg" alt="Alberto from Salchipizza, Havana, Cuba" width="500" height="500" class="gu-image" /> <figcaption> <span class="element-image\_\_caption">Chef Alberto González</span> </figcaption> </figure>  <p>In the 19th century, Havana’s population spread west from the old city into what is now known as Centro Habana. Gritty and raw-feeling, this densely populated working-class area is the perfect place to experience everyday life in the capital. Michelin-starred chef Alberto González, who spent 14 years as a chef in Italy and studied at El Bulli in Spain, set up his shop and community project bakery here in 2014. At Salchipizza, González encourages everyone to taste his breads made with quality ingredients, including seeds, grains and flours he brings back from trips abroad. He also bakes gluten-free loaves (£1.80), and wholewheat sourdough made with an 80-year-old starter. Banana bread and dreamy coffee cake are sold by the slice. Look out for the photographs of Martin Luther King, Malcolm X and Barack Obama at the entrance.<br>•<em> Calle Infanta \#562, between Valle and Zapata, Centro Habana, Tues-Sat 9am-4pm. On </em><a href="https://www.facebook.com/Salchipizza/?rf=765366980262629"><em>Facebook</em></a></p> <h2><strong>TocaMadera</strong></h2>  <figure class="element element-image element--showcase" data-media-id="2ad5f3f805e18402a2ff241ebcccd398cb5faa82"> <img src="https://media.guim.co.uk/2ad5f3f805e18402a2ff241ebcccd398cb5faa82/61\_406\_6473\_3883/1000.jpg" alt="TocaMadera, Havana, Cuba" width="1000" height="600" class="gu-image" /> </figure>  <p>Enrique Suárez leads Cuba’s New Food movement and his latest venture, TocaMadera (touch wood), is the talk of the town. Suárez travelled the world to learn his trade but has returned to his Cuban comfort food roots to serve “traditional food with modern flavours”. His seasonal menu changes depending on the availability of produce, and many ingredients are sourced from Fernando Funes’s organic farm 12 miles west of Havana. His hamburgers cost £3, but fancier dishes such as truffle risotto, torched tuna with capers and black olives, penne with chicken and truffle salsa or avocado and shrimp salad start at about £9. For dessert, in season, there is mango tarte tatin with vanilla ice-cream. <br>• <em>Calle 38 \#118, between Avenida 1 and 3, Miramar, </em><em>Tues-Sat lunch and dinner, and Sun brunch</em><em>. On </em><a href="https://www.facebook.com/tocamaderahabana/"><em>Facebook</em></a></p> <h2><strong>Fábrica de Arte Cubano</strong></h2>  <figure class="element element-image" data-media-id="7728eea0d45960cdd36f81a3628fea4a0e30d7fc"> <img src="https://media.guim.co.uk/7728eea0d45960cdd36f81a3628fea4a0e30d7fc/0\_207\_5616\_3370/1000.jpg" alt="FAC, Havana" width="1000" height="600" class="gu-image" /> </figure>  <p>In a former cooking oil plant near the banks of the Almendares, Fábrica de Arte Cubano (FAC) is still the hippest place in Havana. There’s always a queue around the block to get into this extraordinary venue, part cultural centre, part night spot. Imagine Bauhaus mixed with a hot avant-garde Caribbean vibe in a number of galleries and performance spaces. It’s so big you can get lost in its multimedia labyrinth: lounge, chill, think, drink, eat, meet people, shop, dance, catch a film or concert or enjoy ever-changing art exhibitions. A clever way to avoid the inevitable queue is to book a table at its new Tierra restaurant, in former shipping containters, whose globetrotting menu might include <em>salmorejo</em> (tomato and bread soup from Andalucía), shrimp <em>moqueca</em> (a Brazilian-inspired seafood stew) and South American steak with mustard and fries (around £40 for two). The interior has a comfortable country kitchen atmosphere but outside there’s an Astroturf patio with bar. <br>•<em> </em><a href="http://www.fac.cu/"><em>fac.cu</em></a><em>, Calle 26, corner 11, Vedado, Thurs-Sun 8pm-3 am, entry £2.25. To book at Tierra (essential) ring +535 565 2621</em></p> <h2><strong>El Café</strong></h2>  <figure class="element element-image" data-media-id="2e3b9a8db0fc5f7718f544bb26cf1be75c7f4513"> <img src="https://media.guim.co.uk/2e3b9a8db0fc5f7718f544bb26cf1be75c7f4513/0\_402\_6720\_4032/1000.jpg" alt="El Cafe vegan sandwich" width="1000" height="600" class="gu-image" /> </figure>  <p>After six years working in restaurants and cafes in London, Nelson Rodríguez Tamayo returned to Havana to open El Café. It’s in a beautiful Spanish colonial building that took eight months to renovate, and is the place to go for everything fresh and local. Coffee beans from Café O’Reilly (a Havana institution), freshly squeezed guava, pineapple, carrot or beetroot juice, and breakfasts of fried eggs, tomato, avocado and bacon (around £3). For lunch, there are veggie sandwiches (pictured) with homemade hummus , avocado, red pepper, kale and grilled aubergine on homemade sourdough bread, or the pulled-pork sandwich with yucca and orange marmalade sauce (from £3.75). It also does one of the best mojitos in Havana (£2.25).<br>• <em>Amargura \#358, between Aguacate and Villegas in Habana Vieja. </em><em>On </em><a href="https://www.facebook.com/elcafehavana"><em>Facebook</em></a><em>, open daily 9am-6 pm</em></p> <h2><strong>Café Teatro Bertolt Brecht</strong></h2>  <figure class="element element-image" data-media-id="5266af83b2297e5c00e49f32bd76ce0ee27ffdb7"> <img src="https://media.guim.co.uk/5266af83b2297e5c00e49f32bd76ce0ee27ffdb7/0\_105\_6720\_4032/1000.jpg" alt="Bertolt Brecht, Havana, Cuba" width="1000" height="600" class="gu-image" /> </figure>  <p>All the great Cuban musicians play at Café Teatro Bertolt Brecht, from Telmary (pictured), Cuba’s premier female hip-hop artist, to Interactivo, who play Afro-Cuban jazz fusion every Wednesday night. Formerly the Jewish Community Center, this intimate basement venue now hosts live music throughout the week. There is a cool bar at the back of the 1950s modernist structure (with a Star of David on the facade and elegant front doors covered in Semitic symbols), as well as a small outdoor cafe on the second-floor terrace, where local actors and artists congregate for coffee in the afternoon before rehearsal. <br>• <em>Calle 13, between I and J, Vedado. On </em><em><a href="https://www.facebook.com/brechtcafe/">Facebook</a>. Entrance fee around </em><em>£3.70</em></p>  <figure class="element element-image element--thumbnail" data-media-id="99785889f32eb81f36699ebbbbe0711f8a24016f"> <img src="https://media.guim.co.uk/99785889f32eb81f36699ebbbbe0711f8a24016f/0\_0\_1613\_2475/652.jpg" alt="Book cover for 300 Reasons to Love Havana." width="652" height="1000" class="gu-image" /> <figcaption> <span class="element-image\_\_caption"><strong><a href="http://heidihollinger.com/">Heidi Hollinger </a>is the author of <a href="https://www.amazon.co.uk/Reasons-Love-Havana-Heidi-Hollinger/dp/1988002621">300 Reasons to Love Havana</a></strong></span> </figcaption> </figure>  <p><strong>Getting there<br></strong>Virgin Atlantic flies from Gatwick to Havana from £516; KLM from Heathrow via Amsterdam from £498.</p> <p><strong>Best views<br></strong>Buy a day pass to the spa at Havana’s <a href="https://www.kempinski.com/en/havana/gran-hotel-kempinski-la-habana/luxury-spa/day-spa/">Gran Hotel Manzana Kempinski</a> and spend the day at the rooftop infinity pool and bar with a 360-view view of the Capitolio, Gran Teatro, Parque Central, Museo de Bellas Artes and the Bacardi building.</p> <p><strong>Climate</strong><br>June is the hottest month, with average highs of 31C. August-October is hurricane season. Winter months are still hot and sunny by European standards.</p> <p><strong>What’s on</strong><br><a href="http://jazzhavana.com/">Jo Jazz Havana</a> (14-21 November) features-up-and-coming musicians on the Cuban scene. The internationally renowned <a href="https://jazzcuba.com/">Havana Jazz Festival</a> is on 13-21 January 2019.</p> <p><strong>Exchange rate<br></strong>1 CUC = £0.74.<br>Beer (0.5l): about £1.10.</p>\\
Art and design & <figure class="element element-image element--thumbnail" data-media-id="25bbe1493bf855cd2844c6cfca3fb199f5a837ab"> <img src="https://media.guim.co.uk/25bbe1493bf855cd2844c6cfca3fb199f5a837ab/0\_102\_1815\_2846/638.jpg" alt="Rubber duck eyeball card Four Corners Books" width="638" height="1000" class="gu-image" /> </figure>  <p>Parents who look disappointedly at their children, glued to their phones refusing to engage with real life, would do well to remember the <a href="http://www.chroniclelive.co.uk/news/north-east-news/its-40-years-cb-radio-10802903">Citizens Band (CB) Radio craze of the late 1970s</a>. For a few years, Britain was gripped by hobbyists in their garage or car, having stilted chatroom-like conversations with strangers over unused AM radio, mostly in their area but occasionally hundreds of miles away. It was basically like an early version of <a href="https://www.theguardian.com/media/pda/2010/mar/08/chatroulette">ChatRoulette</a>, but with no chance of having to look at someone’s genitals.</p>  <figure class="element element-image" data-media-id="04e111e3dd5018b57be31fc07012b6559981e910"> <img src="https://media.guim.co.uk/04e111e3dd5018b57be31fc07012b6559981e910/0\_100\_1488\_893/1000.jpg" alt="Star Rider eyeball card." width="1000" height="600" class="gu-image" /> <figcaption> <span class="element-image\_\_caption">Star Rider eyeball card.</span> <span class="element-image\_\_credit">Photograph: Four Corners Books</span> </figcaption> </figure>  <figure class="element element-image element--thumbnail" data-media-id="1765fcfb6e1d3b93d5e4f41600a8c4c04d9e47e5"> <img src="https://media.guim.co.uk/1765fcfb6e1d3b93d5e4f41600a8c4c04d9e47e5/0\_143\_1804\_2504/720.jpg" alt="Garibaldi eyeball card." width="720" height="1000" class="gu-image" /> </figure>  <p>In the US, <a href="https://www.youtube.com/watch?v=uDUXvR79wS4">CB radio was popularised by truckers</a>, but in the UK there was more of a social element. People talking into the abyss, waiting for a response. There would be local meet-ups where CB enthusiasts could finally put faces to muffled, interferency names. When they did, they wanted to leave a calling card so they could be identified back on the airwaves, so they would make “Eyeball cards”: photocopied business cards, to be collected like <a href="https://www.theguardian.com/football/2017/jun/05/womens-euro-2017-panini-stickers">Panini stickers</a>.</p>  <figure class="element element-image" data-media-id="6fd60030792a547067d0e4e2f7fa143a2aa290f5"> <img src="https://media.guim.co.uk/6fd60030792a547067d0e4e2f7fa143a2aa290f5/0\_66\_2948\_1902/1000.jpg" alt="Shy Fox and Gonzo eyeball card." width="1000" height="645" class="gu-image" /> <figcaption> <span class="element-image\_\_caption">Shy Fox and Gonzo eyeball card.</span> <span class="element-image\_\_credit">Photograph: Four Corners Books</span> </figcaption> </figure>  <figure class="element element-image element--thumbnail" data-media-id="c5ef1905e65cd9130ba6e99be13fd12ab7007e07"> <img src="https://media.guim.co.uk/c5ef1905e65cd9130ba6e99be13fd12ab7007e07/0\_127\_2188\_2573/850.jpg" alt="Je Soap eyeball card" width="850" height="1000" class="gu-image" /> </figure>  <p>CB was dominated by impenetrable slang and code names, so these cards would often bear names like “Little Bo Peep” and “Randy Andy”. The cards also told others which area you were broadcasting from, but since it was illegal to broadcast on CB radio until the early 80s, these had to be disguised. “Dyfed in Wales was ‘Sausage Town’, Bexhill was ‘Foggy Town’, Diss in Norfolk was ‘Dodge City’,” explains William Hogan in his new book, <a href="https://www.bookdepository.com/Eyeball-Cards-William-Hogan/9781909829084">Eyeball Cards, The Art of British CB Radio Culture</a>, which includes hundreds of images of the cards.</p>  <figure class="element element-image" data-media-id="9d6582d92209f24e36fb4104133b64448d3a2374"> <img src="https://media.guim.co.uk/9d6582d92209f24e36fb4104133b64448d3a2374/0\_97\_2700\_1749/1000.jpg" alt="Spider eyeball card." width="1000" height="648" class="gu-image" /> <figcaption> <span class="element-image\_\_caption">Spider eyeball card.</span> <span class="element-image\_\_credit">Photograph: Four Corners Books</span> </figcaption> </figure>  <figure class="element element-image element--thumbnail" data-media-id="2fa374c0858a7c1684271d78af2b4233217c6b8a"> <img src="https://media.guim.co.uk/2fa374c0858a7c1684271d78af2b4233217c6b8a/0\_52\_1811\_2636/687.jpg" alt="Magic Man eyeball card." width="687" height="1000" class="gu-image" /> </figure>  <p>Different regions adopted different styles of cards, and illustrators set up small businesses to satisfy demand. Most cards adopted a Viz-like style: Porcupine and Jinx, the code name for a romantically involved couple with a CB, had a cartoon couple with animal faces making love. Others were more literal – an enthusiast who used the handle “Snowberry” simply depicted a berry ski-ing down the mountainside – but most bore the slogan, “You have just eyeballed ...” After broadcasting on CB radios became legal, they lost much of their appeal and when mobile phones became popular, most were consigned to the attic. But the cards live on, little cardboard symbols of Britain’s first social media revolution.</p> <p>• <a href="http://www.fourcornersbooks.co.uk/\#/books/eyeball\_cards/">Eyeball Cards, The Art of British CB Radio Culture</a> by William Hogan and David Titlow will be published by Four Corners Books (RRP £14) on 15 September.</p>\\
\bottomrule
\end{tabular}}
\end{table}

\rowcolors{2}{white}{white}

\rowcolors{2}{gray!6}{white}

\begin{table}[!h]

\caption{\label{tab:data_collection}Table containing TECHNOLOGY  News}
\centering
\resizebox{\linewidth}{!}{
\begin{tabular}[t]{ll}
\hiderowcolors
\toprule
sectionName & body\\
\midrule
\showrowcolors
Technology & <p>Google’s AI assistant will identify itself as a robot when calling up businesses on behalf of human users, the company has confirmed, following accusations that the technology was deceitful and unethical.</p> 
<p>The feature, called Google Duplex, was <a href="https://www.theguardian.com/technology/2018/may/08/google-duplex-assistant-phone-calls-robot-human">demonstrated at the company’s I/O developers’ conference on Tuesday</a>. It is not yet a finished product, but in the two demos played for the assembled crowd, it still managed to be eerily lifelike as it made bookings at a hair salon and a restaurant.</p> 
<p>But the demonstrations sparked concern that the company was misleading those on the other end of the conversation into thinking they were dealing with another human, not a machine. The generated voice not only sounds extremely natural, but also inserts lifelike pauses, um-ing and ah-ing, and even responding with a wordless “mmm-hmm” when asked by the salon worker to “give me one second”.</p> 
<p>The social media theorist Zeynep Tufekci was one of many concerned by the demo. She tweeted that it was “horrifying” and described Silicon Valley as “ethically lost”. </p> 
<figure class="element element-tweet" data-canonical-url="https://twitter.com/zeynep/status/994233568359575552"> 
 <blockquote class="twitter-tweet">
  <p lang="en" dir="ltr">Google Assistant making calls pretending to be human not only without disclosing that it's a bot, but adding "ummm" and "aaah" to deceive the human on the other end with the room cheering it... horrifying. Silicon Valley is ethically lost, rudderless and has not learned a thing.</p>— zeynep tufekci (@zeynep) 
  <a href="https://twitter.com/zeynep/status/994233568359575552?ref\_src=twsrc\%5Etfw">May 9, 2018</a>
 </blockquote> 
</figure> 
<p>In its initial blogpost announcing the tech, Google said: “It’s important to us that users and businesses have a good experience with this service, and transparency is a key part of that. We want to be clear about the intent of the call so businesses understand the context. We’ll be experimenting with the right approach over the coming months.”</p> 
<p>In <a href="https://www.theverge.com/2018/5/10/17342414/google-duplex-ai-assistant-voice-calling-identify-itself-update">a statement to the Verge</a>, the company has confirmed that that will include explicitly letting people know they are interacting with a machine. “We understand and value the discussion around Google Duplex – as we’ve said from the beginning, transparency in the technology is important,” a Google spokesperson said. “We are designing this feature with disclosure built in, and we’ll make sure the system is appropriately identified. What we showed at I/O was an early technology demo, and we look forward to incorporating feedback as we develop this into a product.”</p> 
<p>Google’s hope with Duplex is that it will enable a range of interactions with businesses that only have a phone connection, where this was previously limited to those with more hi-tech set-ups. The company envisages being able to call businesses to ask about opening hours then posting the information on Google; allowing users to schedule a reservation even when a business is closed, lining up the Duplex call for when doors open; and solving accessibility problems by, for instance, letting hearing-impaired users book over the phone, or enabling phone bookings across a language barrier. </p> 
<p>The company said it would begin testing Duplex more widely “this summer … to help users make restaurant reservations, schedule hair salon appointments, and get holiday hours over the phone”.</p> 
<figure class="element element-atom"> 
 <gu-atom data-atom-id="4a4b5894-522c-40cb-ac52-2eddf91d87fb" data-atom-type="qanda"> 
  <div>
   <div class="atom-Qanda">
    <p></p>
    <p>Artificial Intelligence has various definitions, but in general it means a program that uses data to build a model of some aspect of the world. This model is then used to make informed decisions and predictions about future events. The technology is used widely, to provide speech and face recognition, language translation, and personal recommendations on music, film and shopping sites. In the future, it could deliver driverless cars, smart personal assistants, and intelligent energy grids. AI has the potential to make organisations more effective and efficient, but the technology raises serious issues of ethics, governance, privacy and law. </p>
    <p></p>
   </div>
  </div>
 </gu-atom> 
</figure>\\
Education & <p>Former car designer and engineer Samir Abid completed an MBA at <a href="http://www.wbs.ac.uk/courses/mba/">Warwick Business School</a> in 2009. In 2011, he started his own sports technology company, Pace Insights, which helps sports professionals improve athlete and team performance through better use of data.<br></p> <p>“I’ve always wanted to start my own business. I could have done it without an MBA, but I wouldn’t have. I thought long and hard before applying – I’m quite geeky and technical. After a promotion in my previous role, I felt out of my depth managing 15 people, so I looked for a qualification that would help with management. The economic crisis of 2007-2008 gave me the final push to get it done rather than talk about it.</p> <aside class="element element-rich-link element--thumbnail"> <p> <span>Related: </span><a href="https://www.theguardian.com/education/2018/mar/19/why-the-mba-is-still-a-gamechanging-qualification">Why the MBA is still a gamechanging qualification</a> </p> </aside>  <p>Interestingly, I benefited in ways that were different from what I expected when I applied. I’d moved to Warwickshire to work in automotive and motorsports and lived just 20 minutes away from the business school. On day one, there were 100 people sitting in the room and just five of us were from the UK. That was fascinating. There have been two or three marriages among our year group. We were in our late 20s but we went straight back into a university mindset – working hard and having fun.</p> <p>I can’t say I study my files every day, but so much has stayed with me. Coming from a techy world, I didn’t have such a broad perspective on business. Organisational behaviour was an eye-opener. I read case studies about what some managers do and thought: ‘That’s stupid.’ Then I took a step back and thought: ‘Oh, that’s exactly what I did when I was a manager.’</p> <p>All the entrepreneurial modules were really useful, although I’d have liked more practical support. They helped me know how to analyse a business opportunity and how to pitch Dragons’ Den-style in front of investors. I came out the other end confident I could speak the right language in a way that gets things done.</p> <p>Without an MBA I wouldn’t have had the confidence to set up a business and I might have forever remained a contractor, or a poor manager. It was a learning and a maturing experience.</p> <p>In the last few years we’ve supported nearly 40 elite sporting customers, all household names – including British Athletics and Team Sky. Through initiatives such as <a href="https://www.yourdatadriven.com/">yourdatadriven.com</a>, we’re helping teams do more with their data.”</p>\\
Games & <p>One day, on my way past the outskirts of Kabukicho – Tokyo’s red-light district, infamously depicted in the <a href="https://www.theguardian.com/games/2018/mar/15/yakuza-6-song-life-review-gangster-game-saga">Yakuza games</a> – I spot a curious advertisement. At first glance, it looks like nothing out of the ordinary: a woman cheerfully donning a VR headset, with kanji lettering welcoming passersby to come in and try the technology for themselves. As my eyes wander to the logo in the corner, I realise that the poster is promoting Soft On Demand – one of Japan’s biggest porn, or “AV” (adult video), companies. I’m staring at a billboard for a virtual brothel.</p> <p>A stone’s throw away is Bandai Namco’s massive VR Zone complex, an indoor, 38,000 sq ft all-VR theme park that opened just over a year ago. And further south, on the artificial island of Odaiba, Sega recently cleared out a massive room in its Joypolis amusement park to make space for Zero Latency VR, a “warehouse scale, free-roam, multiplayer virtual reality entertainment” where a team of zombie hunters are equipped with “military-grade” motion-tracking backpacks and let loose on the undead with an arsenal of plastic firearms.</p>  <figure class="element element-image" data-media-id="fd4ec3a32d17b4be7f85032d991c9b520ff7e381"> <img src="https://media.guim.co.uk/fd4ec3a32d17b4be7f85032d991c9b520ff7e381/0\_245\_3264\_1957/1000.jpg" alt="VR Zone Shinjuku" width="1000" height="600" class="gu-image" /> <figcaption> <span class="element-image\_\_caption">Game time … VR Zone in Shinjuku, Tokyo.</span> <span class="element-image\_\_credit">Photograph: Alfred Holmgren for the Guardian</span> </figcaption> </figure>  <p>In the west, a couple of years after the <a href="https://www.theguardian.com/technology/2016/jan/10/oculus-rift-facebook-virtual-reality-headset">Oculus Rift</a> and <a href="https://www.theguardian.com/technology/2016/feb/29/htc-vive-vr-headset-price">HTC Vive</a> headsets became available to the public, VR hype is fast evaporating. Rather than setting the standard for interactive entertainment, the technology has remained a novelty – despite the backing of companies such as Facebook and Sony (whose Playstation VR headset, though relatively successful, has been adopted by <a href="https://www.eurogamer.net/articles/2018-05-23-sony-vr-market-growth-below-expectations">just under 3\%</a> of PlayStation 4 owners).</p> <p>In Japan, on the other hand, the hype only started building after 2016. With interest in consumer headsets nonexistent, “the VR pioneers shifted their focus to applying the tech on a different target group, and under a totally different business model”, says Serkan Toto, CEO of the Tokyo-based analyst firm Kantan Games. In other words: they put all those leftover headsets to use in Japan’s “game centres” instead.</p> <p>When I visit Sega Joypolis on a weekday, there’s barely enough people lined up to fill the eight player slots, but the simulation is impressive enough to turn anyone into a VR evangelist. Before we’re shepherded into the cavernous Zero Latency area, one of my Japanese teammates mentions he has little interest in games that don’t predate the 1986 Famicom. Afterward, he breathlessly announces his intention to buy a PlayStation VR headset.</p> <p>“What arcades are trying is to make VR social,” says Toto. “The goal is to get in multiple customers at once: couples, or a group of friends.”</p>  <figure class="element element-image" data-media-id="6bebc35cd6f60d423da2c7e5ed0507a43bb2dd71"> <img src="https://media.guim.co.uk/6bebc35cd6f60d423da2c7e5ed0507a43bb2dd71/156\_407\_3032\_1818/1000.jpg" alt="Mario Kart VR" width="1000" height="600" class="gu-image" /> <figcaption> <span class="element-image\_\_caption">Crash course … Mario Kart VR.</span> <span class="element-image\_\_credit">Photograph: Alfred Holmgren for the Guardian</span> </figcaption> </figure>  <p>With Japan’s arcade market in sharp decline due to increasingly powerful home consoles – the number of game centres has dwindled from 44,000 to 14,000 in the past three decades – VR doesn’t represent a futuristic new paradigm as much as a return to the industry’s roots. The first Japanese arcades were offshoots of the full-blown amusement parks that used to be located in (or on top of) big department stores back in the 1940s and 50s. It’s no coincidence that one of the most successful arcade games in recent memory, the robot combat sim Kido Senshi Gundam: Senjo no Kizuna, with its cockpit-style controls and massive “panoramic optical display”, is more theme park attraction than video game. It is impossible to replicate in an ordinary Japanese living room.</p> <p>Daisuke Watanabe, a gaming historian at Tokyo’s Meiji University (and one of my co-pilots in the VR Zone Evangelion simulator), traces the roots of today’s VR arcades to the <em>taikan</em> (“physical feedback”) games of the 80s, such as Hang-On and Afterburner, which would place you in the seat of a replica motorcycle or fighter jet. In his view, VR Zone and its ilk are not designed for profit but rather to showcase VR and eventually turn the games into literal money machines for regular arcades.</p> <p>Japan’s gradual adoption of VR has given developers time to come up with some impressive technological solutions. Visiting VR Zone, for example, is nothing like plodding through a VR game on your sofa. Instead you’re strapped into a series of increasingly complex machines; one has you lying on your back, twisting, turning and rumbling, as you’re piloting a mech from Neon Genesis Evangelion. Mario Kart VR puts you in a skeletal go-kart frame that mimics your in-game movements and even uses a wind machine to simulate speed. Others, however, feel more like flimsy tech demos, such as the aptly named Segway simulator Jungle of Despair.</p>  <figure class="element element-image" data-media-id="e183b6548dd58fc4b9013aceff67f976dbc80d6e"> <img src="https://media.guim.co.uk/e183b6548dd58fc4b9013aceff67f976dbc80d6e/0\_0\_3264\_2448/1000.jpg" alt="VR Zone" width="1000" height="750" class="gu-image" /> <figcaption> <span class="element-image\_\_caption">Returning to past glories? … VR Zone.</span> <span class="element-image\_\_credit">Photograph: Alfred Holmgren for the Guardian</span> </figcaption> </figure>  <p>Judging by these early attempts, it’s difficult to tell whether VR is indeed the long-term fix the arcade behemoths have been looking for. Shinjuku’s VR Zone – recently outfitted with “field activities” on the same scale as Joypolis’ Zero Latency – is intended to be the blueprint for more than 20 locations worldwide and “the flagship of next-generation entertainment”. Meanwhile, Adores hails its VR Park – which Watanabe considers the “symbol of profitable VR arcade game market” – as a success that will “revolutionise” the industry.</p> <p>And if this gamble doesn’t pay off in the long run, the arcades can always sell their arsenal of VR equipment to Soft On Demand. When “adult VR” was first demonstrated in Akihabara two years ago, the event was so crowded the organisers called it off for fear of a riot. Clearly, virtual reality has already found at least one home.</p>\\
Technology & <p>The French president Emmanuel Macron has warned a gathering of global tech bosses – including Facebook’s Mark Zuckerberg – that they cannot ride the coattails of the digital economy without giving back to society. </p> <p>Macron told key tech figures at the Elysée palace on Wednesday that they could not just be “free riding” without taking into account the common good. He called on them to help improve “social situations, inequalities, climate change.” </p> <p>“It is not possible just to have free-riding on one side, when you make a good business,” the French president said. </p> <p>He joked: “There is no free lunch” and added that he wanted “commitments”.</p> <p>Fresh from <a href="https://www.theguardian.com/technology/2018/may/22/no-repeat-of-data-scandal-vows-mark-zuckerberg-in-brussels-facebook">apologising to European lawmakers in Brussels</a>, Zuckerberg held an hour of talks with Macron in which his company’s tax policies were believed to be among key topics discussed.</p> <p>Macron, who has long sought to boost technology investment as Paris tries to catch up with London, had invited about 60 key figures from the tech world to an event at the Élysée Palace, including Zuckerberg, Uber’s Dara Khosrowshahi and Microsoft’s Satya Nadella. Macron, 40, who likes to style himself as a champion of the digital economy, will then appear at Paris’s technology fair, VivaTech, which opens on Thursday.</p> <p>Zuckerberg is currently on a Facebook apology tour after the Observer reported that the personal data of tens of millions of people was harvested and shared with the political consultancy <a href="https://www.theguardian.com/news/series/cambridge-analytica-files">Cambridge Analytica</a>. Facebook admitted that the data of <a href="https://www.theguardian.com/technology/2018/apr/04/facebook-cambridge-analytica-user-data-latest-more-than-thought">87 million users</a> may have been improperly shared, including that of 1 million users in the UK.</p> <p>The Facebook founder told MEPs at the European parliament there would be no repeat of the Cambridge Analytica data scandal and fielded accusations that his company had too much power on Tuesday. At the Élysée Palace on Wednesday, Zuckerberg was thought to have had to discuss tax issues as France is trying to make major US internet companies pay more tax.</p> <p>Facebook, along with Google, Apple and Amazon, is in the sights of Macron and some other EU leaders over the use of low-tax countries such as Ireland to reduce corporate tax to nominal levels.</p>  <figure class="element element-image" data-media-id="eb506240739efe2e33b2ff39090a112d60a22ae0"> <img src="https://media.guim.co.uk/eb506240739efe2e33b2ff39090a112d60a22ae0/0\_67\_6084\_3650/1000.jpg" alt="French President Emmanuel Macron meets Microsoft CEO Satya Nadella at the Élysée Palace in Paris" width="1000" height="600" class="gu-image" /> <figcaption> <span class="element-image\_\_caption">French President Emmanuel Macron meets Microsoft CEO Satya Nadella at the Élysée Palace in Paris.</span> <span class="element-image\_\_credit">Photograph: Ludovic Marin/AFP/Getty Images</span> </figcaption> </figure> <aside class="element element-rich-link element--thumbnail"> <p> <span>Related: </span><a href="https://www.theguardian.com/technology/2018/may/22/five-things-we-learned-from-mark-zuckerbergs-european-parliament-appearance">Five things we learned from Mark Zuckerberg's European parliament appearance</a> </p> </aside>  <p>Macron’s office said he was seeking a frank exchange about business and accountability. The French president “is looking to start a dialogue” with tech bosses “to have discussions that will sometimes be frank and direct, to talk about regulation and international governance”, an official told AFP.</p> <p>But the meeting is also part of Macron’s drive to woo tech figures in an attempt to increase digital investment and jobs in France, namely in the area of artificial intelligence.</p> <p>Issues up for discussion with Zuckerberg and others included data protection, fighting hate speech and the battle against fake news. The French government is preparing legislation to <a href="https://www.theguardian.com/world/2018/jan/03/emmanuel-macron-ban-fake-news-french-president">ban fake news online</a> during election periods, including new rules for websites to provide more transparency about sponsored content.<br></p> <p>But Macron is also under pressure to make good on his election campaign promise to ensure US internet companies pay a fair amount of tax.</p> <p>Tech multinationals have come under fire in Europe for using complex fiscal arrangements to declare profits in countries with the lowest tax rates, even when they are earned elsewhere in the EU. </p> <p>During last year’s French presidential election campaign, Macron argued that these low tax rates were a source of resentment about globalisation and unfair for European companies.</p> <p>The European commission <a href="https://www.theguardian.com/business/2018/mar/21/facebook-google-and-amazon-to-pay-fair-tax-under-eu-plans">wants big tech companies to pay a 3\% digital tax</a>, to put them on a level-playing field with their bricks-and-mortar rivals. But the plans are already facing strong opposition from some countries, including Ireland.</p>\\
Opinion & <p>Ideology is what determines\&\#xa0;how you think when you don’t know you’re\&\#xa0;thinking. Neoliberalism is a prime example. Less well-known but equally insidious is <a href="https://www.sciencedirect.com/science/article/pii/B0080430767031673" title="">technological determinism</a>, which is a theory about how technology affects development. It comes in two flavours. One says that there is an inexorable <em>internal</em> logic in how technologies evolve. So, for example, when we got to the point where massive processing power and large\&\#xa0;quantities of data became easily available, machine-learning was an inevitable next step.</p> <p>The second flavour of determinism – the most influential one – takes the form of an unshakable conviction that technology is what really drives history. And it turns out that most of\&\#xa0;us are infected with this version.</p> <p>It manifests itself in many ways. A prime example is the way the political earthquakes of 2016 – Brexit and Trump’s election – are being attributed to technology: if only <a href="https://www.theguardian.com/news/series/cambridge-analytica-files" title="">Cambridge Analytica</a> and other\&\#xa0;dubious actors hadn’t weaponised social media, normal life would have continued. Hillary Clinton would be bombing Syria, David Cameron would still be prime\&\#xa0;minister and Jacob Rees-Mogg would just be muttering into\&\#xa0;his Veuve Clicquot.</p> <p>While there’s no doubt that social media played some – as yet unquantified – role in the upheavals\&\#xa0;of 2016, it seems implausible that the technology was\&\#xa0;the key element. Far more important were populist rage against the 2008 banking crisis – in\&\#xa0;which the wages of bankers’ sin\&\#xa0;were paid for by austerity imposed on ordinary citizens – and the social carnage wrought in some regions of western societies by decades of neoliberal economic policies, globalisation and\&\#xa0;outsourcing.</p> <p>But technological determinism not only colours the way we explain the past, it also affects the way we see the future. Take our current concerns about the impact of the next wave of automation on middle-class employment. In 2013, for example, two Oxford researchers, Carl Benedikt Frey and Michael A Osborne, caused a stir when they published <a href="https://www.oxfordmartin.ox.ac.uk/downloads/academic/The\_Future\_of\_Employment.pdf" title="">a report</a> arguing that “about 47\% of total US employment” was at risk from the kinds of “weak AI” technology then available.</p>  <aside class="element element-pullquote element--supporting"> <blockquote> <p>The “hollowing out” of the middle classes is alarming: a stable middle class seems a prerequisite for a stable democracy</p> </blockquote> </aside>  <p>The main reason the Frey/Osborne study generated such interest was that many of the job categories they identified as vulnerable were middle-class or white-collar. And the prospect of the “hollowing out” of the middle\&\#xa0;classes is alarming because,\&\#xa0;at least up to now, the presence of a stable middle class seems to be a prerequisite for a stable democracy.</p> <p>Closer examination of the study,\&\#xa0;however, suggested that panic might be premature. What the\&\#xa0;researchers had done was to take the 702 job categories employed by the <a href="https://www.bls.gov/" title="">US Bureau of Labor</a> and used machine-learning techniques to estimate the vulnerability of each to automation. Within its own terms of reference, it\&\#xa0;was a pretty good piece of research, and it had the useful side-effect of making policymakers and others sit up and take notice. But as a guide to what might actually happen, it wasn’t much use. Why? Because there’s more than a whiff of technological determinism about it.</p> <p>Although the granularity of the research was an advance on much of what had gone before (702 occupations), it couldn’t address the reality that decisions about what gets automated, and why, are usually local and contingent on lots of <a href="https://www.bloomberg.com/view/articles/2018-02-09/lessons-from-a-slow-motion-robot-takeover" title="">factors which have nothing to do with technology</a>. The fact that a machine could in principle do a particular job doesn’t mean that an entrepreneur or a company will make the necessary investment and\&\#xa0;<a href="https://conversableeconomist.blogspot.co.uk/2018/02/four-examples-from-automation-frontier.html" title="">embark on the disruption implicit</a> in its deployment. Many jobs are in fact bundles of tasks, some of which are\&\#xa0;easily automated, while others are not. And so on.</p> <p>So we need analyses of the potential impact of technology that are less deterministic. As luck would have it, <a href="https://read.oecd-ilibrary.org/employment/automation-skills-use-and-training\_2e2f4eea-en\#page8" title="">one such study</a> has recently emerged from the Organisation for Economic Co-operation and Development (OECD). It finds that across 32 of the 35 countries in the OECD, close to 50\% of jobs are “likely to be significantly affected” by automation. But, the report says, “the degree of risk varies” and only 14\% are “highly automatable” – ie the probability of automation in these jobs is over 70\%. And there are significant regional differences: 33\% of jobs in Slovakia are at risk, for example, but only 6\% of jobs in Norway.</p> <p>This difference highlights the inadequacy of technological determinism in helping us predict the future: in every case, one needs to delve deeper. The explanation for the differences between countries, for example, is probably that across the OECD, jobs that are superficially similar actually involve different occupational mixes.</p> <p>The future of employment will, of\&\#xa0;course, be influenced by technology. But it’s never the only force that shapes our destinies.</p> <h2>What I’m reading</h2> <br> <p><strong>Frozen out of Facebook</strong><br>Facebook’s decision to privilege stuff from friends and family in users’ newsfeeds is upsetting some people. Like <a href="https://www.nytimes.com/2018/04/08/technology/facebook-viral-stars.html?emc=edit\_th\_180409\&amp;nl=todaysheadlines\&amp;nlid=45073630409" title="">the guy who runs clickbait material that goes viral</a>, with titles like “Shocking pit bull social experiment” and “World’s hottest pepper on girlfriend’s thong”.</p> <p><strong>Cosying up to AI</strong><br>Why AI v humans ought not to be a zero-sum game is the subject of <a href="https://jods.mitpress.mit.edu/pub/issue3-case" title="">a thoughtful essay</a> by a teacher called Nicky Case. Personally, I favour AI+human.</p> <p><strong>Jeff, meet Alexa</strong><br><a href="https://www.wired.com/story/amazon-artificial-intelligence-flywheel/" title="">Inside Amazon’s Artificial Intelligence Flywheel</a>, a report by Steven Levy, explains how it was the need to make Alexa work that got Amazon into AI.</p>\\
Technology & <p>Apple has poached Google’s AI chief, John Giannandrea, to run its machine learning and AI operations, in the clearest sign yet that the iPhone creator is attempting to fix the problems that saw its early lead in the field crumble.</p> <p>Scottish-born Giannandrea, who joined Google in 2010 after his startup, Metaweb, was acquired, has led the search firm’s push to become market leader in AI and machine learning. Under his command, Google Brain, the company’s main AI research team, has rebuilt the technology that underpins some of Google’s landmark products, including search, translation and voice recognition.</p> <p>He also led Google into its position today, where it battles with Amazon for technological supremacy in the field of voice controlled assistants. That role was once held by Apple, whose Siri technology introduced the feature to many, but which failed to capitalise on the lead.</p> <p>In March, <a href="https://www.theinformation.com/articles/the-seven-year-itch-how-apples-marriage-to-siri-turned-sour">technology site The Information</a> detailed seven years of infighting within the Siri team at Apple, with multiple attempts to reorganise the basic technology that underpins the feature falling prey to internal politics which limited attempts to improve the overall product.</p> <p>Siri’s problems came to a head in February, when the HomePod – Apple’s attempt to compete with Amazon’s Echo and Google’s Home smart speakers – received reviews which praised it for its audio quality even as they damned it for its poor AI. </p> <p>In recent weeks, however, Apple has accelerated hiring for Siri, peaking with 161 openings posted in one day in March – and now Giannandrea’s hiring, <a href="https://www.nytimes.com/2018/04/03/business/apple-hires-googles-ai-chief.html?login=email">first reported by the New York Times</a>.</p> <p>Beyond talent, though, the company has another issue to overcome: persuading customers that it can build an acceptable set of services without taking the same data-heavy approach favoured by Amazon and Google. </p> <p>Where major technology firms have increased their acquisition of customer data in recent years, arguing that large datasets are crucial for training effective personalised AI, Apple has moved in the opposite direction, <a href="https://www.theguardian.com/technology/2018/mar/29/apple-launches-ios-113-privacy-features-gdpr-data-protection">altering its technologies to gather less personal data about users than it used to</a>.</p> <p>It believes its users understand the value of privacy, and will accept a certain amount of friction in exchange for keeping their secrets secret. And the differing approach has offered chief executive Tim Cook the chance to hit out at competitors. Speaking last week, he said: “We could make a ton of money if we monetised our customers, if our customers were our product … We’ve elected not to do that. We’re not going to traffic in your personal life. Privacy to us is a human right, a civil liberty.”</p>\\
\bottomrule
\end{tabular}}
\end{table}

\rowcolors{2}{white}{white}

\rowcolors{2}{gray!6}{white}

\begin{table}[!h]

\caption{\label{tab:data_collection}Table containing CRIME  News}
\centering
\resizebox{\linewidth}{!}{
\begin{tabular}[t]{ll}
\hiderowcolors
\toprule
sectionName & body\\
\midrule
\showrowcolors
Media & <p>Violence by African Australian youths has been a mainstay of the media silly season over summer, with incidents involving young African men making front-page news and being politicised in spats between the federal and state government.</p> <p> Meanwhile the issue has been politicised thanks to statements by the prime minister, Malcolm Turnbull, and the home affairs minister, Peter Dutton, blaming the state government of Victoria for going soft on the issue.</p> <aside class="element element-rich-link element--thumbnail"> <p> <span>Related: </span><a href="https://www.theguardian.com/media/2018/jan/10/africangangs-social-media-responds-to-melbournes-crisis">\#AfricanGangs: social media responds to Melbourne's 'crisis'</a> </p> </aside>  <p>But <a href="https://www.theguardian.com/australia-news/2018/jan/06/were-not-a-gang-the-pain-of-being-african-australian">how real</a> is the supposed spate of African gang violence in Melbourne’s suburbs? In at least one case, it seems that the media’s reporting is having a toxic observer effect. </p> <p>A scuffle described by the media last week as “the latest gang flare-up” involving African teenagers was in fact entirely provoked by the journalists who reported it, according to Victoria police.</p> <p>The article, published by the Daily Mail on 3 January, was <a href="http://www.dailymail.co.uk/news/article-5231329/African-youths-clash-police-outside-Tarneit-shops.html\#ixzz53lQl4fUy">billed as an exclusive</a> and headlined “Police SPAT ON and abused as officers arrest African teenagers outside a shopping centre in Melbourne’s west in broad daylight – in latest gang flare up”<em>.</em></p>  <figure class="element element-image" data-media-id="dba25a69ec262a943e5f91162e7e3ce929fe2fa2"> <img src="https://media.guim.co.uk/dba25a69ec262a943e5f91162e7e3ce929fe2fa2/0\_0\_2100\_1392/1000.png" alt="An article on the Daily Mail in which they reported on purported gang violence in Melbourne." width="1000" height="663" class="gu-image" /> <figcaption> <span class="element-image\_\_caption">An article on the Daily Mail in which they reported on purported gang violence in Melbourne.</span> <span class="element-image\_\_credit">Photograph: Daily Mail</span> </figcaption> </figure>  <p>But according to Victoria police, there was no “gang” involved and no “flare-up” until the aggressive behaviour of the Daily Mail photographer provoked a group of teenagers who were innocently socialising at the shopping centre.<br></p> <p>Two days after the article was published, the Victoria police executive director of media and corporate communications, Merita Tabain, wrote a confidential email to the editors of Melbourne’s main media outlets expressing concern that aggressive behaviour by journalists might “exacerbate the current tensions”. She gave the incident at the Tarneit Central shopping centre as an example. </p> <p> Tabain wrote that the incident had been provoked by the photographer’s decision to “move in to take closeup photos of a group of African teenagers socialising”.</p> <p> “The teenagers had been doing nothing of public interest prior to the photographer’s decision to move in and take the photos and [the group] reacted to the photographer and what he was doing. </p> <aside class="element element-rich-link element--thumbnail"> <p> <span>Related: </span><a href="https://www.theguardian.com/australia-news/2018/jan/06/were-not-a-gang-the-pain-of-being-african-australian">'We’re not a gang': the unfair stereotyping of African-Australians</a> </p> </aside>  <p>“This led to police being called in and a scuffle ensued in which police were spat on and arrests were made. After the event, the photographer acknowledged that his actions had provoked the incident and apologised.”</p> <p>Yet the article published by the Daily Mail made no reference to this. It did claim that abuse had been directed at the Daily Mail photographer and reporter.</p> <p>The article included a number of closeup shots of police struggling with a youth and talking to others. The text of the article described the “flare-up” as coming amid “fears that a new gang of African youths is rising up in the city’s west” and described it as part of “a spate of violence and crime” in the area.</p> <p>The Mail’s piece also quoted, and included a photograph of, Nelly Yoa, described as a “leading Sudanese youth worker”. Yoa was quoted by a number of media outlets including the Guardian, ABC and News Corp after he contradicted claims from senior police that there was not a problem with African gang violence in a New Year’s Day opinion piece in the Melbourne Age.</p> <p>But now, the Age opinion piece has been augmented with an editorial note saying that several assertions made by Yoa in that article “have been challenged, exaggerated or found not to be true”. The Age article also said parts of the piece were plagiarised, a claim Yoa denies. </p> <p>Needless to say, none of this is regarded as particularly helpful by those on the front line of dealing with crime, including Victorian police. </p> <p>When contacted by the Guardian, Tabain confirmed the authenticity of the email, but declined to provide further comment.<br></p> <p>The email, marked “confidential – not for publication” was sent to the editorial heads of the main media organisations that have been reporting on so called gang violence – the Herald Sun, Macquarie Media, Channel Nine, Channel Seven, Network Ten, Fairfax Media, the ABC, SBS and the Australian. Guardian Australia was not included.</p> <aside class="element element-rich-link element--thumbnail"> <p> <span>Related: </span><a href="https://www.theguardian.com/australia-news/2018/jan/09/victorias-gang-crisis-and-how-the-election-fuels-a-double-standard-on">Victoria's 'gang crisis' and how the election creates a double standard on crime</a> </p> </aside>  <p>“Victoria police does not want to see further incidents such as [the Tarneit incident] and I am therefore respectfully asking that you remind your media teams about the importance of not inflaming situations or inciting conflict, and acting responsibly at all times,” Tabain wrote.</p> <p> Previously, police had called on the state’s media to stop referring to African youths behind recent riots and vandalism as “gangs”. The deputy police commissioner, Andrew Crisp, told 3AW the week before that there was no evidence that gangs were responsible for a riot in Werribee. The police commissioner, Graham Ashton, has rubbished Dutton’s claims that Victorians are too scared to go out at night.</p> <p>The Daily Mail was contacted for comment but had not responded by deadline.</p>\\
UK news & <p>Two-thirds of Greater Manchester residents have experienced hateful behaviour on the grounds of characteristics such as race, religion and gender, <a href="https://www.greatermanchester-ca.gov.uk/downloads/download/194/a\_shared\_future">according to a report commissioned in the aftermath of the Manchester Arena bomb</a>.</p> <p>The report, commissioned by the region’s mayor, Andy Burnham, after the attack last year, asked an independent group of experts to consider how to tackle hateful extremism, social exclusion and radicalisation across Greater Manchester.</p> <p>In a survey conducted for the report, 65\% of respondents reported being a victim of hateful behaviour. Proportionately, hate crime based on ethnicity was the most frequently reported, with 33\% of all respondents saying that they had experienced it. Of this group 16\% said that they experienced hate crime related to their ethnicity on a “frequent” basis.</p> <p>The report found that a third of people in the region did not think their local area was a place where people from different backgrounds got on together, and warned of “a dangerous perpetuating cycle of fear” about Prevent, the government’s anti-extremism programme, which “is negatively affecting cohesion in communities across Greater Manchester”.</p> <p>It also found that the social and economic inequalities that exist across Greater Manchester were likely to have a negative effect on social cohesion and may have an impact on the risk of radicalisation. The region has <a href="http://blog.policy.manchester.ac.uk/posts/2016/10/life-on-the-line-life-expectancy-and-where-we-live/">some of the most deprived</a> wards in the country just a few miles away from some of the most affluent.</p> <p>In the weeks after the bombing at the Ariana Grande concert, despite widespread headlines about the “Greater Manchester spirit” and people getting matching tattoos of the Mancunian worker bee, Greater Manchester police reported a 130\% rise in hate crime, including a 500\% rise in Islamophobic hate crime. Questions were also asked how Salman Abedi, the 22-year-old suicide bomber who carried out the attack, had become so radicalised despite growing up in south Manchester and going to state school and college.</p> <p>Abedi had not been referred to Prevent, though MI5 had received information about him on two separate occasions before the 22 May attack. The information was not acted on because the reports were believed to be of a criminal nature, and not related to terrorism.</p> <p>The report blames reductions in public services for increasing isolation in communities. “Feedback suggests there is now little opportunity for people from both similar and different backgrounds to meet naturally and have conversations. This is likely to have exacerbated fear and suspicion of different communities,” the authors write.</p> <p>The authors, who include Nazir Afzal, a former chief crown prosecutor for the north-west of England, commissioned a survey of 1,609 Greater Manchester residents to find out how they felt about their communities.</p> <p>Asked “to what extent do you agree or disagree that your local area is a place where people from different backgrounds get on well together?”, 34\% said they disagreed or strongly disagreed.</p> <p>Thirty-two per cent of respondents said they had direct experience with people “spreading hate or extremist views”. Yet the research found many people did not feel comfortable reporting hate speech.</p> <p>“The counter-terrorism hotline and 999/101 were seen as too formal, and people stated that they would not engage with these services unless they were sure their concerns were right, meaning valuable time may be lost to intervene if someone is being radicalised,” the authors write.</p> <p>“Many people who had concerns about a friend or family member would not link their concerns to radicalisation and wanted someone that they could discuss their concerns with informally. People would be more willing to speak to local neighbourhood police officers, but due to funding cuts these teams are very much depleted.”</p> <p>The report suggests more work needs to be done to disseminate the “positive work” being done by Prevent in communities “where high levels of distrust and suspicion of statutory agencies continues to exist … There is a dangerous perpetuating cycle of fear of Prevent and a lack of communication about Prevent that is negatively affecting cohesion in communities across Greater Manchester.”</p> <p>Whole families should be involved in countering extremist ideology when it takes hold, the authors write. The report refers to “Ahmed”, a 13-year-old boy who was referred to Prevent after making complimentary remarks about the Arena bomber at school. Social workers found evidence of domestic violence at home and the family were referred to <a href="http://www.gmp.police.uk/live/nhoodv3.nsf/WebsitePages/C1DDE8142DE605FC80257EF40036802F?OpenDocument">Greater Manchester’s Strive</a><a href="http://www.gmp.police.uk/live/nhoodv3.nsf/WebsitePages/C1DDE8142DE605FC80257EF40036802F?OpenDocument"> programme</a>, which aims to raise awareness about how abusive behaviours can affect family members.</p> <p>Responding to the report, Burnham said: “We live in polarised times when violent extremism is on the rise in all communities. As the commission concludes, it is families, friends and neighbours who are most likely to be the first to witness changes towards more extreme behaviour that might lead to violence. This is not about encouraging people to spy on each other but creating a greater understanding of the signs that indicate where behaviour has crossed the line and then making it easier to report.</p> <p>“Of course, the primary responsibility for tackling terrorism will always lie with the police and security services. They have our full support and they have done great work since the attack to bring those responsible to justice and reassure our communities. But there is more that we can all do to help them. The more we adopt a ‘whole-society’ approach to tackling extremism, the more effective it will be.”</p>\\
Culture & <p>Late-night hosts on Tuesday discussed Paul Manafort’s trial and Donald Trump’s relationship with North Korea and Iran.</p> <aside class="element element-rich-link element--thumbnail"> <p> <span>Related: </span><a href="https://www.theguardian.com/culture/2018/jul/31/jimmy-kimmel-rudy-giuliani-rabies-late-night">Jimmy Kimmel: 'I think Rudy Giuliani might have rabies'</a> </p> </aside>  <p>“I feel like it’s Christmas morning because all year long Robert Mueller and his team of legal elves have been busy in their workshops making all the indictments for the bad little boys and girls,” began Stephen Colbert. “And the magical day we’ve been waiting for is finally here.”<br></p> <p>Noting that Manafort has been accused of tax evasion and bank fraud, among other crimes, Colbert said that “prosecutors painted a picture of Manafort’s extravagant spending, telling the jury that Manafort once got a \$15,000 coat made from an ostrich.”</p> <p>“Manafort’s team,” the host continued, “shifted the blame in their defense to the Ukrainian oligarchs that Manafort worked for, explaining that, yes, Manafort was paid through secret foreign accounts, but it wasn’t his idea!”<br></p>       <figure class="element element-video" data-canonical-url="http://www.youtube.com/watch?v=8GQSOuXtnL8"                                                                        >  <iframe height="259" width="460" src="https://www.youtube.com/embed/8GQSOuXtnL8?wmode=opaque\&feature=oembed" frameborder="0" allowfullscreen ></iframe>  </figure>   <p>“Now, Manafort, he’s a tough guy. But I’m sure he’s a little worried,” Colbert said. “One person who is apparently not worried about Mueller’s investigation is Donald Trump.”</p> <p>The Trump legal defense team, the host said, have increasingly resorted to saying collusion is not a crime at all. The host then read aloud a <a href="https://twitter.com/realDonaldTrump/status/1024263146008207361">tweet</a> from president Trump in which he wrote: “Collusion is not a crime, but that doesn’t matter because there was No Collusion (except by Crooked Hillary and the Democrats)!”</p> <p>“Okay, so collusion isn’t a crime, but it doesn’t matter because he didn’t do it anyway, Hillary did,” said Colbert. “It’s really going to complicate the chants at his rallies. Lock her up, but collusion’s not a crime. So what are we locking her up for? I am confused.”<br></p> <p>Meanwhile, The Daily Show’s Trevor Noah did a rapid-fire reading of the week’s news.</p> <p>“I’m going to be honest: there is a lot of news,” he began. “Almost too much news. Luckily, though, too much news is just the right amount of news for a segment we call, ‘Ain’t nobody got time for that.’”<br></p> <p>Noah began by discussing Zimbabwe’s presidential elections, the first since Robert Mugabe was ousted in November. He followed that with news of LeBron James’ new elementary school in his hometown of Akron, Ohio, where 240 third and fourth graders will be attending.</p> <p>“But we don’t have the time to talk about all that because the situation between Iran and the US just took a unexpected turn,” Noah said, noting that after ripping up the Iran deal and threatening Iranian president Rouhani on Twitter, Trump is now willing to meet with him with no preconditions.<br></p>  <figure class="element element-tweet" data-canonical-url="https://twitter.com/TheDailyShow/status/1024485871033053184">  <blockquote class="twitter-tweet"><p lang="en" dir="ltr">Beyoncé’s Vogue cover control, Zimbabwe’s first post-Mugabe elections, Lebron’s I Promise School, and North Korea’s new missile development. So much news, so little time! <a href="https://t.co/TSBxvmQTym">pic.twitter.com/TSBxvmQTym</a></p>\&mdash; The Daily Show (@TheDailyShow) <a href="https://twitter.com/TheDailyShow/status/1024485871033053184?ref\_src=twsrc\%5Etfw">August 1, 2018</a></blockquote>  </figure>  <p>“Wow, Iran,” the host said. “I know Iran seems extreme but I get why they’re not eager to meet with Trump. You got to admit, he blows hot and cold like one of those psycho boyfriends.”<br></p> <p>Mocking Trump, Noah went on: “Iran, you will suffer consequences the likes of which few throughout history have suffered before. I’m sorry, Iran. I haven’t had my Big Mac today. Can we meet?”</p> <p>“I understand why Iran would think a meeting with Trump would be unproductive,” Noah added. “But if they took a page out of North Korea’s book, Iran might be able to use that to their advantage.”<br></p> <p>The host then explained that, after promising to halt its nuclear activity, US intelligence suspects the North Korean governments is working on construction of one or possibly two liquid-fueled ICBM’s on the outskirts of Pyongyang. </p> <p>“So, Kim Jong-un made a promise to Trump and then did the opposite,” Noah concluded. “Which basically means Trump is getting a taste of his own medicine.”<br></p> <p>Jimmy Kimmel also addressed Kim Jong-un’s about-face.</p> <p>“You know how president Trump said North Korea is no longer a nuclear threat?,” he asked. “Well, oops.”</p>       <figure class="element element-video" data-canonical-url="http://www.youtube.com/watch?v=Gfo6HDEZjO0"                                                                        >  <iframe height="259" width="460" src="https://www.youtube.com/embed/Gfo6HDEZjO0?wmode=opaque\&feature=oembed" frameborder="0" allowfullscreen ></iframe>  </figure>   <p>“According to US intelligence, not only isn’t North Korea curtailing their nuclear activity, they may be building new intercontinental ballistic missiles,” said Kimmel.<br></p> <p>“Kim Jong-un, as you may recall, said he would not build any more missiles,” the host continued. “But it turns out in Korean ‘would not’ means ‘would’, just like in Trumpian.”<br></p> <aside class="element element-rich-link element--thumbnail"> <p> <span>Related: </span><a href="https://www.theguardian.com/world/2018/jul/31/north-korea-us-detects-new-activity-icbm-factory">North Korea: US 'detects new activity' at ICBM factory</a> </p> </aside>  <p>“Absolutely shocking,” Kimmel added sarcastically. “I mean, Kim Jong-un gave Donald Trump his word. They shook hands on it.”<br></p> <p>“It’s like, who can you trust anymore?” Kimmel joked.</p>\\
Television \& radio & <h2>Picks of the week</h2> <p><strong><a href="https://wondery.com/shows/dr-death/">Dr Death</a></strong></p> <p>The makers of last year’s disturbing hit <a href="https://www.theguardian.com/tv-and-radio/2017/oct/13/dirty-john-your-chilling-new-true-obsession-podcasts-of-the-week">Dirty John</a> have come up with another dark podcast about a dangerous doctor. Laura Beil tells the story of spinal surgeon Dr Christopher Duntsch, who botched operations and killed two people. The six-parter unfolds quickly, from tiny clues such as surgical scrubs that remain unchanged to the doctor going missing mid-operation. Add the mental image of a screw being inserted into a spinal canal, plus sounds effects and it’s shaping up to be even more horrifying than Dirty John. <em><strong>HV</strong></em></p> <p><strong><a href="https://serialpodcast.org/">Serial</a></strong></p> <p>The podcasting blockbuster returns on 20 September, and for season three, queen of storytelling Sarah Koenig turns her attention to Cleveland’s criminal court system. She and Emmanuelle Dzotsi spent a year observing cases and their time delivers a rich variety of crimes, set in a world where there’s “pressure to plead … overworked attorneys, dozing jurors”. There are no big mysteries this time around, but what seem like ordinary cases illuminate the way the justice system works and where it fails. <em><strong>HV</strong></em></p> <p><strong><a href="http://www.bbc.co.uk/programmes/articles/2c8fh5jZhsNWbb5JLNL3gc1/a-chance-to-get-your-creativity-on-radio-4">The Art of Now: Inbox</a></strong></p> <p>The actor and comedian Jo Neary helms a woozy experiment, patched together using listeners’ home recordings: sounds, music, poems, serious musings and comic monologues, most of them captured by pressing Record on a smartphone and all of them tending towards an air of bemused introspection and a delight in oddness. As a format, it is confused by including occasional contributions from professional comedians but, then again, pleasant befuddlement is the aim. <em><strong>JS</strong></em></p> <h2><strong>Your picks</strong></h2>  <figure class="element element-image" data-media-id="45b53ae56acdd4fc49128fa9f3d5741c124664ff"> <img src="https://media.guim.co.uk/45b53ae56acdd4fc49128fa9f3d5741c124664ff/0\_28\_931\_559/500.png" alt="Oh No! Lit Class" width="500" height="300" class="gu-image" /> <figcaption> <span class="element-image\_\_caption">Oh No! Lit Class</span> <span class="element-image\_\_credit">Photograph: Megan Hesse</span> </figcaption> </figure>  <p><strong><a href="http://www.braintrustbros.com/shows/oh-no-lit-class/">Oh No! Lit Class</a></strong></p> <blockquote class="quoted"> <p>It’s like listening to your two funniest friends who just happen to be having a brainy conversation about literature. They are actively democratising academia by breaking down these books and providing a fascinating historical context in an approachable and hysterical way.</p> </blockquote> <p><em>Recommended by Matt Sanderson</em></p> <p><strong><a href="https://www.philips.com/a-w/careers/the-spark-podcast.html">The Spark</a></strong></p> <blockquote class="quoted"> <p>The Spark explores the mind at work through the personal stories of Philips employees. My favourite episode was The Runaway Intellect, which features someone telling his story of fleeing war-torn Angola and starting a new life and a new job.</p> </blockquote> <p><em>Recommend by Adolfo Cazanga</em></p> <h2>Guardian pick: <strong><a href="https://www.theguardian.com/society/series/beyondtheblade">Beyond the Blade</a></strong></h2>  <figure class="element element-image" data-media-id="071534dcb366037afa09fca48a104748e7e28202"> <img src="https://media.guim.co.uk/071534dcb366037afa09fca48a104748e7e28202/0\_0\_2585\_1551/1000.jpg" alt="Beyond the Blade" width="1000" height="600" class="gu-image" /> <figcaption> <span class="element-image\_\_caption">Beyond the Blade</span> <span class="element-image\_\_credit">Photograph: Guardian Design Team</span> </figcaption> </figure>  <p>A vital, creative response to the award-winning Beyond the Blade series led by Gary Younge. From 2017, we have been<a href="https://www.theguardian.com/membership/series/beyond-the-blade"> investigating </a>the impact of knife crime upon Britain’s young people and exposing the myths that surround it with the Beyond the Blade project. The story of knife crime among the young isn’t one of angels and demons. It’s a story of kids, some of whom have made bad choices. Choices that may have left other kids just like them dead, and families and communities to mourn. Trying to understand why they made the choices they did is not an indulgence – it’s a necessity.</p> <p>In this new <a href="https://www.theguardian.com/society/series/beyondtheblade">series of podcasts</a>, we visited groups of parents, key support workers and young people from around the country - real people doing real work across the country to turn back the tide. They have allowed us to listen to, and record, conversations they’re having within their own communities about knife crime. Rather than report on these conversations, we let people speak for themselves.</p> <h2>If you’ve got a podcast that you love, send your recommendations to <a href="mailto:podcasts@theguardian.com">podcasts@theguardian.com</a></h2> <p><em><br>Your picks are compiled by Rowan Slaney</em></p>\\
Film & <p>American History X director Tony Kaye is hoping to cast an artificial intelligence actor as the lead of his new film.</p> <p>According to <a href="https://deadline.com/2018/08/filmmaker-tony-kaye-robot-lead-actor-2nd-born-1202445941/">Deadline</a>, the British film-maker has made the unprecedented decision to employ a robot over a human for his next project, titled 2nd Born. The android will be trained in various techniques and a variety of acting methods and Kaye hopes it will lead to recognition by the Screen Actors Guild which could also lead to awards consideration.</p> <p>The idea is a joint effort from Kaye and producer Sam Khoze.</p> <aside class="element element-rich-link element--thumbnail"> <p> <span>Related: </span><a href="https://www.theguardian.com/film/2012/jul/06/detachment-tony-kaye-interview">Tony Kaye: 'I hope I'm having a moment now'</a> </p> </aside>  <p>The film is a sequel to comedy 1st Born starring Val Kilmer and Denise Richards, the first ever Iranian-American co-production. Set to be released this year, it tells the story of a newlywed couple who encounter problems conceiving.</p> <p>Kaye’s career has included a combination of music videos for artists such as Johnny Cash and the Red Hot Chili Peppers and a string of feature films. His directorial debut was 1998’s American History X starring Edward Norton but Kaye asked for his name to be taken off the credits after battles over the final cut.</p> <p>Since then he has directed the acclaimed abortion documentary Lake of Fire and school-set drama Detachment. He was recently attached to direct <a draggable="true" href="https://variety.com/2018/film/news/tony-kaye-direct-crime-drama-honourable-men-1202841026/">Honorable Man</a>, a crime thriller set in 70s New York and is reportedly working on a \#MeToo-era film called Hollywhore.</p> <p>“I don’t want any more confrontations,” he told <a draggable="true" href="https://www.thedrum.com/news/2018/04/17/director-tony-kaye-happy-be-working-and-out-hollywood-jail">the Drum</a> in April. “I’m not looking for any more fights. I’m not looking to win. I’m just looking to be a part of the process and to work. I don’t consider that to be a defeatist attitude. I consider that to be a very realistic and practical methodology, to get better. And every now and again I’m going to find that pocket of work where it just gels. The time that the work gets really good is when the stars line up.”</p>\\
Television \& radio & <h2>Alex</h2> <p><strong>11.05pm, Channel 4</strong><br>Can you ever escape your past? In a familiar crime drama set-up, made mildly more exotic by being Swedish, dirty cop Alex Leko certainly hopes so as he vows to go straight after accidentally killing his partner and best buddy. But there are complications. Not only is Leko’s new partner, Frida Kanto, trying to build a case against him, but local crime boss BG isn’t keen to lose Leko’s services. The full series, from the Walter Presents strand, will be available to stream from All 4 after this episode has premiered. <em>Jonathan Wright</em></p> <h2>Bad Move</h2> <p><strong>8pm, ITV</strong><br>The caustic rural sitcom grumbles on with hapless urbanites Nicky (Kerry Godliman) and Steve (Jack Dee) still trapped in a clapped-out country cottage. With the septic tank in pungent revolt, a new web-design contract seems to offer a financial lifeline, so long as they can hoodwink the glitzy client. <em>Graeme Virtue</em></p> <h2>Upstart Crow </h2> <p><strong>8.30pm, BBC Two</strong><br>How do you sneak a play about the assassination of an emperor past Elizabeth I, a monarch not known to favour entertainments involving the ending of heads of state? Enjoy what happens when Will decides to write a play about Julius Caesar, prompting dramatist and snake Robert Greene to label him a traitor. <em>Mike Bradley</em></p> <h2>Trust</h2> <p><strong>9pm, BBC Two</strong><br>Things go awry in the wake of J Paul Getty’s refusal to pay “a single, solitary cent” to his grandson’s captors, believing the kidnap was staged by the victim to extract money from him. Donald Sutherland’s absence is keenly felt in an episode set almost entirely in Italy with too much backstory. Still worth sticking with. <em>MB</em></p> <h2>Grand Designs </h2> <p>9pm, Channel 4<br>Who would have dreamed of recreating the American modernist house from 1986 movie Ferris Bueller’s Day Off in a valley in Cornwall? Superfans Harry and Briony Anscombe, that’s who. The pair have paid £490,000 for the plot and they are hoping to spend £400,000, tops, on the build. With a budget like that, they’ll never stretch to a red Ferrari like Cameron’s dad’s. <em>MB</em></p> <h2>The National Lottery Awards 2018 </h2> <p><strong>10.45pm, BBC One</strong><br>Not a ceremony celebrating the lottery itself – although eight, 19 and 37 have really excelled themselves this year – but a chance to learn about and laud the charity projects the lottery funds. Strictly champion Ore Oduba hosts, with recently retired boxer David Haye and actors Meera Syal and Michael Sheen among the star presenters. <em>Jack Seale</em></p> <h2>Film choice</h2>  <figure class="element element-image" data-media-id="7daddef663cb837544f60ee47e3b341cc77841f7"> <img src="https://media.guim.co.uk/7daddef663cb837544f60ee47e3b341cc77841f7/4\_0\_1430\_858/1000.jpg" alt="Daniela Vega in A Fantastic Woman." width="1000" height="600" class="gu-image" /> <figcaption> <span class="element-image\_\_caption">Daniela Vega in A Fantastic Woman.</span> <span class="element-image\_\_credit">Photograph: AP</span> </figcaption> </figure>  <p><strong>A Fantastic Woman (Sebastián Lelio, 2017) 1.55pm</strong><strong>\&amp; 9.45pm, Sky Cinema Premiere</strong><br>Marina is a transgender singer in Santiago, in a loving relationship with divorcee Orlando (Francisco Reyes). When he dies, she is suspected of murder and subjected to grotesque prejudice from police, officials and the dead man’s family. Daniela Vega is, indeed, fantastic as Marina. <em>Paul Howlett</em></p> <h2>Live sport</h2> <p><strong>Cycling: Road World Championships </strong>1pm, Eurosport 1. The men’s individual time trial.</p> <p><strong>Betfred Cup football: St Johnstone v Celtic </strong>7.15pm, BT Sport 1. A quarter-final from McDiarmid Park, Perth.</p> <p><strong>Carabao Cup football: Liverpool v Chelsea </strong>7.30pm, Sky Sports Main Event. The pick of the third-round ties.</p>\\
\bottomrule
\end{tabular}}
\end{table}

\rowcolors{2}{white}{white}

\newpage

\newpage

\begin{Shaded}
\begin{Highlighting}[]
\NormalTok{print_data =}\StringTok{ }\ControlFlowTok{function}\NormalTok{(query,sum_data_count)\{}
  \KeywordTok{print}\NormalTok{(}\KeywordTok{paste}\NormalTok{(}\StringTok{"Number of rows in "}\NormalTok{,}\KeywordTok{toupper}\NormalTok{(query),}\StringTok{" queries: "}\NormalTok{,sum_data_count))}
  \KeywordTok{cat}\NormalTok{(}\StringTok{"}\CharTok{\textbackslash{}n}\StringTok{"}\NormalTok{)}
\NormalTok{\}}
\KeywordTok{mapply}\NormalTok{(print_data,}\DataTypeTok{query =}\NormalTok{ queries,}\DataTypeTok{sum_data_count =}\NormalTok{ sum_data_count)}
\end{Highlighting}
\end{Shaded}

\begin{verbatim}
## [1] "Number of rows in  BUSINESS  queries:  1000"
## 
## [1] "Number of rows in  SPORTS  queries:  1000"
## 
## [1] "Number of rows in  ENTERTAINMENT  queries:  1000"
## 
## [1] "Number of rows in  ECONOMY  queries:  1000"
## 
## [1] "Number of rows in  POLITICS  queries:  1000"
## 
## [1] "Number of rows in  SCIENCE  queries:  1000"
## 
## [1] "Number of rows in  HEALTH  queries:  1000"
## 
## [1] "Number of rows in  ART  queries:  1000"
## 
## [1] "Number of rows in  TECHNOLOGY  queries:  1000"
## 
## [1] "Number of rows in  CRIME  queries:  1000"
\end{verbatim}

\begin{verbatim}
## $business
## NULL
## 
## $sports
## NULL
## 
## $entertainment
## NULL
## 
## $economy
## NULL
## 
## $politics
## NULL
## 
## $science
## NULL
## 
## $health
## NULL
## 
## $art
## NULL
## 
## $technology
## NULL
## 
## $crime
## NULL
\end{verbatim}

\begin{Shaded}
\begin{Highlighting}[]
\KeywordTok{cat}\NormalTok{(}\StringTok{"}\CharTok{\textbackslash{}\textbackslash{}}\StringTok{pagebreak"}\NormalTok{)}
\end{Highlighting}
\end{Shaded}

\begin{verbatim}
## \pagebreak
\end{verbatim}

\begin{Shaded}
\begin{Highlighting}[]
\KeywordTok{print}\NormalTok{(}\KeywordTok{paste}\NormalTok{(}\StringTok{"Total Number of rows in News: "}\NormalTok{,}\KeywordTok{nrow}\NormalTok{(news_data)))}
\end{Highlighting}
\end{Shaded}

\begin{verbatim}
## [1] "Total Number of rows in News:  10000"
\end{verbatim}

\begin{Shaded}
\begin{Highlighting}[]
\KeywordTok{cat}\NormalTok{(}\StringTok{"}\CharTok{\textbackslash{}n}\StringTok{"}\NormalTok{)}
\end{Highlighting}
\end{Shaded}

\begin{Shaded}
\begin{Highlighting}[]
\KeywordTok{summary}\NormalTok{(news_data)}
\end{Highlighting}
\end{Shaded}

\begin{verbatim}
##       id                type            sectionId        
##  Length:10000       Length:10000       Length:10000      
##  Class :character   Class :character   Class :character  
##  Mode  :character   Mode  :character   Mode  :character  
##  sectionName        webPublicationDate   webTitle        
##  Length:10000       Length:10000       Length:10000      
##  Class :character   Class :character   Class :character  
##  Mode  :character   Mode  :character   Mode  :character  
##     webUrl             apiUrl           isHosted         pillarId        
##  Length:10000       Length:10000       Mode :logical   Length:10000      
##  Class :character   Class :character   FALSE:10000     Class :character  
##  Mode  :character   Mode  :character                   Mode  :character  
##   pillarName            body          
##  Length:10000       Length:10000      
##  Class :character   Class :character  
##  Mode  :character   Mode  :character
\end{verbatim}

\subsection{Data Cleaning}\label{data-cleaning}

\begin{Shaded}
\begin{Highlighting}[]
\KeywordTok{library}\NormalTok{(tidyverse)}
\end{Highlighting}
\end{Shaded}

\begin{verbatim}
## -- Attaching packages --------------------------------------- tidyverse 1.2.1 --
\end{verbatim}

\begin{verbatim}
## v ggplot2 3.0.0     v readr   1.1.1
## v tibble  1.4.2     v purrr   0.2.5
## v tidyr   0.8.1     v stringr 1.3.1
## v ggplot2 3.0.0     v forcats 0.3.0
\end{verbatim}

\begin{verbatim}
## -- Conflicts ------------------------------------------ tidyverse_conflicts() --
## x dplyr::filter()  masks stats::filter()
## x purrr::flatten() masks jsonlite::flatten()
## x dplyr::lag()     masks stats::lag()
\end{verbatim}

\begin{Shaded}
\begin{Highlighting}[]
\KeywordTok{library}\NormalTok{(reshape)}
\end{Highlighting}
\end{Shaded}

\begin{verbatim}
## 
## Attaching package: 'reshape'
\end{verbatim}

\begin{verbatim}
## The following objects are masked from 'package:tidyr':
## 
##     expand, smiths
\end{verbatim}

\begin{verbatim}
## The following object is masked from 'package:dplyr':
## 
##     rename
\end{verbatim}

\begin{Shaded}
\begin{Highlighting}[]
\KeywordTok{library}\NormalTok{(stringr)}
\KeywordTok{library}\NormalTok{(qdap)}
\end{Highlighting}
\end{Shaded}

\begin{verbatim}
## Loading required package: qdapDictionaries
\end{verbatim}

\begin{verbatim}
## Loading required package: qdapRegex
\end{verbatim}

\begin{verbatim}
## 
## Attaching package: 'qdapRegex'
\end{verbatim}

\begin{verbatim}
## The following object is masked from 'package:ggplot2':
## 
##     %+%
\end{verbatim}

\begin{verbatim}
## The following object is masked from 'package:dplyr':
## 
##     explain
\end{verbatim}

\begin{verbatim}
## The following object is masked from 'package:jsonlite':
## 
##     validate
\end{verbatim}

\begin{verbatim}
## Loading required package: qdapTools
\end{verbatim}

\begin{verbatim}
## 
## Attaching package: 'qdapTools'
\end{verbatim}

\begin{verbatim}
## The following object is masked from 'package:dplyr':
## 
##     id
\end{verbatim}

\begin{verbatim}
## Loading required package: RColorBrewer
\end{verbatim}

\begin{verbatim}
## 
## Attaching package: 'qdap'
\end{verbatim}

\begin{verbatim}
## The following object is masked from 'package:reshape':
## 
##     condense
\end{verbatim}

\begin{verbatim}
## The following object is masked from 'package:forcats':
## 
##     %>%
\end{verbatim}

\begin{verbatim}
## The following object is masked from 'package:stringr':
## 
##     %>%
\end{verbatim}

\begin{verbatim}
## The following object is masked from 'package:purrr':
## 
##     %>%
\end{verbatim}

\begin{verbatim}
## The following object is masked from 'package:tidyr':
## 
##     %>%
\end{verbatim}

\begin{verbatim}
## The following object is masked from 'package:kableExtra':
## 
##     %>%
\end{verbatim}

\begin{verbatim}
## The following object is masked from 'package:dplyr':
## 
##     %>%
\end{verbatim}

\begin{verbatim}
## The following object is masked from 'package:base':
## 
##     Filter
\end{verbatim}

\begin{Shaded}
\begin{Highlighting}[]
\CommentTok{# print("Columns in the News dataframe:")}
\CommentTok{# names(news_data)}

\CommentTok{# kable(head(news_data), format = "latex", booktabs = T,}
\CommentTok{#           caption=paste("Table containing Unclean News data")) %>%}
\CommentTok{#       kable_styling(latex_options = c("striped","hold_position","scale_down"))}

\NormalTok{news =}\StringTok{ }\NormalTok{news_data }\OperatorTok
\StringTok{      }\KeywordTok{select}\NormalTok{ (}\KeywordTok{c}\NormalTok{(}\StringTok{"id"}\NormalTok{, }\StringTok{"pillarName"}\NormalTok{, }\StringTok{"sectionName"}\NormalTok{, }\StringTok{"webTitle"}\NormalTok{, }\StringTok{"body"}\NormalTok{))}\OperatorTok
\StringTok{      }\KeywordTok{rename}\NormalTok{(}
        \KeywordTok{c}\NormalTok{(}
          \StringTok{'pillarName'}\NormalTok{ =}\StringTok{ 'type'}\NormalTok{,}
          \StringTok{'sectionName'}\NormalTok{ =}\StringTok{ 'section'}\NormalTok{,}
          \StringTok{'webTitle'}\NormalTok{ =}\StringTok{ 'title'}
\NormalTok{        )}
\NormalTok{      )}\OperatorTok
\StringTok{      }\KeywordTok{na.omit}\NormalTok{()}

\NormalTok{clean_function =}\StringTok{ }\ControlFlowTok{function}\NormalTok{(data)\{}
\NormalTok{   data =}\StringTok{ }\KeywordTok{str_replace_all}\NormalTok{(data,}\StringTok{"<.*?>"}\NormalTok{, }\StringTok{" "}\NormalTok{)}
\NormalTok{   data =}\StringTok{ }\KeywordTok{replace_contraction}\NormalTok{(data, }\DataTypeTok{contraction =}\NormalTok{ qdapDictionaries}\OperatorTok{::}\NormalTok{contractions)}
\NormalTok{   data =}\StringTok{ }\KeywordTok{str_replace_all}\NormalTok{(data,}\StringTok{"[^a-zA-Z}\CharTok{\textbackslash{}\textbackslash{}}\StringTok{s]"}\NormalTok{, }\StringTok{" "}\NormalTok{)}
\NormalTok{   data =}\StringTok{ }\KeywordTok{str_replace_all}\NormalTok{(data,}\StringTok{"[}\CharTok{\textbackslash{}\textbackslash{}}\StringTok{s]+"}\NormalTok{, }\StringTok{" "}\NormalTok{)}
\NormalTok{   data =}\StringTok{ }\KeywordTok{tolower}\NormalTok{(data)}
\NormalTok{\}}

\NormalTok{news}\OperatorTok{$}\NormalTok{body =}\StringTok{ }\KeywordTok{clean_function}\NormalTok{(news}\OperatorTok{$}\NormalTok{body)}

\CommentTok{#news$body = gsub("<.*?>", " ", news$body)       #To remove all HTML tags}

\CommentTok{#news$body = gsub("[^a-zA-Z]"," ",news$body)}
\CommentTok{#news$body = gsub("[[:punct:][:blank:]]"," ",news$body)}
\CommentTok{#news$body = gsub("[/[^A-Za-z0-9\textbackslash{}s]/i]"," ",news$body)     #To remove all characters which are not alphabets}
\CommentTok{#news$title = gsub("[^a-zA-Z]"," ",news$title)}



\CommentTok{#news$body =  tolower(news$body)}
\CommentTok{#news$title = tolower(news$title)}



\KeywordTok{View}\NormalTok{(news)}
\end{Highlighting}
\end{Shaded}


\end{document}
